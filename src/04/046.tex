\section{Subgradiants - Revisited via MC/MC}
\label{sect:046}

\paragraph{}We will revist subgradients via MC/MC. Fix a proper convex function $f$ on $\mathbb{R}^m$, write $X=\operatorname{dom}(f)$.

\begin{defn}[Subgradients]\label{defn-subgradients}
	For $x\in X$, define $\operatorname{epi}(f)(x) = \operatorname{epi}(f) - (x,f(x))$. Then we have:
	\[
		\partial f(x) = \left\{
		y\in \mathbb{R}^m,
		\operatorname{epi}(f)(x)\subset E_{(y,-1),0,-}
		\right\}
	\]
	as the "subgradient of $f$ at $x$". Note: since $w^\ast_{\operatorname{epi}(f)(x)}=0$, $\partial f(x)=Q^\ast_{\operatorname{epi}(f)(x)}$ (weak duality). Therefore:
	\begin{align*}
		\partial f(x) & =\left\{y:\inf\left\{(-y,1)^T \operatorname{epi}(f)(x)\right\}=0\right\}                                       \\
		              & =\left\{y:\sup_z \left\{ z^Ty-f(z)\right\}=x^Ty-f(x)\right\}            =\left\{y:f^\star(y)=x^Ty-f(x)\right\}
	\end{align*}
\end{defn}

\begin{prop}[Decomposition of Set of Subgradients]\label{prop:046-set-of-subgrad}
	Write $V=\operatorname{lin}(X)$. For $x\in \operatorname{ri}(X)$, $\partial f(x) = V^\perp + V\cap \partial f(x)$ with $V\cap \partial f(x)\neq\emptyset$ compact and convex. In particular, if $x\in \operatorname{int}(X)$, $\partial f(x)$ is compact.
\end{prop}
\begin{proof}
	By \Cref{prop:032-optimal-set}.
\end{proof}

\paragraph{}Subgradients can be classified via conjugate functions. By \Cref{defn:025-conjugate}, we have Fenchel's inequality:
\[
	f(x) + f^\star(y) = f(x) + \sup_{x}(x^Ty-f(x)) \geq x^Ty
\]

\begin{prop}[Subgradients v.s. Conjugate Functions]\label{prop:046-subgrad-conjugate}
	Consider the conditions:
	\begin{center}
		(a) $f(x) + f^\star(y) = x^Ty$ \quad (b) $y\in \partial f(x)$ \quad (c) $x\in \partial f^\star(x)$
	\end{center}
	then (a) and (b) are equivalent, and they are equivalent to (c) if $f$ is closed.
\end{prop}

\begin{proof}
	"(a) iff (b)": use $\partial f(x) = \left\{y:f^\star(y)=x^Ty-f(x)\right\}$. For (c), use (d) of \Cref{prop:025-conjugacy-theorem}.
\end{proof}

\paragraph{}Subgradients can also be defined as optimization problems: as $d\in\partial f(x)$ iff $x$ minimizes $f(y)-d^Ty$ (over $y$). We will use this to give some yogas of subgradients, and derive a constrained optimality condition.

\begin{prop}[Yoga of Subgradients]\label{prop:046-yoga-subgradients}.
	\begin{enumerate}[label=(\alph*)]
		\item (Chain Rule) Given proper convex $f$ on $\mathbb{R}^m$, $A\in \mathbf{M}_{m,l}(\mathbb{R})$, write $X=\operatorname{dom}(f)$. If $f\circ A$ is proper, then $A^T\partial f(Ax)\subset \partial (f\circ A)(x)$, and "$=$" holds if either $A^{-1}\operatorname{ri}(X)\neq\emptyset$ or $f$ is polyhedral.
		\item (Summation Rule) Given convex proper functions $f_1,\dotsc, f_r,f_{r+1},\dotsc,f_{s}$ each having domain $X_1,\dotsc,X_s$, with $f_1,\dotsc,f_r$ polyhedral, on $\mathbb{R}^m$ such that $g:=\sum_{i=1}^sf_i$ is proper, then $\sum_{i=1}^s\partial f_i(x)\subset \partial g(x)$, and "$=$" holds if $\left(\bigcap_{i=1}^r X_i\right)\cap \left(\bigcap_{i=r+1}^s \operatorname{ri}(X_i)\right)\neq\emptyset$.
	\end{enumerate}
\end{prop}

\begin{proof}
	For (a), to show "$\subset$", use \Cref{prop:046-subgrad-conjugate}: for $y\in \partial f(Ax)$, $F(x)+F^\star(A^Ty)\leq f(Ax)+f^\star(y)=x^TA^Ty$. For "$=$", we use duality: given $d\in \mathbb{R}^l$, consider three problems (with (3) having also its dual):
	\[
		\begin{matrix}
			(1)                             &
			\begin{Bmatrix}
				\text{Min.}\;    f(z) - w^Tz \\
				\text{s.t.}\;  z\in X
			\end{Bmatrix} & (2)                                                & \begin{Bmatrix}
				                                                                     \text{Min.}\;    f(Ay)-d^Ty \\
				                                                                     \text{s.t.}\;  y\in A^{-1}X
			                                                                     \end{Bmatrix} \\
			(3)                             & \begin{Bmatrix}
				                                  \text{Min.}\;   f(z)-d^Ty \\
				                                  \text{s.t.}\;  (y,z)\in \mathbb{R}^l\times X, Ay=z
			                                  \end{Bmatrix} & \leftrightarrow                &
			\begin{Bmatrix}
				\text{Max.}\;    \inf_{(y,z)\in \mathbb{R}^l\times X}f(z)-w^Tz-(d-A^Tw)^Ty \\
				\text{s.t.}\;    w\in \mathbb{R}^m
			\end{Bmatrix}
		\end{matrix}
	\]
	where problem $(1)$ has a vector $w$ to be determined. Write the primal optimal sets, values for problem $(i)$ as $F^{(i),\ast},f^{(i),\ast}$, the dual optimal set, dual optimal value of problem $(3)$ as $q^\ast,Q^\ast$. We have $f^{(2),\ast}=f^{(3),\ast}=q^{\ast},Q^\ast\neq\emptyset$ by \Cref{coro:044-slater-combined}. For any $w\in Q^\ast$, we must have $d=A^Tw$. By taking such $w$ in problem $(1)$, we have $f^{(1),\ast}\leq f^{(2),\ast}$. Note that in this case, $q^\ast=f^{(1),\ast}$ - as $A^Tw=d$, we have:
	\[
		q^\ast = \inf_{(y,z)\in \mathbb{R}^l\times X}f(z)-w^Tz-(d-A^Tw)^Ty =  \inf_{z\in X}f(z)-w^Tz=f^{(1),\ast}
	\]
	giving $f^{(1),\ast}=f^{(2),\ast}$ and $AF^{(2),\ast}\subset F^{(1),\ast}$. Therefore, given $d\in \partial (f\circ A)(x)$, define problem $(1)$ according to any element $w\in Q^\ast$, then $Ax \in AF^{(2),\ast}\subset F^{(1),\ast}$, giving $w\in \partial f(Ax)$, showing $d=A^Tw\in A^T\partial f(Ax)$. For (b), use (a)'s proof with $A:\mathbb{R}^m\to \mathbb{R}^{ms},x\mapsto (x,x,\dotsc,x)$, $f:\mathbb{R}^{ms}\to \overline{\mathbb{R}}$, $(x_i)_{i=1}^s\mapsto \sum_{i=1}^sf_i(x_i)$.
\end{proof}

\begin{prop}[Subgradients and Optimality]\label{prop:046-subgrad-optimality-conditions}
	Let $f$ be proper convex on $\mathbb{R}^m$ with $\operatorname{dom}(f)=X$, and let $C\subset \mathbb{R}^m$ be a convex subset. If any of the following holds:
	\begin{enumerate}[label=(\alph*)]
		\item $\operatorname{ri}(X)\cap \operatorname{ri}(C)\neq\emptyset$
		\item $f$ is polyhedral and $\operatorname{ri}(C)\cap X\neq\emptyset$.
		\item $C$ is polyhedral and $\operatorname{ri}(X)\cap C\neq\emptyset$.
		\item both $f$ and $C$ are polyhedral with $X\cap C\neq\emptyset$
	\end{enumerate}
	then $x\in C$ minimizes $f$ over $C$ iff $O\in\partial f(x)+N_C(x)$, where $N_C(x):=\partial \delta_C(x)=(C-x)^\ast$. That is: the set of subgradients $\partial f(x)$, along with the "normal cone" $N_C(x)$ is enough to determine optimality over $C$.
\end{prop}
\begin{proof}
	This is the same as minimizing $\delta_C+f$, so optimality is the same as $O\in\partial(\delta_C+f)(x)=N_C(x)+\partial f(x)$ (by \Cref{prop:046-yoga-subgradients} and conditions (a)-(d)). That $N_C(x)=(C-x)^\ast$ (polar cone of $C-x$) is easy to show.
\end{proof}
