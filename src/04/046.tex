\section{Theory of Subgradiants (WIP \faWrench)}
\label{sect:046}

\paragraph{}We will define subgradients via MC/MC. For this section, fix a proper convex function $f$ on $\mathbb{R}^m$, write $X=\operatorname{dom}(f)$. In the case where $f$ is differentiable, a property that holds for $\nabla f(x)$ at some $x\in X$ is that the following holds:
\[
	f(z)-f(x)\geq \nabla f(x)^T(z-x),\; \text{ for all }z\in \mathbb{R}^m
\]
which motivates the following definition (written in MC/MC formalism):

\begin{defn}[Subgradients]\label{defn-subgradients}
	For $x\in X$, define $\operatorname{epi}(f)(x) = \operatorname{epi}(f) - (x,f(x))$. Define:
	\[
		\partial f(x) = \left\{
		y\in \mathbb{R}^m,
		\operatorname{epi}(f)(x)\subset E_{(y,-1),0,-}
		\right\}
	\]
	as the "subgradient of $f$ at $x$". Note: since $w^\ast_{\operatorname{epi}(f)(x)}=0$, $\partial f(x)=Q^\ast_{\operatorname{epi}(f)(x)}$ (weak duality). Therefore:
	\[
		\partial f(x) = \left\{y:\inf\left\{(-y,1)^T \operatorname{epi}(f)(x)\right\}=0\right\}=\left\{y:\sup_z \left\{ z^Ty-f(z)\right\}=x^Ty-f(x)\right\}
	\]
\end{defn}

\paragraph{}Now \Cref{prop:032-optimal-set} translates to:

\begin{prop}[Decomposition of Set of Subgradients]\label{prop:046-set-of-subgrad}
	Write $V=\operatorname{lin}(X)$. For $x\in \operatorname{ri}(X)$, $\partial f(x) = V^\perp + V\cap \partial f(x)$ with $V\cap \partial f(x)\neq\emptyset$ compact and convex. In particular, if $x\in \operatorname{int}(X)$, $\partial f(x)$ is compact.
\end{prop}

\paragraph{}Subgradients can be classified via conjugate functions. By \Cref{defn:025-conjugate}, we have Fenchel's inequality:
\[
	f(x) + f^\star(y) = f(x) + \sup_{x}(x^Ty-f(x)) \geq x^Ty
\]

\begin{prop}[Subgradients v.s. Conjugate Functions]\label{prop:046-subgrad-conjugate}
	Consider the conditions:
	\begin{enumerate}[label=(\alph*)]
		\item $f(x) + f^\star(y) = x^Ty$
		\item $y\in \partial f(x)$
		\item $x\in \partial f^\star(x)$
	\end{enumerate}
	then (a) and (b) are equivalent, and they are equivalent to (c) if $f$ is closed.
\end{prop}

\begin{proof}
	"(a) iff (b)": use $\partial f(x) = \left\{y:f^\star(y)=x^Ty-f(x)\right\}$. For (c), use (d) of \Cref{prop:025-conjugacy-theorem}.
\end{proof}

\paragraph{}We can now formulate two yogas for subgradients:

\begin{prop}[Yoga of Subgradients]\label{prop:046-yoga-subgradients}.
	\begin{enumerate}[label=(\alph*)]
		\item (Chain Rule) We have $A^T\partial f(Ax)\subset \partial F(x)$.
		\item
	\end{enumerate}
\end{prop}

\begin{proof}
	For (a), use \Cref{prop:046-subgrad-conjugate}: for $y\in \partial f(Ax)$, $F(x)+F^\star(A^Ty)\leq f(Ax)+f^\star(y)=x^TA^Ty$.
\end{proof}

\paragraph{}The subgradients of the conjugate function is also interesting.

