\section{General Convex Programming Duality}
\label{sect:042}

\paragraph{}As in our discussion of MC/MC in \Cref{sect:031}, we can ask the following analogous questions:
\begin{itemize}
	\item (Weak Duality) Do we have $f^\ast\geq q^\ast$?
	\item (Strong Duality) When do we have $f^\ast=q^\ast$?
	\item (Reachability) Do we have $F^\ast\neq\emptyset,Q^\ast\neq\emptyset$, given information on nonemptiness of $F,Q$? Are they compact? In general, Farkas lemma gives results on $Q^\ast$; the problem is more intricate for $F^\ast$.
\end{itemize}

\paragraph{}We can first give some partial answers, some of which utilizes both the proof and the statement of the nonlinear Farkas lemma:
\begin{itemize}
	\item (Weak duality) By the proof of \Cref{prop:041-farkas-lemma-nonlinear}, $f^\ast$, $q^\ast$ are min-common and max-crossing values, so MC/MC weak duality gives $f^\ast\geq q^\ast$.
	\item (Strong Duality) The duality gap $f^\ast-q^\ast$ has the following bound:
	      \[
		      \underset{(x,y)\in F\times Q}{\operatorname{inf}}\left(f(x)-q(y)\right)\leq
		      \underset{(x,y)\in F\times Q}{\operatorname{inf}}\left(f(x)-L(x,y)\right)=
		      \underset{(x,y)\in F\times Q}{\operatorname{inf}}y^Tg(x)
	      \]
	      Note that $g(F)\leq O,Q\leq O$. One can replace $F\times Q$ by $F^\ast\times Q^\ast$ if both $F^\ast,Q^\ast$ are nonempty. A particular case is given by the notion of complementary slackness; see \Cref{prop:042-complementary-slackness}.
	\item (Reachability) If one applies \Cref{prop:041-farkas-lemma-nonlinear} to \Cref{defn:040-opt-prob} with $f$ being $f-f^\ast$, assuming $f^\ast$ is finite, then the corresponding $Q_0$ is the same as $Q^\ast$. In other words, the "Slater's conditions" in fact can be used to give conditions for nonemptiness or compactness of optimal sets. In fact, the situation is even better: Slater's conditions gives strong duality, as for $y_0\in Q^\ast$ in this case, we have:
	      \[
		      0\leq \underset{x\in X}{\operatorname{inf}} \left(f(x)-f^\ast-y_0^T g(x)\right) \leq q^\ast-f^\ast
	      \]
\end{itemize}

\paragraph{}We collect these as propositions.

\begin{prop}[Weak Duality]\label{prop:042-weak-duality}
	We have $f^\ast\geq q^\ast$. Consequently, if the primal (resp. dual) problem has no optimal values, then the dual (resp. primal) problem is infeasible.
\end{prop}

\begin{prop}[Complementary Slackness]\label{prop:042-complementary-slackness}
	Given $x_0\in F$ and $y_0\leq O\in \mathbb{R}^r$, then TFAE:
	\begin{enumerate}[label=(\alph*)]
		\item (Strong Duality and Reachability) $(x_0,y_0)\in F^\ast\times Q^\ast$ and $f^\ast=q^\ast$.
		\item (Complementary Slackness) $x_0\in \underset{x\in X}{\operatorname{argmin}}L(x,y_0)$, and $y_0^Tg(x_0)=0$.
	\end{enumerate}
\end{prop}

\begin{proof}
	"(b) implies (a)" is by the bound on $f^\ast - q^\ast$ given above. The converse uses weak duality.
\end{proof}

\begin{prop}[Slater's Conditions]\label{prop:042-slater-optimality}
	Assume $f^\ast<\infty$, then
	\begin{enumerate}[label=(\alph*)]
		\item (Slater's Condition I) If there is some $x_0\in X$ with $g(x_0)<O$, then $Q^\ast$ is nonempty and compact.
		\item (Slater's Condition II) If $g$ is affine with some $x_0\in \operatorname{ri}(X)$ with $g(x_0)\leq O$, then $Q^\ast$ is nonempty.
	\end{enumerate}
	If either of (a) or (b) holds, then $f^\ast=q^\ast$.
\end{prop}
