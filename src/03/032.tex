\section{MC/MC - General Duality Theorems}
\label{sect:032}

\paragraph{}Under some assumptions, we have sufficient and necessary conditions for $w^\ast=q^\ast$. Fix some $M\subset \mathbb{R}^{m+1}$.

\begin{prop}[Strong Duality]\label{prop:032-strong-duality}
	Suppose the following holds:
	\begin{enumerate}[label=(\alph*)]
		\item $w^\ast<\infty$, or $M$ has no vertical lines.
		\item $M^+$ is convex.
	\end{enumerate}
	then the following are equivalent:
	\begin{enumerate}[label=(\arabic*)]
		\item For every sequence $\{(x_k,u_k)\}_{k}\subset M$ with $x_k\to O$, we have $\underset{k}{\operatorname{liminf}}u_k\geq w^\ast$.
		\item $q^\ast=w^\ast$
	\end{enumerate}
\end{prop}

\paragraph{}The main idea is to consider non-vertical hyperplane separation between $\operatorname{cl}(M^+)$ and points on $\{O\}\times \mathbb{R}$.

\begin{proof}
	Assuming (2), and $\{(x_k,u_k)\}_{k}\subset M$ with $x_k\to O$. Want to show (1) holds. It suffices to show $q(y)\leq \underset{k}{\operatorname{liminf}}u_k$. This is by applying $\underset{k}{\operatorname{liminf}}$ to:
	\[
		q(y)=\inf (-y,1)^TM \leq -x_k^Ty+u_k
	\]
	Assuming (1), want to show (2). If $w^\ast=-\infty$, we are done by \Cref{prop:031-weak-duality}. Otherwise, it suffices to show for $v<w^\ast$, $(O,v)$ and $\operatorname{cl}(M^+)$ can be non-verticallly strictly separated (which would imply $q^\ast >v$).
	\begin{enumerate}[label=(\roman*)]
		\item $(O,v)\notin \operatorname{cl}(M^+)$: otherwise, take sequences $M^+\supset\{(x_k,v_k)\}_{k}\to (O,v)$, $M\supset \{(x_k,u_k)\}_{k}$ with $u_k\leq v_k$, then $\underset{k}{\operatorname{liminf}}u_k\leq v<w^\ast$, contradicting assumption (1).
		\item $M^+$ has no vertical lines: there are two cases -
		      \begin{itemize}
			      \item $w^\ast=\infty$: then this is true by assumption (a).
			      \item $w^\ast<\infty$: then $M_{O,:}\neq\emptyset$, so if $M^+$ has vertical lines, then the closed convex set $\operatorname{cl}(M^+)$ would contain the axis $\{O\}\times \mathbb{R}$ (see (d) of \Cref{prop:013-yoga-recession}), contradicting (i).
		      \end{itemize}
	\end{enumerate}
	As (i), (ii) are true, the conditions for (g) of \Cref{prop:015-hyperplane-sep} are satisfied, and we are done.
\end{proof}

\begin{coro}[Reachability - Special Cases]\label{coro:032-reachability-special-cases}Suppose $w^\ast<\infty$.
	\begin{enumerate}[label=(\alph*)]
		\item Suppose $M$ is closed and convex, then $M^{+,\vee}$ is convex, and $q^\ast=w^\ast$. If $w^\ast > -\infty$, $M^{+,\vee}$ is also closed, proper.
		\item In general, we have $q_M^\ast = w_{\operatorname{clconv}(M)}^\ast = q_{\operatorname{clconv}(M)}^\ast$.
		\item Suppose $M=M'+C\times \{0\}$ for some compact $M'\subset \mathbb{R}^{m+1}$, convex closed $C\subset \mathbb{R}^m$, then $q^\ast_{M}=q^\ast_{\operatorname{conv}(M)}=w^\ast_{\operatorname{conv}(M)}$.
	\end{enumerate}
\end{coro}
\begin{proof}
	For (c), by (a), we show $\operatorname{conv}(M)$ is closed and convex. Note that $\operatorname{conv}(M)=\operatorname{conv}(M')+C\times \{O\}$. By \Cref{prop:011-cara}, this is a sum of a two closed convex sets with one being compact, so by \Cref{coro:014-closed-sum}, this set is convex and closed. For (a), we only need $M^{+,\vee}$ closed proper when $w^\ast\in \mathbb{R}$, which follows from the definition $M^+=M+\{O\}\times \mathbb{R}_{+}$, and the fact that $(O, -1)\notin R_M$ (as $w^\ast \in \mathbb{R}$). For (b), use (a).
\end{proof}

\begin{prop}[Reachability of Optimal Values]\label{prop:032-reachability}
	Let $\pi :\mathbb{R}^{m+1}\to \mathbb{R}^m$ be $(x,u)\mapsto x$. Suppose:
	\begin{enumerate}[label=(\alph*)]
		\item $w^\ast >-\infty$
		\item $M^+$ is convex
		\item $O\in \operatorname{ri}(\pi M)$
	\end{enumerate}
	then $q^\ast=w^\ast$, with $q^\ast=q(y)$ for some $y\in \mathbb{R}^m$ (that is, $Q^\ast\neq\emptyset$).
\end{prop}
\begin{proof}
	Condition (c) and (a) gives $w^\ast \in \mathbb{R}$. We now try to use (e) of \Cref{prop:015-hyperplane-sep} (see also the remark following the proposition) on $(O, w^\ast)$ and $M^+$. Indeed, we have $(O, w^\ast)\notin \operatorname{ri}(M^+)$, as $\operatorname{ri}(M^+_{O,:})=(w^\ast,0)$, by (g) of \Cref{prop:012-yoga-ri-cl}. Therefore, we have some $b\in \mathbb{R}$, $y\in \mathbb{R}^m$, $\varepsilon \in \{0,\pm 1\}$ with $(y,\varepsilon )\neq (O,0)$ that:
	\[
		(O,w^\ast)\in H_{(y,\varepsilon),b}\nsupset M^+ \subset E_{(y,\varepsilon ),b,-}
	\]
	As $M^+$ is upward-closed, $\varepsilon\leq 0$. If $\varepsilon =0$, then $H_{y,b}$ properly separates $O,\pi M^+=\pi M$, a contradiction to condition (c) paired with (e) of \Cref{prop:015-hyperplane-sep}. Therefore, $\varepsilon =-1$, giving the assertion by:
	\[
		w^\ast \leq q(y) \leq q^\ast \leq w^\ast \qedhere
	\]
\end{proof}

\paragraph{}Next we look at the set $Q^\ast$. At the very least, $Q^\ast$ is closed and convex, by:
\[
	Q^\ast = q^{-1}q^\ast = V_{-q,-q^\ast}
\]

\begin{prop}[Geometry of $Q^\ast$]\label{prop:032-optimal-set}
	Suppose (a)-(c) of \Cref{prop:032-reachability} holds. Let $V=\operatorname{lin}(\pi M)$, then:
	\begin{enumerate}[label=(\alph*)]
		\item $L_{Q^\ast}=R_{Q^\ast}=V^\perp$.
		\item $Q^\ast = V^\perp + V\cap Q^\ast$, with $V\cap Q^\ast$ nonempty, convex, compact.
		\item In particular, if $O\in \operatorname{int}(\pi M)$, $Q^\ast$ is compact.
	\end{enumerate}
\end{prop}

\begin{proof}
	(c) follows from (b). The decomposition in (b) is by (a) and \Cref{prop:013-decomposition}, and the compactness statement follows from (d) of \Cref{prop:013-yoga-recession} and (a). To show (a), only need $L_{Q^\ast}\supset V^\perp \supset R_{Q^\ast}$.
	\begin{itemize}
		\item $L_{Q^\ast}\supset V^\perp$: this is by
		      \[
			      q^\ast \geq \inf q(Q^\ast + V^\perp )=\inf \left((-Q^\ast - V^\perp)\times \{1\}\right)^T M = \inf \left((-Q^\ast) \times \{1\}\right)^TM = q^\ast
		      \]
		\item $V^\perp\supset R_{Q^\ast}$: Since $O\in \operatorname{ri}(\pi M)$, $B({O,\varepsilon})\cap V = B({O,\varepsilon})\cap (\pi M)$ for some $\varepsilon >0$. We show:
		      \[
			      R_{Q^\ast}^T \left(B({O,\varepsilon})\cap V\right) = \{0\}
		      \]
		      Fix $x\in V$ with $\|x\|<\varepsilon $; note that $x\in \pi M$. Take $u\in \mathbb{R}$ with $\{\pm x\}\times \{u\}\subset M^+$. We get:
		      \[
			      q^\ast = q(Q^\ast + R_{Q^\ast})\leq \inf \left(\left(-Q^\ast-R_{Q^\ast}\right)\times \{1\}\right)^T\left(\{\pm x\}\times \{u\}\right)
			      =u+\inf \left(-Q^\ast-R_{Q^\ast}\right)^T\{\pm x\}
		      \]
		      As $R_{Q^\ast}^T\{\pm x\}$ is a subspace of $\mathbb{R}$, if $R_{Q^\ast}^Tx\neq \{O\}$, RHS is $-\infty$; this concludes the proof.\qedhere
	\end{itemize}
\end{proof}

\paragraph{}A variant of \Cref{prop:032-reachability} is given in the following corollary. The proof is similar, with separation between $(O,w^\ast)$, $M^+$ replaced by $P\times \{w^\ast\}$, $M'^{+}$.

\begin{coro}[Variants of Reachability Theorems]\label{coro:032-variants-reachability}
	Suppose:
	\begin{enumerate}[label=(\alph*)]
		\item $w^\ast>-\infty$
		\item $M^+=M'-P\times \{0\}$, with $M'\subset \mathbb{R}^{m+1},P\subset \mathbb{R}^m$ both being convex.
		\item Either $\operatorname{ri}(\pi M')\cap \operatorname{ri}(P)\neq\emptyset$ or $\operatorname{ri}(\pi M')\cap P\neq\emptyset$ with $P$ being polyhedral
	\end{enumerate}
	then $q^\ast=w^\ast$, $\emptyset\neq Q^\ast\subset R_P^\ast$ (polar cone of $P$). Furthermore, $Q^\ast$ is compact if $\operatorname{int}(\pi M')\cap P\neq\emptyset$.
\end{coro}

\begin{proof}
	Strong duality $q^\ast=w^\ast$ is ensured by condition (c) (giving $w^\ast\neq \infty$) and \Cref{prop:032-strong-duality}. We show $Q^\ast\neq\emptyset$: take $P\times \{w^\ast\},M'^{,+}$. Suppose $\operatorname{ri}(\pi M')\cap P\neq\emptyset$ with $P$ being polyhedral (resp. $\operatorname{ri}(\pi M')\cap \operatorname{ri}(P)\neq\emptyset$), then by \Cref{prop:017-polyhedral-proper-sep} (resp. (c) of \Cref{prop:015-hyperplane-sep}), we have separation:
	\[
		P\times \{w^\ast\}\subset E_{(y,\varepsilon ),b,+},\; H_{(y,\varepsilon ),b}\nsupset M'^{,+}\subset E_{(y,\varepsilon ),b,-}
	\]
	\[
		\text{(resp. }P\times \{w^\ast\}\subset E_{(y,\varepsilon ),b,+},\; M'^{,+}\subset E_{(y,\varepsilon ),b,-},\; H_{(y,\varepsilon ),b}\nsupset P\times \{w^\ast\}\cup M'^{,+}\text{ )}
	\]
	with $b\in \mathbb{R}$, $y\in \mathbb{R}^m$, $\varepsilon \in \{0,\pm 1\}$ and $(y,\varepsilon )\neq (O,0)$. As $M'^{,+}$ is upward-closed and condition (c), $\varepsilon =-1$. This same hyperplane gives $w^\ast\geq q^\ast\geq q(y)\geq w^\ast$, giving $Q^\ast\neq\emptyset$. For $Q^\ast\subset R_P^\ast$, take $y\in Q^\ast$, we have:
	\[
		q(y)=\inf (-y, 1)^TM^+ = \inf (-y, 1)^T\left(M'-P\times \{0\}\right) = \inf (-y, 1)^T\left(M'-(P+R_P)\times \{0\}\right)
	\]
	so if $\sup y^TR_P>0$, $q(y)=-\infty$ as $y^TR_P$ is a cone. For compactness of $Q^\ast$, use (c) of \Cref{prop:032-optimal-set}.
\end{proof}

