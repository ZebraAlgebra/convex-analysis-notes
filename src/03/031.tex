\section{MC/MC - Definitions and Basic Facts}
\label{sect:031}

\paragraph{}For this subsection, let us fix $m\in \mathbb{N}$, and consider a non-empty subset $M\subset \mathbb{R}^{m+1}$. Elements in $\mathbb{R}^{m+1}$ are often expressed as a tuple $(x,s)$ with $x\in \mathbb{R}^m$ and $s\in \mathbb{R}$.

\begin{defn}[MC/MC]\label{defn:031-mc-mc}
	The min-common and max-crossing values of $M$, denoted as $w^\ast,q^\ast$, are:
	\[
		w^\ast = \inf M_{O,:},\; q^\ast = \sup_{y\in \mathbb{R}^m} q(y)
	\]
	where $q(y)$ - is called the "dual function" - defined as the function (compare to \Cref{defn:025-conjugate}):
	\[
		q(y):= -\inf\left\{ b\in \mathbb{R}: M\subset E_{(y,-1),b,-} \right\} = - \sup_{(x,u)\in M}\left\{y^Tx-u\right\} = \inf (-y,1)^TM
	\]
	When we want to emphasize the role of $M$, we will decorate $q,q^\ast,w^\ast$ as $q_M,q^\ast_M,w^\ast_M$. Note that by the description $q(y) =\inf (-y,1)^TM$, $-q$ is convex and lower-semicontinuous (see (c) of \Cref{prop:021-yoga-convex-functions}). The set of points where $q(y)=q^\ast$ is written as $Q^\ast$.
\end{defn}

\begin{rmrk}
	Our definition of $q$ is a bit different from \cite{bertsekas2009convex}; namely, our $q(y)$ is the same as $q(-y)$ in the book.
\end{rmrk}

\paragraph{}Geometrically, $q(y)$ describes intersections of halfspaces with fixed direction $(y,-1)$ containing $M$ with the axis $\{O\}\times \mathbb{R}$. This is similar to \Cref{defn:025-conjugate}, except a difference in sign: when $M=\operatorname{epi}(f)$ for some scalar function $f$ on $\mathbb{R}^m$, we get $q_{\operatorname{epi}(f)}=-f^\star$.

\paragraph{}The remainder of this section is dedicated to the study of the relations between $q^\ast,w^\ast$. In particular:

\begin{itemize}
	\item (Weak Duality) We will see shortly (\Cref{prop:031-weak-duality}) that in general, we have $q^\ast\leq w^\ast$.
	\item (Strong Duality) When do we have $q^\ast = w^\ast$?
	\item (Dual Reachability) Is the set of dual optimal solutions $Q^\ast$ nonempty, or even compact?
\end{itemize}

\begin{prop}[Weak Duality]\label{prop:031-weak-duality}
	We have $q^\ast\leq w^\ast$.
\end{prop}

\begin{proof}
	This is by $w_M^\ast = q^\ast_{\{O\}\times M_{O,:}} \geq q^\ast_M$.
\end{proof}

\paragraph{}Since the halfspaces $E_{(y,-1),b,-}$ are upward-closed, it would make sense to try to make $M$ upward-closed.

\begin{defn}[Upwardification]\label{defn:031-upwardification}
	The upwardification of $M$ is defined as:
	\[
		M^+= M+\{O\}\times \mathbb{R}_{+}
	\]
\end{defn}

\paragraph{}Note that if for each $x\in \mathbb{R}^m$ that either $M_{x,:}=\emptyset$ or $\inf M_{x,:}\in M_{x,:}$, then $M^+$ is epigraph-like.

\paragraph{}The following are some direct consequences of \Cref{defn:031-mc-mc}, \Cref{prop:031-weak-duality}; proofs are omitted.

\begin{lemm}[Basic Consequences of Definitions and Weak Duality].
	\label{lemm:031-direct-consequences}
	\begin{enumerate}[label=(\alph*)]
		\item (When $w^\ast=-\infty$ or $q^\ast=\infty$) If $w^\ast=-\infty$ or $q^\ast=\infty$, we have $q^\ast=w^\ast$.
		\item (When $w^\ast=+\infty$) We have $w^\ast=\infty$ iff $M_{O,:}=\emptyset$.
		\item (Upwardification) We have $w_M^\ast=w_{M^+}^\ast$, $q_M=q_{M^+}$.
		\item (Convex Closure) We have $q_{\operatorname{cl}(\operatorname{conv}(M))}=q_{\operatorname{conv}(M)}=q_{\operatorname{cl}(M)}=q_M$.
	\end{enumerate}
\end{lemm}


