\section{Polyhedral Sets}
\label{sect:017}

\paragraph{}We have seen in \Cref{coro:015-halfspaces-intersection} that closed convex sets are essentially intersections of closed halfspaces. A special class of closed convex sets - called polyhedral - are the ones that are given by finite intersection of closed convex halfspaces.

\begin{defn}[Polyhedral Sets and Cones]\label{defn:017-polyhedral-sets}
	A set $P\subset \mathbb{R}^m$ is polyhedral if it is of the following form:
	\[
		P=\bigcap_{i=1}^rE_{a_i,b_i,-},\text{ for some }r\in \mathbb{N},\{a_i\}_{i=1}^r\subset\mathbb{R}^m,\{b_i\}_{i=1}^r\subset\mathbb{R}
	\]
	Equivalently, it is the set:
	\[
		P=\{v\in \mathbb{R}^m:Av\leq b\}
	\]
	where $A\in \mathbf{M}_{r,m}(\mathbb{R}),b\in \mathbb{R}^r$ given by $A_{i,:}=a_i^T,b_i=b_i$. Under such representation, we call $Av\leq b$ a defining equation of $P$. A polyhedral set is a polyhedral cone if the $b_i$'s can be taken to be $0$.
\end{defn}

\paragraph{}Note that a polyhedral cone is precisely a polyhedral set that is also a cone: suppose a cone $X$ is contained in some $E_{a,b,-}$, then $b\geq 0$ and $X\subset E_{a,0,-}$.

\begin{exmp}
	A "linear programming (LP) problem in standard form" is given as the follows:
	\begin{align*}
		\text{Maximize}\quad   & c^Tx              \\
		\text{subject to}\quad & Ax\leq b, x\geq 0
	\end{align*}
	where $A\in \mathbf{M}_{r,m}(\mathbb{R}),b\in \mathbb{R}^r,c\in \mathbb{R}^m$. In such case, the region given by the constraints (called the "feasibility region") is a polyhedral set with a defining equation of the form:
	\[
		\left[
			\begin{array}{r}
				A \\\hline
				-I
			\end{array}
			\right]
		x\leq
		\left[
			\begin{array}{r}
				b \\\hline
				O
			\end{array}
			\right]
	\]
\end{exmp}

\subsection{Extreme Points of Polyhedral Sets}

\paragraph{}Extreme points of polyhedral sets admits quite explicit descriptions. We first define a notation:

\begin{defn}[A Submatrix Construction]\label{defn:017-submatrix}
	Given $A\in \mathbf{M}_{r,m}(\mathbb{R}),b\in \mathbb{R}^r,v\in \mathbb{R}^m$ with $Av\leq b$, we denote $A_{v; b}$ as the matrix given by keeping the rows of $A$ where $"="$ holds.
\end{defn}

\begin{prop}[Characterization of Extreme points of Polyhedral Sets]\label{prop:017-characterization-polyhedral-extreme}
	Given a defining equation $Ax\leq b$ of a polyhedral set $P$ for some $A\in \mathbf{M}_{r,m}(\mathbb{R}),b\in \mathbb{R}^m$, given $v\in P$, $v\in \operatorname{ext}(P)$ iff $A_{v; b}$ is non-empty, injective.
\end{prop}

\begin{proof}
	Let $I_v$ be a subset of row indices $\{i:A_{i,:}v=b_i\}$. If $A_{v; b}$ is injective, then:
	\[
		\operatorname{ext}(P)\cap\left(\bigcap_{i\in I_v}H_{a_i,b_i}\right)=
		\operatorname{ext}\left(P\cap\bigcap_{i\in I_v}H_{a_i,b_i}\right)=\operatorname{ext}\left(\{v\}\right)=\{v\}
	\]
	by \Cref{lemm:016-extreme-and-hyperlane}. Otherwise, notice:
	\[
		v\in\left(\bigcap_{i\notin I_v}E_{a_i,b_i,-}^o\right)\cap\left(\bigcap_{i\in I_v}H_{a_i,b_i}\right)\subset P
	\]
	If $I_v$ (and hence $A_{v;b}$) is empty, then $P$ contains an open neighborhood of $v$. If $A_{v;b}$ is not injective, take $w\in \operatorname{ker}(A_{v;b})\smallsetminus\{0\}$, then there is some $\varepsilon >0$ with $\{v\pm \varepsilon w\}\subset P$. In each case, $v$ cannot be extreme.
\end{proof}

\begin{coro}[Variants]\label{coro:017-char-variants}
	Let $P$ be a polyhedral set, $A\in \mathbf{M}_{r,m}(\mathbb{R}),b\in \mathbb{R}^r,c,d\in \mathbb{R}^m$,
	\begin{enumerate}[label=(\alph*)]
		\item Suppose $P=\{x\in \mathbb{R}^m:Ax=b,x\geq 0\}$. Given $v\in P$, we have $v\in \operatorname{ext}(P)$ iff the matrix $A'_v$ given by keeping columns in $A$ corresponding to indices of $v$ where $v> 0$ is empty or injective.
		\item Suppose $P=\{x\in \mathbb{R}^m:Ax=b,c\leq x\leq d\}$. Given $v\in P$, we have $v\in \operatorname{ext}(P)$ iff the matrix $A'_v$ given by keeping columns in $A$ corresponding to indices of $v$ where $c<v<d$ is empty or injective.
	\end{enumerate}
\end{coro}

\begin{proof}
	One can write the defining equations of (a) and (b) in block form as:
	\[
		\left[
			\begin{array}{r}
				A  \\\hline
				-A \\\hline
				-I
			\end{array}
			\right]
		x\leq
		\left[
			\begin{array}{r}
				b  \\\hline
				-b \\\hline
				0
			\end{array}
			\right],\;
		\left[
			\begin{array}{r}
				A  \\\hline
				-A \\\hline
				I  \\\hline
				-I
			\end{array}
			\right]
		x\leq
		\left[
			\begin{array}{r}
				b  \\\hline
				-b \\\hline
				d  \\\hline
				-c
			\end{array}
			\right]
	\]
	The rest is an application of \Cref{prop:017-characterization-polyhedral-extreme}.
\end{proof}

\begin{coro}[Nonemptiness of $\operatorname{ext}(P)$ - Polyhedral Case]\label{coro:017-extreme-existence-poly}
	Suppose $P\subset \mathbb{R}^m$ is nonempty, polyhedral with defining equation $Ax\leq b$, then $\operatorname{ext}(P)\neq\emptyset$ iff $A$ is injective and nonempty.
\end{coro}
\begin{proof}
	If $A$ is injective, nonempty, $P$ has no lines (see \Cref{lemm:016-extreme-existence}). Otherwise, use \Cref{prop:017-characterization-polyhedral-extreme}.
\end{proof}

\begin{exmp}[Birkhoff-von Neumann Theorem]
	Fix $m\in \mathbb{N}$. Identify $\mathbf{M}_{m,m}(\mathbb{R})$ with $\mathbb{R}^{m^2}$ via $g:e_{i,j}\mapsto e_{i+m(j-1)}$ (that is, flattening by stacking columns). The space of doubly stochastic matrices - those with entries in $[0,1]$ and with each column and row summing to $1$ - denoted as $P_{m}$, contains the set of permutation matrices - denoted as $X_m$. We show that $\operatorname{ext}(P_m)=X_m$:
	\begin{itemize}
		\item $X_m\subset \operatorname{ext}(P_m)$: we have $J(g(P_m))\subset J\left([0,1]^{m^2}\right)$, $g(X_m)\subset \{0,1\}^m=\operatorname{ext}\left([0,1]^{m^2}\right)$.
		\item $X_m\supset \operatorname{ext}(P_m)$: the idea is to use \Cref{prop:017-characterization-polyhedral-extreme}. Note that $g(P_m)$ can be described as:
		      \[
			      g(P_m)=\left\{
			      x\in \mathbb{R}^{m^2}:x\geq 0,Ax=1
			      \right\}
		      \]
		      where $A\in \mathbf{M}_{2m,m^2}(\mathbb{R})$ has rows encoding the requirement that each row (resp. column) sum to $1$; note that $A$ is not injective, so if $M\in \operatorname{ext}(P_m)$, $M$ has at most $(2m-1)$ zero entries, so $M_{i,j}=1$ for some $i,j$ (pigeon-hole principle). Now we show $M\in X_m$ by induction using this $i,j$: if $m=1$, $i=j=1$, and we are done; otherwise, take $M_{(i,j)}$, defined as $M$ with $i$-th row and $j$-th column removed, lies in $\operatorname{ext}(P_{m-1})$ (by \Cref{lemm:016-extreme-and-hyperlane}), so by inductive hypothesis, $M_{(i,j)}\in X_{m-1}$, and hence $M\in X_m$.
	\end{itemize}
\end{exmp}

\subsection{Polyhedral Cones and Finitely-Generated Cones}

\paragraph{}Polyhedral cones has some unexpected structures.

\begin{defn}[Finitely Generated Cones]\label{defn:017-fg-cones}
	A convex cone $C$ is called finitely generated if it can be written as $\operatorname{c.cone}(X)$ for some finite subset $X=\{a_i\}_{i=1}^k\subset \mathbb{R}^m$.
\end{defn}

\paragraph{}Finitely generated cones are closed - they are images of $\mathbb{R}_+^k\subset \mathbb{R}^k$ under some linear map, so by \Cref{coro:014-closed-affine} applied to $C_0=\mathbb{R}_+^k$ (this set is retractive - being a finite intersection of halfspaces which are also retractive) and $C=\mathbb{R}^k$, it is closed.

\begin{prop}[Farkas' Cone Lemma]\label{prop:017-farkas}
	Given a finite set $X\subset \mathbb{R}$, the closed convex cones:
	\[
		\bigcap_{a\in X}E_{a,0,-},\;\operatorname{c.cone}(X)
	\]
	are polar to each other.
\end{prop}

\begin{proof}
	By definition we have $X^\ast=\bigcap_{a\in X}E_{a,0,-}$, so we are done by \Cref{lemm:016-polar-cone-theorem}.
\end{proof}

\begin{exmp}[An application of Farkas Lemma]
	Let $X,Y$ be finite subsets of $\mathbb{R}^m$, then:
	\[
		\left(\left(\bigcap_{a\in X}E_{a,0,-}\right)\cap \left(\bigcap_{a'\in Y}H_{a',0}\right)\right)^\ast
		=\operatorname{lin}(Y)+\operatorname{c.cone}(X)
	\]
\end{exmp}

\begin{prop}[Minkowski-Weyl Theorem]\label{prop:017-Minkowski-Weyl-Theorem}
	A convex cone $C\subset \mathbb{R}^m$ is polyhedral iff finitely generated.
\end{prop}
\begin{proof}
	By \Cref{prop:017-farkas} and \Cref{lemm:016-polar-cone-theorem}, it suffices to show that a finitely generated cone is polyhedral. Suppose $X=\{a_i\}_{i=1}^{k}\subset C$ with $C=\operatorname{c.cone}(X)$. If we write $V=\operatorname{lin}(C)$, then since $C=(C+V^\perp)\cap V$,	we may assume $V=\mathbb{R}^m\neq C$ by replacing $C$ by $C+V^\perp$. Given $b\notin C$, by (b) of \Cref{prop:015-hyperplane-sep}, there are $x\in \mathbb{S}^{m-1},y\in \mathbb{R}$ with $b\in E_{x,2y,+}^o,\; C\subset E_{x,2y,-}^o$.	Since $C$ is a non-trivial cone, we have $y>0$, giving $b\in E_{x,y,+},\; C\subset E_{x,0,-}$. In this sense, $x$ belongs to the following polyhedral set:
	\[
		P = E_{-b,-y,-}\cap\left(\bigcap_{i=1}^k E_{a_i,0,-}\right)
	\]
	In terms of defining equations (where $A_{i,:}=a_i^T$; note that $A$ is injective, as $V=\mathbb{R}^m$):
	\[
		P=
		\left\{x\in \mathbb{R}^m:\overline{A}x\leq\overline{O}
		,\text{ where }
		\overline{A}=\left[
			\begin{array}{r}
				A \\\hline
				-b^T
			\end{array}
			\right],\;
		\overline{O}=\left[
			\begin{array}{r}
				O \\\hline
				-y
			\end{array}
			\right]
		\right\}
	\]
	Existence and characterziations of extreme points of this set are given by \Cref{prop:017-characterization-polyhedral-extreme}, \Cref{coro:017-extreme-existence-poly}. Take $v\in \operatorname{ext}(P)$, and consider $\overline{A}_{v;\overline{O}}$, which should be injective. As $O\notin P$, we have:
	\[
		\overline{A}_{v;\overline{O}}=
		\left[
			\begin{array}{r}
				A_{v;O} \\\hline
				-b^T
			\end{array}
			\right]
	\]
	and $\operatorname{rank}(A_{v;O})=m-1$, giving separation between $b$ and $C$:
	\[
		b\in E^o_{v,0,+},\; \operatorname{im}(A^T_{v;O})=H_{v,0},\; C\subset E_{v,0,-}
	\]
	Note that $\operatorname{im}(A^T_{v;O})$ is $\operatorname{lin}(Y)$ for some $Y\subset X$. This gives:
	\[
		C = \bigcap_{\substack{
				v\in \mathbb{R}^{m-1}\smallsetminus\{0\},Y\subset X:\\
				\dim \operatorname{lin}(Y)=m-1,\;C\subset E_{v,0,-},\\
				H_{v,0}=\operatorname{lin}(Y)
			}
		}E_{v,0,-}=
		\bigcap_{\substack{
				v\in \mathbb{S}^{m-1},Y\subset X:\\
				\dim \operatorname{lin}(Y)=m-1,\;C\subset E_{v,0,-},\\
				H_{v,0}=\operatorname{lin}(Y)
			}
		}E_{v,0,-}
	\]
	which is a finite intersection of halfspaces (as for given $Y\subset X$ with $\operatorname{dim}\operatorname{lin}(Y)=m-1$, the conditions $C\subset E_{v,0,-},H_{v,0}=\operatorname{lin}(Y)$ determines $v\in \mathbb{S}^{m-1}$ uniquely).
\end{proof}

\begin{prop}[Minkowski-Weyl Representation]\label{prop:017-Minkowski-Weyl-Representation}
	A set $P\subset \mathbb{R}^m$ is polyhedral iff it is of the form $\operatorname{c.cone}(X)+\operatorname{conv}(Y)$ for some finite subsets $X,Y$ of $\mathbb{R}^m$.
\end{prop}
\begin{proof}
	If $P=\operatorname{c.cone}(X)+\operatorname{conv}(Y)$, then $P=\overline{P}_{:,1}$ where $\overline{P}\subset \mathbb{R}^{m+1}$ is defined as:
	\[
		\overline{P}=\operatorname{c.cone}\left(\{(x,0):x\in X\}\cup\{(y,1):y\in Y\}\right)
	\]
	therefore both $\overline{P}$ and $P$ are polyhedral by \Cref{prop:017-Minkowski-Weyl-Theorem}. On the other hand, if $P$ is polyhedral, with:
	\[
		P=\left(\bigcap_{x\in X} E_{x,0,-}\right)\cap\left(\bigcap_{y\in Y} E_{y,1,-}\right)\cap \left(\bigcap_{z\in Z} E_{z,-1,-}\right)
	\]
	where $X,Y,Z$ are finite subsets of $\mathbb{R}^m\smallsetminus\{0\}$, then $P=\overline{P}_{:,1}$ where $\overline{P}\subset \mathbb{R}^{m+1}$ is defined as:
	\[
		\overline{P}=E_{-e_{m+1},0,-}\cap\left(\bigcap_{x\in X} E_{(x,0),0,-}\right)\cap\left(\bigcap_{y\in Y} E_{(y,-1),0,-}\right)\cap \left(\bigcap_{z\in Z} E_{(z,1),0,-}\right)
	\]
	which is a polyhedral cone. By \Cref{prop:017-Minkowski-Weyl-Theorem}, we may write:
	\[
		\overline{P}=\operatorname{c.cone}\left(\{(x,0):x\in X'\}\cup \{(y, 1):y\in Y'\}\right)
	\]
	for some finite subsets $X',Y'$ in $\mathbb{R}^m$. Under this representation, $P=\overline{P}_{:,1}=\operatorname{c.cone}(X')+\operatorname{conv}(Y')$.
\end{proof}

\subsection{Yogas of Polyhedral Sets}

\paragraph{}We are now ready to formulate some yogas of polyhedral sets:

\begin{prop}[Yoga of Polyhedral sets]\label{prop:017-polyhedral-yoga}.
	\begin{enumerate}[label=(\alph*)]
		\item (Finite Intersection) Given polyhedral sets $\{P_i\}_{i=1}^{k}\subset \mathbb{R}^m$, $\bigcap_{i=1}^kP_i$ is polyhedral.
		\item (Finite Product) Given polyhedral sets $\{P_i\}_{i=1}^{k}$ with $P_i\subset \mathbb{R}^{m_i}$, $\bigtimes_{i=1}^kP_i$ is polyhedral.
		\item (Direct Image) Given affine map $A:\mathbb{R}^m\to \mathbb{R}^n$, polyhedral $P\subset \mathbb{R}^m$, $AP$ is polyhedral.
		\item (Inverse Image) Given affine map $A:\mathbb{R}^l\to \mathbb{R}^m$, polyhedral $P\subset \mathbb{R}^m$, $A^{-1}P$ is polyhedral if nonempty.
		\item (Linear Combination) Given polyhedral sets $\{P_i\}_{i=1}^{k}\subset \mathbb{R}^m$ and $\{a_i\}_{i=1}^k\subset \mathbb{R}^m$, $\sum_{i=1}^ka_iP_i$ is polyhedral.
	\end{enumerate}
\end{prop}
\begin{proof}
	For (a), note that a finite intersection of finite intersection of halfspaces is a finite intersection of halfspaces. For (b), may assume $k=2$, then use the identity $E_{a,b,-}\times \mathbb{R}^{m_2}=E_{(a,O),b,-}$. For (c), use \Cref{prop:017-Minkowski-Weyl-Representation}. For (d), one can verify it by writing out the defining equation of $A^{-1}P$ using a defining equation of $P$ and an explicit representation of $A$. For (e), use (b), (c).
\end{proof}

\subsection{Polyhedral Sets and Separation Theorems}

\paragraph{}We formulated several hyperplane separation notions and facts in \Cref{sect:015} (see \Cref{prop:015-hyperplane-sep}). When dealing with polyhedral sets, some of these have interesting implications in the polyhedral settings:

\begin{enumerate}[label=(\arabic*)]
	\item For strict separation theorem ((b) of \Cref{prop:015-hyperplane-sep}), the condition $C_1-C_2$ can be automatically verified in some cases, utilizing results from \Cref{sect:014} and this section (see \Cref{prop:017-closedness-vec-diff}).
	\item For proper separation theorem ((c) of \Cref{prop:015-hyperplane-sep}), the condition $\operatorname{ri}(C_1)\cap\operatorname{ri}(C_2)=\emptyset$ has some interesting variants if one of them is polyhedral (see \Cref{prop:017-polyhedral-proper-sep}); in particular, one can control that the separating hyperplane to not contain the non-polyhedral convex set (in other words: the "properness" can be fine-tuned).
\end{enumerate}

\paragraph{}The following fact was mentioned previously (see \Cref{defn:017-fg-cones}).
\begin{lemm}[Polyhedral Sets are Retractive]\label{lemm:017-polyhedral-retractive}
	Polyhedral sets are retractive.
\end{lemm}

\begin{prop}[Closedness of Vector Difference]\label{prop:017-closedness-vec-diff}
	Let $C_1,C_2$ be closed convex subsets of $\mathbb{R}^m$. Then $C_1-C_2$ is closed if any of the following holds:
	\begin{enumerate}[label=(\alph*)]
		\item $R_{C_1}\cap R_{C_2}=L_{C_1}\cap L_{C_2}$.
		\item $C_1$ is polyhedral with $R_{C_1}\cap R_{C_2}\subset L_{C_2}$.
		\item $C_1$ is compact.
		\item $C_1,C_2$ are both polyhedral.
	\end{enumerate}
\end{prop}

\begin{proof}
	For (c), use (a). For (d), use (e) of \Cref{prop:017-polyhedral-yoga}. For (a), use \Cref{coro:014-closed-sum}. For (b), use \Cref{coro:014-closed-affine}, \Cref{lemm:017-polyhedral-retractive} with $C_0=C_1\times \mathbb{R}^{m}$, $C=\mathbb{R}^m\times C_2$, $A:\mathbb{R}^{m}\times \mathbb{R}^m\to \mathbb{R}^m$ be $(x,y)\mapsto x-y$.
\end{proof}

\begin{prop}[Polyhedral Proper Separation]\label{prop:017-polyhedral-proper-sep}
	Let $C_1=C,C_2=P$ be nonempty convex subsets of $\mathbb{R}^m$ with $P$ polyhedral, then $\operatorname{ri}(C)\cap P=\emptyset$ iff there are $a\in \mathbb{S}^{m-1},b\in \mathbb{R}^m$ with $H_{a,b}\nsupset C\subset E_{a,b,-},P\subset E_{a,b,+}$.
\end{prop}

\paragraph{}The proof of this proposition is quite involved. We introduce some lemmas before going into the proof:

\begin{lemm}[Hyperplane Containment Criterion]\label{lemm:017-criterion-hyperplane-containment}
	Given nonempty convex set $C\subset \mathbb{R}^m$, $a\in \mathbb{S}^{m-1},b\in \mathbb{R}$, with $C\subset E_{a,b,-}$, then $C\subset H_{a,b}$ iff $\operatorname{ri}(C)\cap H_{a,b}\neq\emptyset$.
\end{lemm}

\begin{proof}
	We just show "If". May assume $O\in \operatorname{ri}(C)\cap H_{a,b}$ (so $b=0$). Write $V=\operatorname{lin}(C)$, then $V\subset H_{a,0}$: otherwise take $v\in V\cap E_{a,0,-}$, then $\varepsilon v\subset \operatorname{ri}(C)$ for some $\varepsilon >0$, giving $\operatorname{ri}(C)\cap E_{a,0,-}\neq\emptyset$, a contradiction.
\end{proof}

\begin{lemm}[Cone of Polyhedral Set]\label{lemm:017-polyhedral-cone-cone-lemma}
	For a polyhedral set $P$ with $O\in P$, $\operatorname{cone}(P)$ is a polyhedral cone.
\end{lemm}
\begin{proof}
	If $P=\left(\bigcap_{x\in X}E_{x,0,-}\right)\cap \left(\bigcap_{y\in Y}E_{y,1,-}\right)$ for finite subsets $X,Y$ of $\mathbb{R}^m$, $\operatorname{cone}(P)=\bigcap_{x\in X}E_{x,0,-}$.
\end{proof}

\begin{lemm}[Polyhedral Proper Separation - Special Cases]\label{lemm:017-polyhedral-proper-sep-special-case}
	In the settings of \Cref{prop:017-polyhedral-proper-sep}, write $V= \operatorname{aff}(C)$. We can properly separate $C, P$ by a hyperplane not containing $C$ in any of the following cases:
	\begin{enumerate}[label=(\alph*)]
		\item If $P\subset V$.
		\item If we know we can do so with $C$ replaced by some convex $C'$ for some $C\subset C'\subset V$.
		\item If $C'\cap P=\emptyset$ for some polyhedral $C'$ with $C\subset C'\subset V$.
		\item If the following conditions are all satisfied:
		      \begin{itemize}
			      \item both $P,C$ are polyhedral cones
			      \item $C$ has a representation $C=AE_{e_1,0,-}$ for some linear injection $A:\mathbb{R}^d\to \mathbb{R}^m$
			      \item and that $P\supset AH_{e_1,0}$, the "relative boundary" of $C$; here we denote this as $\operatorname{rb}(C)$.
		      \end{itemize}
		      In this case, we can take the separating hyperplane to pass $O$.
	\end{enumerate}
\end{lemm}

\begin{proof}
	For (a), use (e) of \Cref{prop:017-polyhedral-proper-sep}, and note that a hyperplane containing $C$ contains $P$. For (b), note that a hyperplane containing $C$ contains $C'$. For (c), one can use (d) of \Cref{prop:017-closedness-vec-diff} and (b) of \Cref{prop:015-hyperplane-sep} to find a strict separation for $C', P$, and apply (b). For (d), write $P=\bigcap_{x\in X}E_{x,0,-}$ for some finite subset $X\subset \mathbb{R}^m\smallsetminus \{O\}$, then:
	\[
		P\cap \operatorname{ri}(C)=\bigcap_{x\in X}\left(E_{x,0,-}\cap \operatorname{ri}(C)\right)=\emptyset
	\]
	The intersection $\left(E_{x,0,-}\cap \operatorname{ri}(C)\right)$ is either empty or contains $C$, as for any $y\in E_{x,0,-}\cap\operatorname{ri}(C)$, we have:
	\[
		E_{x,0,-}\supset \operatorname{c.cone}\left(\operatorname{rb}(C)\cup \{y\}\right) = A\operatorname{c.cone}\left(H_{e_1,0} \cup \{A^{-1}y\}\right) = AE_{e_1,0,-} = C
	\]
	Therefore, $E_{x_0,0,-}\cap \operatorname{ri}(C)=\emptyset$ for some $x_0\in X$, giving $H\nsupset C\subset E_{x_0,0,+}$ and $P\subset E_{x_0,0,-}$.
\end{proof}

\begin{proof}[Proof of \Cref{prop:017-polyhedral-proper-sep}]
	For "If", assuming $H_{a,b}\nsupset C\subset E_{a,b,-},P\subset E_{a,b,+}$, by \Cref{lemm:017-criterion-hyperplane-containment}, we have:
	\[
		\operatorname{ri}(C)\cap P\subset \operatorname{ri}(C)\cap E_{a,b,+}=\operatorname{ri}(C)\cap H_{a,b}=\emptyset
	\]
	Now we consider "Only if". For what follows, given subsets $X,Y$ in $\mathbb{R}^m$, $X_Y:=X\cap Y$. Write $V=\operatorname{aff}(C)$.
	\begin{itemize}
		\item Separate $P_V,C$: By (c) of \Cref{lemm:017-polyhedral-proper-sep-special-case}, may assume $P_V\neq\emptyset$. By (a) of \Cref{lemm:017-polyhedral-proper-sep-special-case}, we get hyperplane and halfspaces $H,E_{\pm}$ with $H\nsupset C\subset E_-,P_V\subset E_+$.
		      % \[
		      %  \begin{tikzcd}[column sep=2ex,row sep=2ex]
		      %   H\rar{}&E_-\rar{}&\mathbb{R}^m&\lar{}E_+&\lar{}P_V\\
		      %   H_V\rar{}\uar{}&E_{-,V}\rar{}\uar{}&V\uar{}&\lar{}E_{+,V}\uar{}&\lar{}P_V\uar{}\\
		      %   H_C\rar{}\uar{}&C\rar{}\uar{}&C\uar{}&\lar{}H_{C}\uar{}&\lar{}P_C\uar{}\\
		      %  \end{tikzcd}
		      % \]
		      % with the property that for each subgraph $A\to B$, $A\subset B$, and for each subgraph $A\leftarrow B\rightarrow C$, we have $B=A\cap C$. Note that the second and third rows are just the intersection of the first row with $V$ or $C$.
		\item Separate $P,E_{-,V}$: Note that $E_{-,V}\supset C$ is a polyhedral cone (after translation), with:
		      \[
			      P\cap \operatorname{ri}(E_{-,V})=P_V\cap E^o_{-} =\emptyset,\;\text{ as }P_V\subset E_+
		      \]
		      By (c) of \Cref{lemm:017-polyhedral-proper-sep-special-case}, may assume $O\in P\cap E_{-,V}=P\cap H_{-,V}\neq\emptyset$ to make $H$, $V$ linear.
		\item Separate $K:=\operatorname{cone}(P)+H_V,E_{-,V}$: Note that $K\supset P$ and $E_{-,V}\supset C$ are polyhedral cones (see (e) of \Cref{prop:017-polyhedral-yoga} and \Cref{lemm:017-polyhedral-cone-cone-lemma}). We are done by (d) and (c) of \Cref{lemm:017-polyhedral-proper-sep-special-case} once we know $K\cap \operatorname{ri}(E_{-,V})=\emptyset$. This is by:
		      \[
			      K\cap \operatorname{ri}(E_{-,V}) = \operatorname{cone}(P) \cap \operatorname{ri}(E_{-,V}) = \operatorname{cone}(P \cap \operatorname{ri}(E_{-,V})) = \emptyset
			      \qedhere
		      \]
	\end{itemize}
\end{proof}
%
% \begin{rmrk}
% 	The proof of this proposition is rather technical, but the implications are useful (both the "if" and "only if"). In particular, we will use this to study Slater's condition (to be introduced in \Cref{chap:04}), where polyhedral proper separation is used for the study of affine constraints. It is worth mentioning some possible motivations of the steps in the proof of the "only if" part:
% 	\begin{itemize}
% 		\item Recal that the main goal is to find a hyperplane separation such that the hyperplane $H$ does not contain $C$. Ordinary proper separation shows that we can find proper separation, but with no guaranteeing that there is no containment of the hyperplane and $C$.
% 		\item The consideration for a proper separation between $C,P_V$ has some implications:
% 		      \begin{itemize}
% 			      \item Since $P_V\subset V=\operatorname{aff}(C)$, a hyperplane separating these two sets cannot contain $C$.
% 			      \item However, this hyperplane needn't separate $P$ and $C$, so one proceeds with the next idea.
% 		      \end{itemize}
% 		\item
% 	\end{itemize}
% \end{rmrk}
