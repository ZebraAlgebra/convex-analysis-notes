\section{Polyhedral Sets}
\label{sect:017}

\paragraph{}We have seen in \Cref{coro:015-halfspaces-intersection} that closed convex sets are essentially intersections of closed halfspaces. A special class of closed convex sets - called polyhedral - are the ones that are given by finite intersection of closed convex halfspaces.

\begin{defn}[Polyhedral Sets and Cones]\label{defn:017-polyhedral-sets}
	A set $P\subset \mathbb{R}^m$ is polyhedral if it is of the following form:
	\[
		P=\bigcap_{i=1}^rE_{a_i,b_i,-},\text{ for some }r\in \mathbb{N},\{a_i\}_{i=1}^r\subset\mathbb{R}^m,\{b_i\}_{i=1}^r\subset\mathbb{R}
	\]
	Equivalently, it is the set:
	\[
		P=\{v\in \mathbb{R}^m:Av\leq b\}
	\]
	where $A\in \mathbf{M}_{r,m}(\mathbb{R}),b\in \mathbb{R}^r$ given by $A_{i,:}=a_i^T,b_i=b_i$. Under such representation, we call $Av\leq b$ a defining equation of $P$. A polyhedral set is a polyhedral cone if the $b_i$'s can be taken to be $0$.
\end{defn}

\paragraph{}Here $\mathbf{M}_{r,m}(\mathbb{R})$ is the space of $r$ by $m$ matrices, and for a given matrix $A$, $A_{i,:}$ is the $i$-th row, $A_{:,i}$ is the $i$-th column, and the inequality is to be interpreted componentwise-ly.

\paragraph{}Note that a polyhedral cone is precisely a polyhedral set that is also a cone: suppose a cone $X$ is contained in some $E_{a,b,-}$, then $b\geq 0$ and $X\subset E_{a,0,-}$.

\begin{exmp}
	A "linear programming (LP) problem in standard form" is given as the follows:
	\begin{align*}
		\text{Maximize}\quad   & c^Tx              \\
		\text{subject to}\quad & Ax\leq b, x\geq 0
	\end{align*}
	where $A\in \mathbf{M}_{r,m}(\mathbb{R}),b\in \mathbb{R}^r,c\in \mathbb{R}^m$. In such case, the region given by the constraints (called the "feasibility region") is a polyhedral set with a defining equation of the form:
	\[
		\left[
			\begin{array}{r}
				A \\\hline
				-I
			\end{array}
			\right]
		x\leq
		\left[
			\begin{array}{r}
				b \\\hline
				0
			\end{array}
			\right]
	\]
\end{exmp}

\subsection{Extreme Points of Polyhedral Sets}

\paragraph{}Extreme points of polyhedral sets admits quite explicit descriptions:

\begin{prop}[Characterization of Extreme points of Polyhedral Sets]\label{prop:017-characterization-polyhedral-extreme}
	Given a defining equation $Ax\leq b$ of a polyhedral set $P$ for some $A\in \mathbf{M}_{r,m}(\mathbb{R}),b\in \mathbb{R}^m$, a point $v\in P$ satisfies $v\in \operatorname{ext}(P)$ iff the matrix $A_v$ given by keeping rows in $A$ where "$=$" holds in $Av\leq b$ is non-empty and injective.
\end{prop}

\begin{proof}
	Let $I_v$ be a subset of row indices $\{i:A_{i,:}v=b_i\}$. If $A_v$ is injective, then:
	\[
		\operatorname{ext}(P)\cap\left(\bigcap_{i\in I_v}H_{a_i,b_i}\right)=
		\operatorname{ext}\left(P\cap\bigcap_{i\in I_v}H_{a_i,b_i}\right)=\operatorname{ext}\left(\{v\}\right)=\{v\}
	\]
	by \Cref{lemm:016-extreme-and-hyperlane}. Otherwise, notice:
	\[
		v\in\left(\bigcap_{i\notin I_v}E_{a_i,b_i,-}^o\right)\cap\left(\bigcap_{i\in I_v}H_{a_i,b_i}\right)\subset P
	\]
	If $I_v$ (and hence $A_v$) is empty, then $P$ contains an open neighborhood of $v$. If $A_v$ is not injective, take $w\in \operatorname{ker}(A_v)\smallsetminus\{0\}$, then there is some $\varepsilon >0$ with $\{v\pm \varepsilon w\}\subset P$. In each case, $v$ cannot be extreme.
\end{proof}

\begin{coro}[Variants]\label{coro:017-char-variants}
	Let $P$ be a polyhedral set, $A\in \mathbf{M}_{r,m}(\mathbb{R}),b,c,d\in \mathbb{R}^m$,
	\begin{enumerate}[label=(\alph*)]
		\item Suppose $P=\{x\in \mathbb{R}^m:Ax=b,x\geq 0\}$. Given $v\in P$, we have $v\in \operatorname{ext}(P)$ iff the matrix $A'_v$ given by keeping columns in $A$ corresponding to indices of $v$ where $v> 0$ is empty or injective.
		\item Suppose $P=\{x\in \mathbb{R}^m:Ax=b,c\leq x\leq d\}$. Given $v\in P$, we have $v\in \operatorname{ext}(P)$ iff the matrix $A'_v$ given by keeping columns in $A$ corresponding to indices of $v$ where $c<v<d$ is empty or injective.
	\end{enumerate}
\end{coro}

\begin{proof}
	One can write the defining equations of (a) and (b) in block form as:
	\[
		\left[
			\begin{array}{r}
				A  \\\hline
				-A \\\hline
				-I
			\end{array}
			\right]
		x\leq
		\left[
			\begin{array}{r}
				b  \\\hline
				-b \\\hline
				0
			\end{array}
			\right],\;
		\left[
			\begin{array}{r}
				A  \\\hline
				-A \\\hline
				I  \\\hline
				-I
			\end{array}
			\right]
		x\leq
		\left[
			\begin{array}{r}
				b  \\\hline
				-b \\\hline
				d  \\\hline
				-c
			\end{array}
			\right]
	\]
	The rest is an application of \Cref{prop:017-characterization-polyhedral-extreme}.
\end{proof}

\begin{coro}[Nonemptiness of $\operatorname{ext}(P)$ - Polyhedral Case]\label{coro:017-extreme-existence-poly}
	Suppose $P\subset \mathbb{R}^m$ is nonempty, polyhedral with defining equation $Ax\leq b$, then $\operatorname{ext}(P)\neq\emptyset$ iff $A$ is injective and nonempty.
\end{coro}
\begin{proof}
	If $A$ is injective, then $P$ contains no lines (see \Cref{lemm:016-extreme-existence}). Otherwise, use \Cref{prop:017-characterization-polyhedral-extreme}.
\end{proof}

\subsection{Polyhedral Cones and Finitely-Generated Cones}

\paragraph{}A form of Farkas lemma characterizes polyhedral cones - they are the polars of finitely generated cones, a notion we shall define now. Minkowski-

\begin{defn}[Finitely Generated Cones]\label{defn:017-fg-cones}
	A convex cone $C$ is called finitely generated if it can be written as $\operatorname{c.conv}(X)$ for some finite subset $X=\{a_i\}_{i=1}^k\subset \mathbb{R}^m$.
\end{defn}

\paragraph{}Finitely generated cones are closed - they are images of $\mathbb{R}_+^k\subset \mathbb{R}^k$ under some linear map, so by \Cref{coro:014-closed-affine} applied to $C_0=\mathbb{R}_+^k$ (this set is retractive - being a finite intersection of halfspaces which are also retractive) and $C=\mathbb{R}^k$, it is closed.

\begin{prop}[Farkas Lemma - Basic Form]\label{prop:017-farkas}
	Given a finite set $X\subset \mathbb{R}$, the closed convex cones:
	\[
		\bigcap_{a\in X}E_{a,0,-},\;\operatorname{c.cone}(X)
	\]
	are polar to each other.
\end{prop}

\begin{proof}
	By definition we have $X^\ast=\bigcap_{a\in X}E_{a,0,-}$, so we are done by \Cref{lemm:016-polar-cone-theorem}.
\end{proof}

\begin{exmp}[An application of Farkas Lemma]
	Let $X,Y'$ be finite subsets of $\mathbb{R}^m$, then:
	\[
		\left(\left(\bigcap_{a\in X}E_{a,0,-}\right)\cap \left(\bigcap_{a'\in X'}H_{a',0}\right)\right)^\ast
		=\operatorname{lin}(X')+\operatorname{c.cone}(X)
	\]
\end{exmp}

\begin{prop}[Minkowski-Weyl Theorem]\label{prop:017-Minkowski-Weyl-Theorem}
	A convex cone $C\subset \mathbb{R}^m$ is polyhedral iff finitely generated.
\end{prop}
\begin{proof}
	By \Cref{prop:017-farkas} and \Cref{lemm:016-polar-cone-theorem}, it suffices to show that a finitely generated cone is polyhedral. Suppose $X=\{a_i\}_{i=1}^{k}\subset C$ with $C=\operatorname{c.cone}(X)$. If we write $V=\operatorname{lin}(C)$, then by the identity:
	\[
		C=(C+V^\perp)\cap V
	\]
	we may assume $V=\mathbb{R}^m\neq C$. Given $b\notin C$, by (b) of \Cref{prop:015-hyperplane-sep}, there are $x\in \mathbb{S}^{m-1},y\in \mathbb{R}$ with:
	\[
		b\in E_{x,2y,+}^o,\; C\subset E_{x,2y,-}^o
	\]
	Since $C$ is a non-trivial cone, we have $y>0$, giving:
	\[
		b\in E_{x,y,+},\; C\subset E_{x,0,-}
	\]
	In this sense, $x$ belongs to the following polyhedral set:
	\[
		P = E_{-b,-y,-}\cap\left(\bigcap_{i=1}^k E_{a_i,0,-}\right)
	\]
	In terms of defining equations (where $A_{i,:}=a_i^T$):
	\[
		P=
		\left\{x\in \mathbb{R}^m:\overline{A}x\leq
		\left[
			\begin{array}{r}
				0 \\\hline
				-y
			\end{array}
			\right]
		,\text{ where }
		\overline{A}=\left[
			\begin{array}{r}
				A \\\hline
				-b^T
			\end{array}
			\right]
		\right\}
	\]
	Extreme points of this set is given by \Cref{prop:017-characterization-polyhedral-extreme}; say $v\in \operatorname{ext}(P)$, and consider the matrix $\overline{A_v}$, which should be injective. As $A$ is injective and $0\notin P$, we have:
	\[
		\overline{A_v}=
		\left[
			\begin{array}{r}
				A_v \\\hline
				-b^T
			\end{array}
			\right]
	\]
	and $\operatorname{rank}(A_v)=m-1$, giving:
	\[
		b\in E^o_{v,0,+},\; \operatorname{im}(A_v)=H_{v,0},\; C\subset E_{v,0,-}
	\]
	This gives:
	\[
		C = \bigcap_{\substack{
				v\in \mathbb{R}^{m-1}\smallsetminus\{0\},Y\subset X:\\
				\dim \operatorname{lin}(Y)=m-1,\;C\subset E_{v,0,-},\\
				H_{v,0}=\operatorname{lin}(Y)
			}
		}E_{v,0,-}=
		\bigcap_{\substack{
				v\in \mathbb{S}^{m-1},Y\subset X:\\
				\dim \operatorname{lin}(Y)=m-1,\;C\subset E_{v,0,-},\\
				H_{v,0}=\operatorname{lin}(Y)
			}
		}E_{v,0,-}
	\]
	which is a finite intersection of halfspaces (as for given $Y$ with $\operatorname{dim}\operatorname{lin}(Y)=m-1$, the conditions $C\subset E_{v,0,-},H_{v,0}=\operatorname{lin}(Y)$ determines $v\in \mathbb{S}^{m-1}$ uniquely).
\end{proof}

\begin{prop}[Minkowski-Weyl Representation]\label{prop:017-Minkowski-Weyl-Representation}
	A set $P\subset \mathbb{R}^m$ is polyhedral iff it is of the form $\operatorname{c.cone}(X)+\operatorname{conv}(Y)$ for some finite subsets $X,Y$ of $\mathbb{R}^m$.
\end{prop}
\begin{proof}
	If $P=\operatorname{c.cone}(X)+\operatorname{conv}(Y)$, then $P=\overline{P}_{:,1}$ where $\overline{P}\subset \mathbb{R}^{m+1}$ is defined as:
	\[
		\overline{P}=\operatorname{c.cone}\left(\{(x,0):x\in X\}\cup\{(y,1):y\in Y\}\right)
	\]
	therefore both $\overline{P}$ and $P$ are polyhedral by \Cref{prop:017-Minkowski-Weyl-Theorem}. On the other hand, if $P$ is polyhedral, with:
	\[
		P=\left(\bigcap_{x\in X} E_{x,0,-}\right)\cap\left(\bigcap_{y\in Y} E_{y,1,-}\right)\cap \left(\bigcap_{z\in Z} E_{z,-1,-}\right)
	\]
	where $X,Y,Z$ are finite subsets of $\mathbb{R}^m\smallsetminus\{0\}$, then $P=\overline{P}_{:,1}$ where $\overline{P}\subset \mathbb{R}^{m+1}$ is defined as:
	\[
		\overline{P}=E_{-e_{m+1},0,-}\cap\left(\bigcap_{x\in X} E_{(x,0),0,-}\right)\cap\left(\bigcap_{y\in Y} E_{(y,-1),0,-}\right)\cap \left(\bigcap_{z\in Z} E_{(z,1),0,-}\right)
	\]
	By \Cref{prop:017-Minkowski-Weyl-Theorem}, we may write:
	\[
		\overline{P}=\operatorname{c.cone}\left(\{(x,0):x\in X'\}\cup \{(y, 1):y\in Y'\}\right)
	\]
	for some finite subsets $X',Y'$ in $\mathbb{R}^m$. Under this representation, $P=\overline{P}_{:,1}=\operatorname{c.cone}(X')+\operatorname{conv}(Y')$.
\end{proof}
