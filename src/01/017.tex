\section{Polyhedral Sets}
\label{sect:017}

\paragraph{}We have seen in \Cref{coro:015-halfspaces-intersection} that closed convex sets are essentially intersections of closed halfspaces. A special class of closed convex sets - called polyhedral - are the ones that are given by finite intersection of closed convex halfspaces.

\begin{defn}[Polyhedral Sets and Cones]\label{defn:017-polyhedral-sets}
	A set $P\subset \mathbb{R}^m$ is polyhedral if it is of the following form:
	\[
		P=\bigcap_{i=1}^rE_{a_i,b_i,-},\text{ for some }r\in \mathbb{N},\{a_i\}_{i=1}^r\subset\mathbb{R}^m,\{b_i\}_{i=1}^r\subset\mathbb{R}
	\]
	Equivalently, it is the set:
	\[
		P=\{v\in \mathbb{R}^m:Av\leq b\}
	\]
	where $A\in \mathbf{M}_{r,m}(\mathbb{R}),b\in \mathbb{R}^r$ given by $A_{i,:}=a_i^T,b_i=b_i$. Under such representation, we call $Av\leq b$ a defining equation of $P$. A polyhedral set is a polyhedral cone if the $b_i$'s can be taken to be $0$.
\end{defn}

\paragraph{}Here $\mathbf{M}_{r,m}(\mathbb{R})$ is the space of $r$ by $m$ matrices, and for a given matrix $A$, $A_{i,:}$ is the $i$-th row, $A_{:,i}$ is the $i$-th column, and the inequality is to be interpreted componentwise-ly.

\paragraph{}Note that a polyhedral cone is precisely a polyhedral set that is also a cone: suppose a cone $X$ is contained in some $E_{a,b,-}$, then $b\geq 0$ and $X\subset E_{a,0,-}$.

\begin{exmp}
	A "linear programming (LP) problem in standard form" is given as the follows:
	\begin{align*}
		\text{Maximize}\quad   & c^Tx              \\
		\text{subject to}\quad & Ax\leq b, x\geq 0
	\end{align*}
	where $A\in \mathbf{M}_{r,m}(\mathbb{R}),b\in \mathbb{R}^r,c\in \mathbb{R}^m$. In such case, the region given by the constraints (called the "feasibility region") is a polyhedral set with a defining equation of the form:
	\[
		\left[
			\begin{array}{r}
				A \\\hline
				-I
			\end{array}
			\right]
		x\leq
		\left[
			\begin{array}{r}
				b \\\hline
				0
			\end{array}
			\right]
	\]
\end{exmp}

\subsection{Extreme Points of Polyhedral Sets}

\paragraph{}Extreme points of polyhedral sets admits quite explicit descriptions:

\begin{prop}[Characterization of Extreme points of Polyhedral Sets]\label{prop:017-characterization-polyhedral-extreme}
	Given a defining equation $Ax\leq b$ of a polyhedral set $P$ for some $A\in \mathbf{M}_{r,m}(\mathbb{R}),b\in \mathbb{R}^m$, a point $v\in P$ satisfies $v\in \operatorname{ext}(P)$ iff the matrix $A_v$ given by keeping rows in $A$ where "$=$" holds in $Av\leq b$ is non-empty and injective.
\end{prop}

\begin{proof}
	Let $I_v$ be a subset of row indices $\{i:A_{i,:}v=b_i\}$. If $A_v$ is injective, then:
	\[
		\operatorname{ext}(P)\cap\left(\bigcap_{i\in I_v}H_{a_i,b_i}\right)=
		\operatorname{ext}\left(P\cap\bigcap_{i\in I_v}H_{a_i,b_i}\right)=\operatorname{ext}\left(\{v\}\right)=\{v\}
	\]
	by \Cref{lemm:016-extreme-and-hyperlane}. Otherwise, notice:
	\[
		v\in\left(\bigcap_{i\notin I_v}E_{a_i,b_i,-}^o\right)\cap\left(\bigcap_{i\in I_v}H_{a_i,b_i}\right)\subset P
	\]
	If $I_v$ (and hence $A_v$) is empty, then $P$ contains an open neighborhood of $v$. If $A_v$ is not injective, take $w\in \operatorname{ker}(A_v)\smallsetminus\{0\}$, then there is some $\varepsilon >0$ with $\{v\pm \varepsilon w\}\subset P$. In each case, $v$ cannot be extreme.
\end{proof}

\begin{coro}[Variants]\label{coro:017-char-variants}
	Let $P$ be a polyhedral set, $A\in \mathbf{M}_{r,m}(\mathbb{R}),b,c,d\in \mathbb{R}^m$,
	\begin{enumerate}[label=(\alph*)]
		\item Suppose $P=\{x\in \mathbb{R}^m:Ax=b,x\geq 0\}$. Given $v\in P$, we have $v\in \operatorname{ext}(P)$ iff the matrix $A'_v$ given by keeping columns in $A$ corresponding to indices of $v$ where $v> 0$ is empty or injective.
		\item Suppose $P=\{x\in \mathbb{R}^m:Ax=b,c\leq x\leq d\}$. Given $v\in P$, we have $v\in \operatorname{ext}(P)$ iff the matrix $A'_v$ given by keeping columns in $A$ corresponding to indices of $v$ where $c<v<d$ is empty or injective.
	\end{enumerate}
\end{coro}

\begin{proof}
	One can write the defining equations of (a) and (b) in block form as:
	\[
		\left[
			\begin{array}{r}
				A  \\\hline
				-A \\\hline
				-I
			\end{array}
			\right]
		x\leq
		\left[
			\begin{array}{r}
				b  \\\hline
				-b \\\hline
				0
			\end{array}
			\right],\;
		\left[
			\begin{array}{r}
				A  \\\hline
				-A \\\hline
				I  \\\hline
				-I
			\end{array}
			\right]
		x\leq
		\left[
			\begin{array}{r}
				b  \\\hline
				-b \\\hline
				d  \\\hline
				-c
			\end{array}
			\right]
	\]
	The rest is an application of \Cref{prop:017-characterization-polyhedral-extreme}.
\end{proof}

\begin{coro}[Nonemptiness of $\operatorname{ext}(P)$ - Polyhedral Case]\label{coro:017-extreme-existence-poly}
	Suppose $P\subset \mathbb{R}^m$ is nonempty, polyhedral with defining equation $Ax\leq b$, then $\operatorname{ext}(P)\neq\emptyset$ iff $A$ is injective and nonempty.
\end{coro}
\begin{proof}
	If $A$ is injective, then $P$ contains no lines (see \Cref{lemm:016-extreme-existence}). Otherwise, use \Cref{prop:017-characterization-polyhedral-extreme}.
\end{proof}

\begin{exmp}[Birkhoff-von Neumann Theorem]
	Fix $m\in \mathbb{N}$. Identify $\mathbf{M}_{m,m}(\mathbb{R})$ with $\mathbb{R}^{m^2}$ via $g:e_{i,j}\mapsto e_{i+m(j-1)}$ (that is, flattening by stacking columns). The space of doubly stochastic matrices - those with entries in $[0,1]$ and with each column and row summing to $1$ - denoted as $P_{m}$, contains the set of permutation matrices - denoted as $X_m$. We show that $\operatorname{ext}(P_m)=X_m$:
	\begin{itemize}
		\item $X_m\subset \operatorname{ext}(P_m)$: we have $J(g(P_m))\subset J\left([0,1]^{m^2}\right)$, $g(X_m)\subset \{0,1\}^m=\operatorname{ext}\left([0,1]^{m^2}\right)$.
		\item $X_m\supset \operatorname{ext}(P_m)$: the idea is to use \Cref{prop:017-characterization-polyhedral-extreme}. Note that $g(P_m)$ can be described as:
		      \[
			      g(P_m)=\left\{
			      x\in \mathbb{R}^{m^2}:x\geq 0,Ax=1
			      \right\}
		      \]
		      where $A\in \mathbf{M}_{2m,m^2}(\mathbb{R})$ has rows encoding the requirement that each row (resp. column) sum to $1$; note that $A$ is not injective, so if $M\in \operatorname{ext}(P_m)$, $M$ has at most $(2m-1)$ zero entries, so $M_{i,j}=1$ for some $i,j$ (pigeon-hole principle). Now we show $M\in X_m$ by induction using this $i,j$: if $m=1$, $i=j=1$, and we are done; otherwise, take $M_{(i,j)}$, defined as $M$ with $i$-th row and $j$-th column removed, lies in $\operatorname{ext}(P_{m-1})$ (by \Cref{lemm:016-extreme-and-hyperlane}), so by inductive hypothesis, $M_{(i,j)}\in X_{m-1}$, and hence $M\in X_m$.
	\end{itemize}
\end{exmp}

\subsection{Polyhedral Cones and Finitely-Generated Cones}

\paragraph{}A form of Farkas lemma characterizes polyhedral cones - they are the polars of finitely generated cones, a notion we shall define now. Minkowski-

\begin{defn}[Finitely Generated Cones]\label{defn:017-fg-cones}
	A convex cone $C$ is called finitely generated if it can be written as $\operatorname{c.conv}(X)$ for some finite subset $X=\{a_i\}_{i=1}^k\subset \mathbb{R}^m$.
\end{defn}

\paragraph{}Finitely generated cones are closed - they are images of $\mathbb{R}_+^k\subset \mathbb{R}^k$ under some linear map, so by \Cref{coro:014-closed-affine} applied to $C_0=\mathbb{R}_+^k$ (this set is retractive - being a finite intersection of halfspaces which are also retractive) and $C=\mathbb{R}^k$, it is closed.

\begin{prop}[Farkas Lemma - Basic Form]\label{prop:017-farkas}
	Given a finite set $X\subset \mathbb{R}$, the closed convex cones:
	\[
		\bigcap_{a\in X}E_{a,0,-},\;\operatorname{c.cone}(X)
	\]
	are polar to each other.
\end{prop}

\begin{proof}
	By definition we have $X^\ast=\bigcap_{a\in X}E_{a,0,-}$, so we are done by \Cref{lemm:016-polar-cone-theorem}.
\end{proof}

\begin{exmp}[An application of Farkas Lemma]
	Let $X,Y'$ be finite subsets of $\mathbb{R}^m$, then:
	\[
		\left(\left(\bigcap_{a\in X}E_{a,0,-}\right)\cap \left(\bigcap_{a'\in X'}H_{a',0}\right)\right)^\ast
		=\operatorname{lin}(X')+\operatorname{c.cone}(X)
	\]
\end{exmp}

\begin{prop}[Minkowski-Weyl Theorem]\label{prop:017-Minkowski-Weyl-Theorem}
	A convex cone $C\subset \mathbb{R}^m$ is polyhedral iff finitely generated.
\end{prop}
\begin{proof}
	By \Cref{prop:017-farkas} and \Cref{lemm:016-polar-cone-theorem}, it suffices to show that a finitely generated cone is polyhedral. Suppose $X=\{a_i\}_{i=1}^{k}\subset C$ with $C=\operatorname{c.cone}(X)$. If we write $V=\operatorname{lin}(C)$, then since $C=(C+V^\perp)\cap V$,	we may assume $V=\mathbb{R}^m\neq C$ by replacing $C$ by $C+V^\perp$. Given $b\notin C$, by (b) of \Cref{prop:015-hyperplane-sep}, there are $x\in \mathbb{S}^{m-1},y\in \mathbb{R}$ with $b\in E_{x,2y,+}^o,\; C\subset E_{x,2y,-}^o$.	Since $C$ is a non-trivial cone, we have $y>0$, giving $b\in E_{x,y,+},\; C\subset E_{x,0,-}$. In this sense, $x$ belongs to the following polyhedral set:
	\[
		P = E_{-b,-y,-}\cap\left(\bigcap_{i=1}^k E_{a_i,0,-}\right)
	\]
	In terms of defining equations (where $A_{i,:}=a_i^T$; note that $A$ is injective, as $V=\mathbb{R}^m$):
	\[
		P=
		\left\{x\in \mathbb{R}^m:\overline{A}x\leq
		\left[
			\begin{array}{r}
				0 \\\hline
				-y
			\end{array}
			\right]
		,\text{ where }
		\overline{A}=\left[
			\begin{array}{r}
				A \\\hline
				-b^T
			\end{array}
			\right]
		\right\}
	\]
	Existence and characterziations of extreme points of this set are given by \Cref{prop:017-characterization-polyhedral-extreme}, \Cref{coro:017-extreme-existence-poly}. Take $v\in \operatorname{ext}(P)$, and consider $\overline{A_v}$, which should be injective. As $0\notin P$, we have:
	\[
		\overline{A_v}=
		\left[
			\begin{array}{r}
				A_v \\\hline
				-b^T
			\end{array}
			\right]
	\]
	and $\operatorname{rank}(A_v)=m-1$, giving:
	\[
		b\in E^o_{v,0,+},\; \operatorname{im}(A_v)=H_{v,0},\; C\subset E_{v,0,-}
	\]
	This gives:
	\[
		C = \bigcap_{\substack{
				v\in \mathbb{R}^{m-1}\smallsetminus\{0\},Y\subset X:\\
				\dim \operatorname{lin}(Y)=m-1,\;C\subset E_{v,0,-},\\
				H_{v,0}=\operatorname{lin}(Y)
			}
		}E_{v,0,-}=
		\bigcap_{\substack{
				v\in \mathbb{S}^{m-1},Y\subset X:\\
				\dim \operatorname{lin}(Y)=m-1,\;C\subset E_{v,0,-},\\
				H_{v,0}=\operatorname{lin}(Y)
			}
		}E_{v,0,-}
	\]
	which is a finite intersection of halfspaces (as for given $Y$ with $\operatorname{dim}\operatorname{lin}(Y)=m-1$, the conditions $C\subset E_{v,0,-},H_{v,0}=\operatorname{lin}(Y)$ determines $v\in \mathbb{S}^{m-1}$ uniquely).
\end{proof}

\begin{prop}[Minkowski-Weyl Representation]\label{prop:017-Minkowski-Weyl-Representation}
	A set $P\subset \mathbb{R}^m$ is polyhedral iff it is of the form $\operatorname{c.cone}(X)+\operatorname{conv}(Y)$ for some finite subsets $X,Y$ of $\mathbb{R}^m$.
\end{prop}
\begin{proof}
	If $P=\operatorname{c.cone}(X)+\operatorname{conv}(Y)$, then $P=\overline{P}_{:,1}$ where $\overline{P}\subset \mathbb{R}^{m+1}$ is defined as:
	\[
		\overline{P}=\operatorname{c.cone}\left(\{(x,0):x\in X\}\cup\{(y,1):y\in Y\}\right)
	\]
	therefore both $\overline{P}$ and $P$ are polyhedral by \Cref{prop:017-Minkowski-Weyl-Theorem}. On the other hand, if $P$ is polyhedral, with:
	\[
		P=\left(\bigcap_{x\in X} E_{x,0,-}\right)\cap\left(\bigcap_{y\in Y} E_{y,1,-}\right)\cap \left(\bigcap_{z\in Z} E_{z,-1,-}\right)
	\]
	where $X,Y,Z$ are finite subsets of $\mathbb{R}^m\smallsetminus\{0\}$, then $P=\overline{P}_{:,1}$ where $\overline{P}\subset \mathbb{R}^{m+1}$ is defined as:
	\[
		\overline{P}=E_{-e_{m+1},0,-}\cap\left(\bigcap_{x\in X} E_{(x,0),0,-}\right)\cap\left(\bigcap_{y\in Y} E_{(y,-1),0,-}\right)\cap \left(\bigcap_{z\in Z} E_{(z,1),0,-}\right)
	\]
	By \Cref{prop:017-Minkowski-Weyl-Theorem}, we may write:
	\[
		\overline{P}=\operatorname{c.cone}\left(\{(x,0):x\in X'\}\cup \{(y, 1):y\in Y'\}\right)
	\]
	for some finite subsets $X',Y'$ in $\mathbb{R}^m$. Under this representation, $P=\overline{P}_{:,1}=\operatorname{c.cone}(X')+\operatorname{conv}(Y')$.
\end{proof}

\subsection{Yogas of Polyhedral Sets}

\paragraph{}We are now ready to formulate some yogas of polyhedral sets:

\begin{prop}[Yoga of Polyhedral sets]\label{prop:017-polyhedral-yoga}.
	\begin{enumerate}[label=(\alph*)]
		\item (Finite Intersection) Given polyhedral sets $\{P_i\}_{i=1}^{k}\subset \mathbb{R}^m$, $\bigcap_{i=1}^kP_i$ is polyhedral.
		\item (Finite Product) Given polyhedral sets $\{P_i\}_{i=1}^{k}$ with $P_i\subset \mathbb{R}^{m_i}$, $\bigtimes_{i=1}^kP_i$ is polyhedral.
		\item (Direct Image) Given affine map $A:\mathbb{R}^m\to \mathbb{R}^n$, polyhedral $P\subset \mathbb{R}^m$, $AP$ is polyhedral.
		\item (Inverse Image) Given affine map $A:\mathbb{R}^l\to \mathbb{R}^m$, polyhedral $P\subset \mathbb{R}^m$, $A^{-1}P$ is polyhedral.
		\item (Linear Combination) Given polyhedral sets $\{P_i\}_{i=1}^{k}\subset \mathbb{R}^m$ and $\{a_i\}_{i=1}^k\subset \mathbb{R}^m$, $\sum_{i=1}^ka_iP_i$ is polyhedral.
	\end{enumerate}
\end{prop}
\begin{proof}
	For (a), note that a finite intersection of finite intersection of halfspaces is a finite intersection of halfspaces. For (b), may assume $k=2$, then use the identity $E_{a,b,-}\times \mathbb{R}^{m_2}=E_{(a,0_{m_2}),b,-}$. For (c), use \Cref{prop:017-Minkowski-Weyl-Representation}. For (d), note that the inverse image of a halfspace under an affine map is still a halfspace. For (e), use (b), (c).
\end{proof}

\subsection{Polyhedral Sets and Separation Theorems}

\paragraph{}We formulated several hyperplane separation notions and facts in \Cref{sect:015} (see \Cref{prop:015-hyperplane-sep}). When dealing with polyhedral sets, some of these have interesting implications in the polyhedral settings:

\begin{enumerate}[label=(\arabic*)]
	\item For strict separation theorem ((b) of \Cref{prop:015-hyperplane-sep}), the condition $C_1-C_2$ can be automatically verifeid in some cases, utilizing results from \Cref{sect:014} and this section (see \Cref{prop:017-closedness-vec-diff}).
	\item For proper separation theorem ((c) of \Cref{prop:015-hyperplane-sep}), the condition $\operatorname{ri}(C_1)\cap\operatorname{ri}(C_2)=\emptyset$ can be relaxed if one of them is polyhedral (see \Cref{prop:017-polyhedral-proper-sep}).
\end{enumerate}

\paragraph{}The following fact was mentioned previously (see \Cref{defn:017-fg-cones}).
\begin{lemm}[Polyhedral Sets are Retractive]\label{lemm:017-polyhedral-retractive}
	Polyhedral sets are retractive.
\end{lemm}

\begin{prop}[Closedness of Vector Difference]\label{prop:017-closedness-vec-diff}
	Let $C_1,C_2$ be closed convex subsets of $\mathbb{R}^m$. Then $C_1-C_2$ is closed if any of the following holds:
	\begin{enumerate}[label=(\alph*)]
		\item $R_{C_1}\cap R_{C_2}=L_{C_1}\cap L_{C_2}$.
		\item $C_1$ is polyhedral with $R_{C_1}\cap R_{C_2}\subset L_{C_2}$.
		\item $C_1$ is compact.
		\item $C_1,C_2$ are both polyhedral.
	\end{enumerate}
\end{prop}

\begin{proof}
	For (c), use (a). For (d), use (e) of \Cref{prop:017-polyhedral-yoga}. For (a), use \Cref{coro:014-closed-sum}. For (b), use \Cref{coro:014-closed-affine}, \Cref{lemm:017-polyhedral-retractive} with $C_0=C_1\times \mathbb{R}^{m}$, $C=\mathbb{R}^m\times C_2$, $A:\mathbb{R}^{m}\times \mathbb{R}^m\to \mathbb{R}^m$ be $(x,y)\mapsto x-y$.
\end{proof}

\begin{prop}[Polyhedral Proper Separation]\label{prop:017-polyhedral-proper-sep}
	Let $C_1=C,C_2=P$ be nonempty convex subsets of $\mathbb{R}^m$ with $P$ polyhedral, then $\operatorname{ri}(C)\cap P=\emptyset$ iff there are $a\in \mathbb{S}^{m-1},b\in \mathbb{R}^m$ with $H_{a,b}\nsupset C\subset E_{a,b,-},P\subset E_{a,b,+}$.
\end{prop}

\paragraph{}The proof of this proposition is quite involved. One important fact that we will use is the following:

\begin{lemm}[Hyperplane Containment Criterion]\label{lemm:017-criterion-hyperplane-containment}
	Given nonempty convex set $C\subset \mathbb{R}^m$, $a\in \mathbb{S}^{m-1},b\in \mathbb{R}$, with $C\subset E_{a,b,-}$, then $C\subset H_{a,b}$ iff $\operatorname{ri}(C)\cap H_{a,b}\neq\emptyset$.
\end{lemm}

\begin{proof}
	It suffices to show "If". May assume $0\in \operatorname{ri}(C)\cap H_{a,b}$ (so $b=0$). Write $V=\operatorname{lin}(C),\dim(V)=l$. We show $V\subset H_{a,0}$: otherwise, take $v\in V\cap E_{a,0,+}$, then $\varepsilon v\subset \operatorname{ri}(C)$ for some $\varepsilon >0$, giving $\operatorname{ri}(C)\cap E_{a,0,+}\neq\emptyset$, a contradiction.
\end{proof}

\begin{proof}[Proof of \Cref{prop:017-polyhedral-proper-sep}]
	For "If", assuming $H_{a,b}\nsupset C\subset E_{a,b,-},P\subset E_{a,b,+}$, by \Cref{lemm:017-criterion-hyperplane-containment}, we have:
	\[
		\operatorname{ri}(C)\cap P\subset \operatorname{ri}(C)\cap E_{a,b,+}=\operatorname{ri}(C)\cap H_{a,b}=\emptyset
	\]
	"Only if" is the difficult part. For what follows, given two subsets $X,Y$ in $\mathbb{R}^m$, the subset $X_Y$ is defined to be $X\cap Y$. Write $V=\operatorname{aff}(C)$, which is also affine (hence polyhedral; later after translation, this subset will become a subspace).
	\begin{itemize}
		\item Consider separation between $P_V,C$, which are both subsets of $V$; both of them are polyhedral.
		      \begin{itemize}
			      \item If $P_V=\emptyset$, strict separation (for the polyhedral sets $P,V$) between $P,C$ is possible; see (d) of \Cref{prop:017-closedness-vec-diff} and (b) of \Cref{prop:015-hyperplane-sep}. Therefore, we may assum $P_V\neq\emptyset$.
			      \item Otherwise, we still have $\operatorname{ri}(P_V)\cap \operatorname{ri}(C)\subset P\cap \operatorname{ri}(C)=\emptyset$, so we can properly separate $P_V,C$; say the separating hyperplane and the separating halfspaces are $H,E_{\pm}$.
			      \item As $\operatorname{aff}(P_V)\subset\operatorname{aff}(C)$ and $H$ is affine, by properness, may write $H\nsupset C\subset  E_{-},P_V\subset E_+$.
		      \end{itemize}
		      In order to keep track of the various subsets that we will use, we have this diagram:
		      \[
			      \begin{tikzcd}[column sep=2ex,row sep=2ex]
				      H\rar{}&E_-\rar{}&\mathbb{R}^m&\lar{}E_+&\lar{}P\\
				      H_V\rar{}\uar{}&E_{-,V}\rar{}\uar{}&V\uar{}&\lar{}E_{+,V}\uar{}&\lar{}P_V\uar{}\\
				      H_C\rar{}\uar{}&C\rar{=}\uar{}&C\uar{}&\lar{}H_{C}\uar{}&\lar{}P_C\uar{}\\
			      \end{tikzcd}
		      \]
		      with the property that for each subgraph $A\to B$, $A\subset B$, and for each subgraph $A\leftarrow B\rightarrow C$, we have $B=A\cap C$. Note that the second and third rows are just the intersection of the first row with $V$ or $C$.
		\item Consider separation between $P,E_{-,V}$; note that $E_{-,V}\supset C$ is a polyhedral cone (after translation).
		      \begin{itemize}
			      \item We have $P\cap \operatorname{ri}(E_{-,V})=\emptyset$, using \Cref{lemm:017-criterion-hyperplane-containment} (on $E_{-,V}\subset E_{-}$).
			      \item If $P\cap E_{-,V}=\emptyset$, then strict separation between $P,C$ is possible.
			      \item Therefore, may assume $0\in P\cap E_{-,V}$ by translation; in this case, $H$, $V$ are linear.
		      \end{itemize}
		\item Consider separation between $K:=\operatorname{cone}(P)+H_V,E_{-,V}$; note that $K\supset P$ is a cone (in fact polyhedral).
		      \begin{itemize}
			      \item Firstly, $\operatorname{cone}(P)$ and hence (see (e) of \Cref{prop:017-polyhedral-yoga}) $K$ is a polyhedral cone: write
			            \[
				            P=\left(\bigcap_{x\in X}E_{x,0,-}\right)\cap\left(\bigcap_{y\in Y}E_{y,1,-}\right)
			            \]
			            for some finite subsets $X,Y$ in $\mathbb{R}^m\smallsetminus \{0\}$ (this is possible as $0\in P$), which gives:
			            \[
				            \operatorname{cone}(P)=\left(\bigcap_{x\in X}\operatorname{cone}(E_{x,0,-})\right)\cap\left(\bigcap_{y\in Y}\operatorname{cone}(E_{y,1,-})\right)
				            =\left(\bigcap_{x\in X}E_{x,0,-}\right)
			            \]
			      \item Next, we have $K\cap \operatorname{ri}(E_{-,V})=K\cap V\cap E_{-}^o=\emptyset$. If not, take an element in the intersection:
			            \[
				            \alpha x+y,\;\text{ where }\alpha \in \mathbb{R}_{++}, x\in P, y\in H_V\;\text{(note that }0\in P\text{)}
			            \]
			            Since $H_V=H\cap V\subset L_{E^o_-}\cap L_{V}=L_{\operatorname{ri}(E_{-,V})}$, we have $\alpha x\in \operatorname{cone}(P) \cap \operatorname{ri}(E_{-,V})$.\\
			            Since $\operatorname{ri}(E_{-,V})=E_-^o\cap V$, we have $x\in P\cap \operatorname{ri}(E_{-,V})$.
			      \item Express the polyhedral cone $K$ as $K=\bigcap_{z\in Z}E_{z,0,-}$ for some finite subset $Z\subset \mathbb{R}^m$.
			            \begin{itemize}
				            \item For fixed $z\in Z$, if $E_{z,0,-}\cap \operatorname{ri}(E_{-,V})\neq\emptyset$, then as $H_V\subset K\subset E_{z,0,-}$, $E_{-,V}\subset E_{z,0,-}$.
				            \item Therefore, in view of $K\cap \operatorname{ri}(E_{-,V})=\emptyset$, $E_{z,0,-}\cap \operatorname{ri}(E_{-,V})=\emptyset$ for some $z\in Z$; in other words, we have $\operatorname{ri}(E_{-,V})\subset E_{-,V}\subset E_{z,0,+}$
				            \item Note that for this $E_{z,0,-}$, we would have $P\subset E_{z,0,-},C\subset E_{z,0,+}$. To see $H_{z,0}\nsupset C$, say if $H_{z,0}\supset C$, then $H_{z,0}\supset V\supset E_{-,V}$, giving $E_{z,0,-}\cap \operatorname{ri}(E_{-,V})=\operatorname{ri}(E_{-,V})=\emptyset$, a contradiction. In short, $H_{z,0}$ properly separates $C,P$.\qedhere{}
			            \end{itemize}
		      \end{itemize}
	\end{itemize}
\end{proof}
