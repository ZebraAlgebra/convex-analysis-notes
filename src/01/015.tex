\section{Hyperplane Separation}
\label{sect:015}
\paragraph{}For this section, let us work within some $\mathbb{R}^m$.

\begin{defn}[Hyperplanes, Halfspaces]\label{defn:015-hyperplanes-halfspaces}
	Given $a\in \mathbb{R}^{m}\smallsetminus \{0\},b\in \mathbb{R}$, we define:
	\[
		H_{a,b}=\{x\in \mathbb{R}^m:a^Tx=b\},\;
		E_{a,b,+}=\{x\in \mathbb{R}^m:a^Tx-b\geq 0\},\;
		E_{a,b,-}=\{x\in \mathbb{R}^m:a^Tx-b\leq 0\},\;
	\]
	called the hyperplane, closed halfspaces corresponding to $(a,b)$. The open halfspaces are denoted as:
	\[
		E_{a,b,+}^o=\{x\in \mathbb{R}^m:a^Tx-b> 0\},\;
		E_{a,b,-}^o=\{x\in \mathbb{R}^m:a^Tx-b< 0\},\;
	\]
\end{defn}

\begin{rmrk}
	One would think of uniqueness of such representations. We have the equalities:
	\[
		E_{a,b,+}=E_{-a,-b,-},\;E^o_{a,b,+}=E^o_{-a,-b,-}
	\]
	\[
		H_{a,b}=\operatorname{bdy}(E_{a,b,\pm})=\operatorname{bdy}(E_{a,b,\pm}^o),\;
		E_{a,b,\pm}=\operatorname{cl}(E_{a,b,\pm}^o),\;
		E^o_{a,b,\pm}=\operatorname{int}(E_{a,b,\pm})
	\]
	If we restrict $a\in \mathbb{S}^{m-1}$, then $E_{a,b,+}$ (resp. $E_{a,b,-},E_{a,b,+}^o,E_{a,b,-}^o$) is unique, while $H_{a,b}$ is unique up to $\pm 1$.
\end{rmrk}

\paragraph{}In the case where $m\geq 2$, we can consider whether halfspaces are "vertical"; such notion is important in the context of min-common / max-crossing geometric duality of convex sets.

\begin{defn}[Vertical, Upward]\label{defn:015-verticality}
	Suppose $m\geq 2$, write $e_m=(0,0,\dotsc,0,1)\in \mathbb{R}^m$. We say that:
	\begin{enumerate}[label=(\alph*)]
		\item A hyperplane $H$ (resp. halfspace $E$) is vertical if $e_m\in L_H$ (resp. $e_m\in L_E$).
		\item A non-vertical halfspace $E$ is upwards if $e_m\in R_E$; otherwise it is called downwards.
	\end{enumerate}
\end{defn}

\paragraph{}One can verify that $H_{a,b},E_{a,b,\pm},E^o_{a,b,\pm}$ is vertical iff $e_m^Ta=0$, and that $E_{a,b,+},E_{a,b,+}^o$ (resp. $E_{a,b,-},E^o_{a,b,-}$) is non-vertical and upward iff $e_m^Ta>0$ (resp. $e_m^Ta<0$).

\paragraph{}One use of hyperplanes and halfspaces are results on the seperation of convex sets. We define what it means to separate two sets as follows:
\begin{defn}[Separation]\label{defn:015-hyperplane-sep}
	Let $C_1,C_2$ be nonempty subsets of $\mathbb{R}^m$, $(a,b)\in \mathbb{S}^{m-1}\times \mathbb{R}$. We say:
	\begin{itemize}
		\item $C_1,C_2$ is separated by $H_{a,b}$ if each $E_{a,b,+},E_{a,b,-}$ contains at least one of $C_1,C_2$.
		\item $C_1,C_2$ is properly separated by $H_{a,b}$ if it separates $C_1,C_2$ and that $H_{a,b}$ doesn't contain both $C_1,C_2$.
		\item $C_1,C_2$ is strictly separated by $H_{a,b}$ if each $E^o_{a,b,+},E^o_{a,b,-}$ contains at least one of $C_1,C_2$.
	\end{itemize}
\end{defn}

\paragraph{}Now we present the separation theorems.

\begin{prop}[Hyperplane Separation]\label{prop:015-hyperplane-sep}
	Let $C_1,C_2$ be non-empty convex subsets of $\mathbb{R}^m$.
	\begin{enumerate}[label=(\alph*)]
		\item (Basic Separation) $C_1,C_2$ can be separated by a hyperplane if $C_1\cap C_2=\emptyset$.
		\item (Strict Separation) $C_1,C_2$ can be strictly separated by a hyperplane if $C_1\cap C_2=\emptyset$ and $C_2-C_1$ closed.
		\item (Proper Separation) $C_1,C_2$ can be properly separated by a hyperplane iff $\operatorname{ri}(C_1)\cap \operatorname{ri}(C_2)=\emptyset$.
	\end{enumerate}
	In the case where $C_2=\{x_0\}$ for some $x_0\in \mathbb{R}^m$:
	\begin{enumerate}[label=(\alph*)]
		\setcounter{enumi}{3}
		\item (Basic Separation) $C_1,C_2$ can be separated by a hyperplane if $x_0\notin \operatorname{int}(C_1)$.
		\item (Proper Separation) $C_1,C_2$ can be properly separated by a hyperplane iff $x_0\notin \operatorname{ri}(C_1)$.
	\end{enumerate}
	In the case where $m\geq 2$ and that $C_1$ contains no vertical lines (that is, $(C_{1})_{x,:}\neq \mathbb{R}$ for any $x\in \mathbb{R}^{m-1}$):
	\begin{enumerate}[label=(\alph*)]
		\setcounter{enumi}{5}
		\item (Existence of Ambient Nonvertical Halfspace) $C_1$ is contained in some $E_{a,b,+}$ with $H_{a,b}$ nonvertical.
		\item (Nonvertical, Strict Separation) Suppose $C_2=\{x_0\}$ for some $x_0\in \mathbb{R}^m$, then $C_1,C_2$ can be strictly separated by a nonvertical hyperplane if $x_0\notin \operatorname{cl}(C_1)$.
	\end{enumerate}
\end{prop}

\begin{proof}[Proof of (d)]
	We use \Cref{lemm:014-projection} on $\operatorname{cl}(C_1)$. Want to show that there is some $a\in \mathbb{S}^{m-1}$ with:
	\[
		a^Tx_0\leq \underset{x\in C_1}{\operatorname{inf}}a^Tx
	\]
	To this end, take $\mathbb{R}^m\smallsetminus \operatorname{cl}(C_1)\supset \{x_i\}_i\to x_0$ (this sequence exists in the case where $x_0\notin \operatorname{aff}(C_1)$; otherwise, use (a) of \Cref{prop:012-yoga-ri-cl}). Let $\operatorname{cl}(C_1)\supset\{y_i\}_i$ with $y_i$ being the projection of $x_i$ on $\operatorname{cl}(C_1)$, and let $\mathbb{S}^{m-1}\supset \{a_i\}_i$ with $a_i$ being $\nu(y_i-x_i)$. By rechoosing a subsequence, may assume $\{a_i\}_i\to a$ for some $a\in \mathbb{S}^{m-1}$. Now since $y_i$ is projection of $x_i$ to $C$, we have $a_i^Tx_i\leq a_i^Tx$ for each $i$ and each $x\in C$; this would suffice.
\end{proof}

\begin{proof}[Proof of (a) and (b)]
	This is by applying (d) to $C_2-C_1$ and the point $\underset{x\in C_2-C_1}{\operatorname{inf}}\|x\|$.
\end{proof}
\begin{proof}[Proof of (e)]
	We show "if". Assume $x_0\notin \operatorname{ri}(C_1)$. If $x_0\notin \operatorname{aff}(C_1)$, apply (b) to the sets $\operatorname{aff}(C_1),\{x_0\}$. Otherwise, one may reduce to the case where $\dim(C_1)=m$, in which case $\operatorname{ri}(C_1)=\operatorname{int}(C_1)$, so (a) gives a basic hyperplane separation; this would be proper as $C_1$ has non-empty interior. For "only if", assume $x_0\in \operatorname{ri}(C_1)$. May reduce to the case $\dim(C_1)=m$, then it suffices to show that $0,B(0;1)$ cannot be separated by a hyperplane; the proof of this is left to the readers.
\end{proof}
\begin{proof}[Proof of (c)]
	One applies (d) of \Cref{prop:012-yoga-ri-cl} and (e) to $C_2-C_1,\{0\}$.
\end{proof}
\begin{proof}[Proof of (f)]
	We use recession cones. Note that:
	\[
		R_{\operatorname{ri}(C_1)}=
		R_{\operatorname{cl}(C_1)}=
		\bigcap_{\substack{a,b\in \mathbb{S}^{m-1}\times \mathbb{R}:\\E_{a,b,+}\supset C_1}}R_{E_{a,b,+}}
	\]
	where we used (f), (g) of \Cref{prop:013-yoga-recession} and \Cref{coro:015-halfspaces-intersection}. If all $E_{a,b,+}$ containing $C_1$ has vertical line, then $\{0\}\times\mathbb{R}\subset R_{\operatorname{ri}(C_1)}$, in which case $\operatorname{ri}(C_1)$ and hence $C_1$ will have a vertical line.
\end{proof}
\begin{proof}[Proof of (g)]
	By (b), (f), may take $(a,b),(a',b')$ in $\mathbb{S}^{m-1}\times \mathbb{R}$ with $C_1\subset E^o_{a,b,+}\cap E_{a',b',+},C_2\subset E^o_{a,b,-}$, where $a'\notin \mathbb{S}^{m-2}\times \{0\}$. May assume $a\in \mathbb{S}^{m-2}\times \{0\}$, or we are done. It suffices to show for some $\varepsilon >0$ that (i) $C_1\subset E^o_{a_\varepsilon,b_\varepsilon,+}$ (ii) $C_2\subset E^o_{a_\varepsilon,b_\varepsilon,-}$, where $a_\varepsilon:=a+\varepsilon a',b_\varepsilon:=b+\varepsilon b'$. (i) holds for all $\varepsilon >0$; (ii) is the same as $a_\varepsilon^T x_0>b_\varepsilon $, but since $a_\varepsilon\to a,b_\varepsilon \to 0$ as $\varepsilon \to 0^+$ and that this holds for $\varepsilon =0$, such $\varepsilon >0$ exists.
\end{proof}

\begin{coro}[Closure of Convex Hull as Intersection of Halfspaces]\label{coro:015-halfspaces-intersection}
	Suppose $X\subset \mathbb{R}^m$, then:
	\[
		\operatorname{cl}(\operatorname{conv}(X))=\bigcap_{\substack{a,b\in \mathbb{S}^{m-1}\times \mathbb{R}:\\E_{a,b,+}\supset X}}E_{a,b,+}
	\]
\end{coro}
\begin{proof}
	Evidently "$\subset$" holds. For "$\supset$", write $C=\operatorname{cl}(\operatorname{conv}(X))$, given $x_0\notin C$, apply (b) to $C,\{x_0\}$ to find some $E_{a,b,+}$ with $x_0\notin E_{a,b,+}$.
\end{proof}

\begin{rmrk}[Supporting Hyperplane]
	In the setting of (d) of \Cref{prop:015-hyperplane-sep}, one can take the hyperplane to pass $x_0$; in such case, this hyperplane is called "supporting".
\end{rmrk}
