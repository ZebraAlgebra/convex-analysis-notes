\section{Relative Interiors and Closures}
\label{sect:012}

\paragraph{}A nonempty convex set might have empty interior, but if one restricts attention to the affine hull, then it is a fact that in this subspace its interior is not empty. This subset is called the relative interior. Another nice fact is that convex functions (to be defined in the next chapter) is continuous on the relative interior. We will define relative interiors and give an alternative characterization of such sets (see (b), (c) of \Cref{prop:012-basic-ri}). After that, we formulate a list of yogas for the closure and relative interior operations for convex sets. For this section, fix nonempty convex sets $C,C'$ of $\mathbb{R}^m$, write $Y=\operatorname{aff}(C)$ with $\dim(Y)=d\leq m$.

\begin{defn}[Relative Interior]
	\label{defn:012-relint}
	Define $\operatorname{ri}(C)\subset Y$ as the interior of $C$ with respect to the subspace topology of $Y$. Explicitly, if $T:\mathbb{R}^d\to \mathbb{R}^m$ is an affine surjection onto $Y$, then $\operatorname{ri}(C)=T (\operatorname{int}(T^{-1}(C)))$.
\end{defn}

\begin{prop}[Basic Facts about Relative Interiors]We have the following basic facts about $\operatorname{ri}(C)$:
	\label{prop:012-basic-ri}
	\begin{enumerate}[label=(\alph*)]
		\item (Relative Interior retains Dimension) $\operatorname{aff}(\operatorname{ri}(C))=Y$. Hence $\operatorname{ri}(C)\neq\emptyset$.
		\item (Line Segment Principle) $I(\operatorname{ri}(C),\operatorname{cl}(C);[0,1))=\operatorname{ri}(C)$.
		\item (Relative Interior is Convex) $\operatorname{ri}(C)$ is convex with $\operatorname{dim}(\operatorname{ri}(C))=d$.
		\item (Prolongation Lemma) Given $x\in C$, then $x\in \operatorname{ri}(C)$ iff for all $I(y,x;[0,1])\subset C$, there exists $\varepsilon>0$ with $I(y,x;[0,1+\varepsilon ])\subset C$.
	\end{enumerate}
\end{prop}

\begin{proof}
	For (a), consider $(e_i)_{i=1}^d$ and $e_0=O$. Take $\{x_i\}_{i=0}^d\subset C$ with $\operatorname{aff}(\{x_i\}_{i=0}^d)=Y$, and define an affine map $T:\mathbb{R}^d\to \mathbb{R}^m$ by $T(e_i)=x_i$ for $i=0,\dotsc,d$; this is a surjection to $Y$. We have:
	\[
		\operatorname{aff}(\operatorname{ri}(C)) \supset T\left(\operatorname{aff}\left(\operatorname{int}\left(\operatorname{conv}\left(\{e_i\}_{i=0}^d\right)\right)\right)\right)=T(\mathbb{R}^d)=Y
	\]
	For (b), take $x\in \operatorname{ri}(C),y\in \operatorname{cl}(C), \alpha \in[0, 1)$. Want $z:=I(x, y; \alpha )\in\operatorname{ri}(C)$. May assume $d=m$ so ri and int coincides. Take $C\supset\{y_i\}_i\to y$, write $z_i=I(x,y_i;\alpha )$. Suppose $B(x,\varepsilon )\subset C$ for some $\varepsilon >0$, then:
	\[
		B(z_i;(1-\alpha )\varepsilon )\subset I(B(x;\varepsilon ),y_i;\alpha)\subset C
	\]
	By taking $i$ large enough with $B(z;(1-\alpha )\varepsilon /2)\subset B(z_i;(1-\alpha )\varepsilon )$, we get what we want: $z\in \operatorname{int}(C)$.\\
	For (c), (b) gives $I(\operatorname{ri}(C),\operatorname{ri}(C),[0, 1))\subset I(\operatorname{ri}(C),\operatorname{cl}(C);[0,1))=\operatorname{ri}(C)$; (a) shows $\operatorname{dim}(\operatorname{ri}(C))=d$.\\
	For (d), "only if" is easy after reduction to the case $d=m$. For "if", given $y\in \operatorname{ri}(C)$, $\varepsilon>0$ with $z=I(y,x;1+\varepsilon )\in C$, we have $x\in I(y,z;[0,1))\subset \operatorname{ri}(C)$ by (b).
\end{proof}

\begin{prop}[Yoga of ri and cl]\label{prop:012-yoga-ri-cl}.
	\begin{enumerate}[label=(\alph*)]
		\item (ri and cl) We have the following basic relation between ri and cl:
		      \begin{enumerate}[label=(\roman*)]
			      \item $\operatorname{cl}(\operatorname{ri}(C))=\operatorname{cl}(C)=\operatorname{cl}(\operatorname{cl}(C))$
			      \item $\operatorname{ri}(\operatorname{ri}(C))=\operatorname{ri}(C)=\operatorname{ri}(\operatorname{cl}(C))$
			      \item If $C'$ is another convex set, then $\operatorname{cl}(C)=\operatorname{cl}(C')$ iff $\operatorname{ri}(C)=\operatorname{ri}(C')$ iff $\operatorname{ri}(C)\subset C'\subset\operatorname{cl}(C)$.
		      \end{enumerate}
		\item (Image Rule) Given affine $A:\mathbb{R}^l\to \mathbb{R}^m$, then:
		      \begin{enumerate}[label=(\roman*)]
			      \item $A\operatorname{ri}(C)=\operatorname{ri}(AC)$
			      \item $A\operatorname{cl}(C)\subset \operatorname{cl}(AC)$. "$\supset$" holds if $C$ is bounded.
		      \end{enumerate}
		\item (Product Rule) Let $C'$ be another nonempty convex set in some $\mathbb{R}^n$. Then:
		      \begin{enumerate}[label=(\roman*)]
			      \item $\operatorname{ri}(C)\times\operatorname{ri}(C')=\operatorname{ri}(C\times C')$
			      \item $\operatorname{cl}(C)\times\operatorname{cl}(C')=\operatorname{cl}(C\times C')$
		      \end{enumerate}
		\item (Linear Combination Rule) Let $C'$ be another nonempty convex set in $\mathbb{R}^m$. Then:
		      \begin{enumerate}[label=(\roman*)]
			      \item $\operatorname{ri}(C)+\operatorname{ri}(C')=\operatorname{ri}(C+C')$
			      \item $\operatorname{cl}(C)+\operatorname{cl}(C')\subset\operatorname{cl}(C+C')$. "$\supset$" holds if $C$ or $C'$ is bounded.
		      \end{enumerate}
		      These statements can be generalized to linear combinations of convex sets.
		\item (Finite Intersection Rule) Let $C'$ be another nonempty convex set in $\mathbb{R}^m$. Then:
		      \begin{enumerate}[label=(\roman*)]
			      \item $\operatorname{ri}(C)\cap\operatorname{ri}(C')\subset\operatorname{ri}(C\cap C')$.
			      \item $\operatorname{cl}(C)\cap\operatorname{cl}(C')\supset\operatorname{cl}(C\cap C')$.
			      \item "=" holds in (i) and (ii) if $\operatorname{ri}(C)\cap\operatorname{ri}(C')\neq\emptyset$.
		      \end{enumerate}
		\item (Inverse Image Rule) Let $A:\mathbb{R}^l\to \mathbb{R}^m$ be an affine map. Suppose $A^{-1}\operatorname{ri}(C)\neq \emptyset$.
		      \begin{enumerate}[label=(\roman*)]
			      \item $A^{-1}\operatorname{ri}(C)=\operatorname{ri}(A^{-1}C)$
			      \item $A^{-1}\operatorname{cl}(C)=\operatorname{cl}(A^{-1}C)$
		      \end{enumerate}
		\item (Slice and Fiber Rule) Given $l< m$, consider set of "slice supports" $D=\{x\in \mathbb{R}^l:C_{x,:}\neq\emptyset\}$.\\
		      Then we have:
		      $\operatorname{ri}(C)_{x,:}=
			      \begin{cases}
				      \emptyset                  & \text{ if }x\notin \operatorname{ri}(D) \\
				      \operatorname{ri}(C_{x,:}) & \text{ if }x\in \operatorname{ri}(D)
			      \end{cases}$. An analogous statement holds for fibers.
	\end{enumerate}
\end{prop}

\begin{proof}The proof of the items are as follows.
	\begin{enumerate}[label=(\alph*)]
		\item For (i) and (ii), take $x\in \operatorname{ri}(C)$ (exists by (a) of \Cref{prop:012-basic-ri}). We have:
		      \begin{itemize}
			      \item $\operatorname{cl}(\operatorname{ri}(C))\subset\operatorname{cl}(C)=\operatorname{cl}(\operatorname{cl}(C))$: Direct from the definition of closures.
			      \item $\operatorname{ri}(\operatorname{ri}(C))=\operatorname{ri}(C)\subset\operatorname{ri}(\operatorname{cl}(C))$: By (a) of \Cref{prop:012-basic-ri}, $\operatorname{ri}(C),C,\operatorname{cl}(C)$ all have affine hull equal to $Y$, so these containments holds by definition of relative interiors (taking int in $Y$).
			      \item $\operatorname{cl}(C)\subset \operatorname{cl}(\operatorname{ri}(C))$: For $y\in \operatorname{cl}(C)$, $y\in \operatorname{cl}(I(x,y,[0, 1)))\subset \operatorname{cl}(\operatorname{ri}(C))$ ((b) of \Cref{prop:012-basic-ri}).
			      \item $\operatorname{ri}(\operatorname{cl}(C))\subset \operatorname{ri}(C)$: For $y\in \operatorname{ri}(\operatorname{cl}(C))$, take $\varepsilon>0$ with $I(x,y;[0,1+\varepsilon ])\subset \operatorname{cl}(C)$ ((d) of \Cref{prop:012-basic-ri}); let $I(x,y;1+\varepsilon)=z$, by (b) of \Cref{prop:012-basic-ri}, $y\in I(x,z;[0, 1))\subset \operatorname{ri}(C)$.
		      \end{itemize}
		      These gives (i) and (ii). Item (iii) is straight-forward from (i) and (ii).
		\item
		      \begin{itemize}
			      \item[(ii)] For "$\subset$", note that for $C\supset \{x_i\}_i\to x\in \operatorname{cl}(C)$, we have $AC\supset \{Ax_i\}_{i}\to Ax\in A\operatorname{cl}(C)$. Conversely, when $C$ is bounded, given $AC\supset \{Ax_i\}_i\to z\in \operatorname{cl}(AC)$, boundedness of $C$ gives a subsequence of $x_i$ converging to some $x\in \operatorname{cl}(C)$, making $Ax=z\in A\operatorname{cl}(C)$.
			      \item[(i)] For "$\supset$", it suffices to show $AC\subset \operatorname{cl}(A \operatorname{ri}(C))$ in view of (a). By (ii):
			            \[
				            AC\subset A \operatorname{cl}(C) = A\operatorname{cl}(\operatorname{ri}(C)) \subset \operatorname{cl}(A \operatorname{ri}(C))
			            \]
			            For "$\subset$", use (d) of \Cref{prop:012-basic-ri}: for $x\in C$, $y\in \operatorname{ri}(C)$, take $\varepsilon >0$ with $I(x,y;[0,1+\varepsilon ])\subset C$:
			            \[
				            I(Ax,Ay;[0,1+\varepsilon ])=AI(x,y;[0,1+\varepsilon ])\subset AC
			            \]
			            showing that $Ay\in \operatorname{ri}(AC)$.
		      \end{itemize}
		\item For (i), use (d) of \Cref{prop:012-basic-ri}. Statement (ii) is direct by a sequence argument.
		\item By (b) and (c) (applied to summation map $\mathbb{R}^{m}\times\mathbb{R}^{m}\to \mathbb{R}^m$), we only need to show "$\supset$" in (ii) under boundedness condition. Given $\{x_i\}_i\subset C,\{y_i\}_i\subset C'$ with $\{x_i+y_i\}_i\to z\in \operatorname{cl}(C+C')$, then both $\{x_i\}_i,\{y_i\}_i$ are bounded, so $\{(x_i,y_i)\}_i$ is bounded, hence has a subsequence converging to some $(x,y)\in \operatorname{cl}(C\times C')$; we have $z=x+y\in \operatorname{cl}(C)+\operatorname{cl}(C')$.
		\item \begin{enumerate}[label=(\roman*)]
			      \item We use (d) of \Cref{prop:012-basic-ri}. Take $x\in C\cap C'$, $y\in \operatorname{ri}(C)\cap \operatorname{ri}(C')$, choose $\varepsilon,\varepsilon'>0$ with
			            \[
				            I(x,y;[0,1+\varepsilon ])\subset C,\;
				            I(x,y;[0,1+\varepsilon' ])\subset C'
			            \]
			            then by letting $\varepsilon''=\min(\varepsilon ,\varepsilon')$, we have $I(x,y;[0,1+\varepsilon'' ])\subset C\cap C'$.
			      \item It suffices to show $\operatorname{cl}(C\cap C')\subset \operatorname{cl}(C)$, which is evident.
			      \item Take $x\in \operatorname{ri}(C) \cap\operatorname{ri}(C')$. Let $y\in \operatorname{cl}(C)\cap \operatorname{cl}(C')$. By (b) of \Cref{prop:012-basic-ri}, we have:
			            \[
				            y\in I(x, y;[0, 1])=\operatorname{cl}(I(x, y;[0, 1)))\subset\operatorname{cl}(\operatorname{ri}(C)\cap \operatorname{ri}(C'))
			            \]
			            giving the converse of (ii): $\operatorname{cl}(C)\cap \operatorname{cl}(C')\subset \operatorname{cl}(\operatorname{ri}(C)\cap \operatorname{ri}(C'))\subset \operatorname{cl}(C\cap C')$. To show the converse of (i), note that by this identity, $\operatorname{ri}(C)\cap \operatorname{ri}(C')$ and $C\cap C'$ have the same cl, so they have the same ri (by (a)), while $\operatorname{ri}(\operatorname{ri}(C)\cap \operatorname{ri}(C'))\subset\operatorname{ri}(C)\cap \operatorname{ri}(C)$.
		      \end{enumerate}
		\item Consider the following factorization of maps:
		      \[
			      \begin{tikzcd}
				      \mathbb{R}^l \arrow[rr,"A"]\arrow[dr,swap,"g"] &&\mathbb{R}^{m}\\
				      &\mathbb{R}^{l+m}\arrow[ru,swap,"\pi"]&
			      \end{tikzcd}
		      \]
		      where $g$ is the map $x\mapsto (x,Ax)$, $\pi$ is the map $(x, y)\mapsto y$. Write $\Gamma_A$ as the image of $g$ - this is sometimes referred to as the graph of $A$. We have:
		      \[
			      A^{-1}C=\pi (\Gamma_A\cap \mathbb{R}^l\times C)
		      \]
		      We will use this description of inverse images as well as (b), (c), (e) to proceed.
		      \begin{enumerate}[label=(\roman*)]
			      \item To compute $\operatorname{ri}(A^{-1}C)$, note first that by (b), (c), we have:
			            \[
				            \operatorname{ri}(\Gamma_A)=\Gamma_A,\; \operatorname{ri}(\mathbb{R}^m\times C)= \mathbb{R}^m \times \operatorname{ri}(C)
			            \]
			            so by (b), (e), and the condition $A^{-1}\operatorname{ri}(C)\neq\emptyset$:
			            \begin{center}
				            \begin{tabular}{llll}
					            $\operatorname{ri}(A^{-1}C)$ & $=\operatorname{ri}(\pi (\Gamma_A\cap \mathbb{R}^m\times C))$                      & $=\pi \operatorname{ri}(\Gamma_A\cap \mathbb{R}^m\times C)$                                     \\
					                                         & $=\pi (\operatorname{ri}( \Gamma_A )\cap \operatorname{ri}(\mathbb{R}^m\times C))$ & $=\pi ( \Gamma_A \cap \mathbb{R}^m\times \operatorname{ri}(C))$ & $=A^{-1}\operatorname{ri}(C)$
				            \end{tabular}
			            \end{center}
			      \item To compute $\operatorname{cl}(A^{-1}C)$, we have:
			            \[
				            \operatorname{cl}(\Gamma_A)=\Gamma_A,\; \operatorname{cl}(\mathbb{R}^m\times C)= \mathbb{R}^m \times \operatorname{cl}(C)
			            \]
			            where the first "$=$" is by the fact that $A$ is affine. In a similar way, we get:
			            \begin{center}
				            \begin{tabular}{llll}
					            $\operatorname{cl}(A^{-1}C)$ & $=\operatorname{cl}(\pi (\Gamma_A\cap \mathbb{R}^m\times C))$                      & $\supset\pi \operatorname{cl}(\Gamma_A\cap \mathbb{R}^m\times C)$                                 \\
					                                         & $=\pi (\operatorname{cl}( \Gamma_A )\cap \operatorname{cl}(\mathbb{R}^m\times C))$ & $=\pi ( \Gamma_A \cap \mathbb{R}^m\times \operatorname{cl}(C))$   & $=A^{-1}\operatorname{cl}(C)$
				            \end{tabular}
			            \end{center}
			            Conversely, given $A^{-1}C\supset (x_k)_k\to x\in \operatorname{cl}(A^{-1}C)$, then $C\supset (Ax_k)_k$ converges, to $Ax\in \operatorname{cl}(C)$.
		      \end{enumerate}
		\item Again, let $\pi:\mathbb{R}^{m}\to \mathbb{R}^l$ be projection $(x,y)\mapsto x$. We then have:
		      \[
			      \operatorname{ri}(D)=\operatorname{ri}(\pi C)=\pi \operatorname{ri}(C)
		      \]
		      Therefore, if $x\notin \operatorname{ri}(D)$, then we get $C_{x,:}=\emptyset$. Otherwise, we have:
		      \begin{align*}
			      \{x\}\times \operatorname{ri}(C)_{x,:} & = \{x\}\times \mathbb{R}^m\cap \operatorname{ri}(C) = \operatorname{ri}(\{x\}\times \mathbb{R}^m)\cap \operatorname{ri}(C) \\
			                                             & = \operatorname{ri}(\{x\}\times \mathbb{R}^m\cap C) = \operatorname{ri}(\{x\}\times C_{x,:})
		      \end{align*}
		      where we used (c) for the second "=", (e) for the third "=" along with the proven fact $\operatorname{ri}(D)=\pi \operatorname{ri}(C)$ and the assumption $x\in \operatorname{ri}(D)$. \qedhere{}
	\end{enumerate}
\end{proof}
