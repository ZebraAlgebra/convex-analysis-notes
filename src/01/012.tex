\section{Relative Interior and Closure}
\label{sect:012}

\paragraph{}A nonempty convex set might have empty interior, but if one restricts attention to the affine hull, then it is a fact that in this subspace its interior is not empty. This subset is called the relative interior. Another nice fact is that convex functions (to be defined in the next chapter) is continuous on the relative interior. We will define relative interiors and give an alternative characterization of such sets (see (b), (c) of \Cref{prop:012-basic-ri}). After that, we formulate a list of yogas for the closure and relative interior operations for convex sets. For this section, fix a nonempty convex sets $C,C'$ of $\mathbb{R}^m$, write $Y=\operatorname{aff}(C)$ with $\dim(Y)=d\leq m$.

\begin{defn}[Relative Interior]
	\label{defn:012-relint}
	Define $\operatorname{ri}(C)\subset Y$ as the interior of $C$ with respect to the subspace topology of $Y$. Explicitly, if $T:\mathbb{R}^d\to \mathbb{R}^m$ is an affine surjection onto $Y$, then $\operatorname{ri}(C)=T (\operatorname{int}(T^{-1}(C)))$.
\end{defn}

\begin{prop}[Basic Facts about Relative Interiors]We have the following basic facts about $\operatorname{ri}(C)$:
	\label{prop:012-basic-ri}
	\begin{enumerate}[label=(\alph*)]
		\item (Relative Interior retains Dimension) $\operatorname{aff}(\operatorname{ri}(C))=Y$. Hence $\operatorname{ri}(C)\neq\emptyset$.
		\item (Line Segment Principle) $I(\operatorname{ri}(C),\operatorname{cl}(C);[0,1))=\operatorname{ri}(C)$.
		\item (Relative Interior is Convex) $\operatorname{ri}(C)$ is convex with $\operatorname{dim}(\operatorname{ri}(C))=d$.
		\item (Prolongation Lemma) Given $x\in C$, then $x\in \operatorname{ri}(C)$ iff for all $I(y,x;[0,1])\subset C$, there exists $\varepsilon>0$ with $I(y,x;[0,1+\varepsilon ])\subset C$.
	\end{enumerate}
\end{prop}

\begin{proof}
	For (a), let $(e_i)_{i=1}^d$ be the standard basis of $\mathbb{R}^d$ and write $e_0=0$. Take $\{x_i\}_{i=0}^d\subset C$ with $\operatorname{aff}(\{x_i\}_{i=0}^d)=Y$, define affine map $T:\mathbb{R}^d\to \mathbb{R}^m$ by $T(e_i)=x_i$ for $i=0,\dotsc,d$; this is a surjection to $Y$. We have:
	\[
		\operatorname{aff}(\operatorname{ri}(C)) \supset T\left(\operatorname{aff}\left(\operatorname{int}\left(\operatorname{conv}\left(\{e_i\}_{i=0}^d\right)\right)\right)\right)=T(\mathbb{R}^d)=Y
	\]
	For (b), take $x\in \operatorname{ri}(C),y\in \operatorname{cl}(C), \alpha \in[0, 1)$. Want $z:=I(x, y; \alpha )\in\operatorname{ri}(C)$. May assume $d=m$ so ri and int coincides. Take $C\supset\{y_i\}_i\to y$, write $z_i=I(x,y_i;\alpha )$. Suppose $B(x,\varepsilon )\subset C$ for some $\varepsilon >0$, then:
	\[
		B(z_i;(1-\alpha )\varepsilon )\subset I(B(x;\varepsilon ),y_i;\alpha)\subset C
	\]
	By taking $i$ large enough with $B(z;(1-\alpha )\varepsilon /2)\subset B(z_i;(1-\alpha )\varepsilon )$, we get what we want.\\
	For (c), (b) gives $I(\operatorname{ri}(C),\operatorname{ri}(C),[0, 1))\subset I(\operatorname{ri}(C),\operatorname{cl}(C);[0,1))=\operatorname{ri}(C)$; (a) shows $\operatorname{dim}(\operatorname{ri}(C))=d$.\\
	For (d), "only if" is easy after reduction to the case $d=m$. For "if", given $y\in \operatorname{ri}(C)$, $\varepsilon>0$ with $z=I(y,x;1+\varepsilon )\in C$, we have $x\in I(y,z;[0,1))\subset \operatorname{ri}(C)$ by (b).
\end{proof}

\begin{prop}[Yoga of ri and cl]\label{prop:012-yoga-ri-cl}.
	\begin{enumerate}[label=(\alph*)]
		\item (ri and cl) We have the following basic relation between ri and cl:
		      \begin{enumerate}[label=(\roman*)]
			      \item $\operatorname{cl}(\operatorname{ri}(C))=\operatorname{cl}(C)=\operatorname{cl}(\operatorname{cl}(C))$
			      \item $\operatorname{ri}(\operatorname{ri}(C))=\operatorname{ri}(C)=\operatorname{ri}(\operatorname{cl}(C))$
			      \item If $C'$ is another convex set, then TFAE:
			            \begin{enumerate}[label=(\arabic*)]
				            \item $\operatorname{cl}(C)=\operatorname{cl}(C')$
				            \item $\operatorname{ri}(C)=\operatorname{ri}(C')$
				            \item $\operatorname{ri}(C)\subset C'\subset\operatorname{cl}(C)$
			            \end{enumerate}
		      \end{enumerate}
		\item (Image Rule) Given affine $A:\mathbb{R}^l\to \mathbb{R}^m$, then:
		      \begin{enumerate}[label=(\roman*)]
			      \item $A\operatorname{ri}(C)=\operatorname{ri}(AC)$
			      \item $A\operatorname{cl}(C)\subset \operatorname{cl}(AC)$, with "$\supset$" holds if $C$ is bounded.
		      \end{enumerate}
		\item (Product Rule)
		\item (Sum Rule)
		\item (Finite Intersection Rule)
		\item (Inverse Image Rule)
		\item (Slice and Fiber Rule)
	\end{enumerate}
\end{prop}

\begin{proof}The proof of the items are as follows.
	\begin{enumerate}[label=(\alph*)]
		\item For (i) and (ii), take $x\in \operatorname{ri}(C)$ (exists by (a) of \Cref{prop:012-basic-ri}). We have:
		      \begin{itemize}
			      \item $\operatorname{cl}(\operatorname{ri}(C))\subset\operatorname{cl}(C)=\operatorname{cl}(\operatorname{cl}(C))$: Direct from the definition of closures.
			      \item $\operatorname{ri}(\operatorname{ri}(C))=\operatorname{ri}(C)\subset\operatorname{ri}(\operatorname{cl}(C))$: By (a) of \Cref{prop:012-basic-ri}, $\operatorname{ri}(C),C,\operatorname{cl}(C)$ all have affine hull equal to $Y$, so these containments holds by definition of relative interiors (taking int in $Y$).
			      \item $\operatorname{cl}(C)\subset \operatorname{cl}(\operatorname{ri}(C))$: For $y\in \operatorname{cl}(C)$, $y\in \operatorname{cl}(I(x,y,[0, 1)))\subset \operatorname{cl}(\operatorname{ri}(C))$ ((b) of \Cref{prop:012-basic-ri}).
			      \item $\operatorname{ri}(\operatorname{cl}(C))\subset \operatorname{ri}(C)$: For $y\in \operatorname{ri}(\operatorname{cl}(C))$, take $\varepsilon>0$ with $I(x,y;[0,1+\varepsilon ])\subset \operatorname{cl}(C)$ ((d) of \Cref{prop:012-basic-ri}); let $I(x,y;1+\varepsilon)=z$, by (b) of \Cref{prop:012-basic-ri}, $y\in I(x,z;[0, 1))\subset \operatorname{ri}(C)$.
		      \end{itemize}
		      This concludes (i) and (ii). Item (iii) is straight-forward from (i) and (ii).
		\item
		      \begin{itemize}
			      \item[(ii)] For "$\subset$", note that for $C\supset \{x_i\}_i\to x\in \operatorname{cl}(C)$, we have $AC\supset \{Ax_i\}_{i}\to Ax\in A\operatorname{cl}(C)$. Conversely, when $C$ is bounded, given $AC\supset \{Ax_i\}_i\to z\in \operatorname{cl}(AC)$, boundedness of $C$ gives a subsequence of $x_i$ converging to some $x\in \operatorname{cl}(C)$, making $Ax=z\in A\operatorname{cl}(C)$.
			      \item[(i)] For "$\supset$", it suffices to show $AC\subset \operatorname{cl}(A \operatorname{ri}(C))$ in view of (a) of \Cref{prop:012-yoga-ri-cl}. By (ii):
			            \[
				            AC\subset A \operatorname{cl}(C) = A\operatorname{cl}(\operatorname{ri}(C)) \subset \operatorname{cl}(A \operatorname{ri}(C))
			            \]
			            For "$\subset$", use (d) of \Cref{prop:012-basic-ri}: for $x\in C$, $y\in \operatorname{ri}(C)$, take $\varepsilon >0$ with $I(x,y;[0,1+\varepsilon ])\subset C$:
			            \[
				            I(Ax,Ay;[0,1+\varepsilon ])=AI(x,y;[0,1+\varepsilon ])\subset AC
			            \]
			            showing that $Ay\in \operatorname{ri}(AC)$.
		      \end{itemize}
	\end{enumerate}
\end{proof}

\begin{prop}[Product Rule]
	\label{prop:012-product-rules}
	Let $C'$ be another nonempty convex set in some $\mathbb{R}^n$. Then:
	\begin{enumerate}[label=(\alph*)]
		\item $\operatorname{ri}(C)\times\operatorname{ri}(C')=\operatorname{ri}(C\times C')$
		\item $\operatorname{cl}(C)\times\operatorname{cl}(C')=\operatorname{cl}(C\times C')$
	\end{enumerate}
\end{prop}

\begin{proof}
	For (a), use (b) of \Cref{prop:012-basic-ri}. Statement (b) is also trivial.
\end{proof}

\begin{prop}[Sum Rule]
	\label{prop:012-sum-rules}
	Let $C'$ be another nonempty convex set in $\mathbb{R}^m$. Then:
	\begin{enumerate}[label=(\alph*)]
		\item $\operatorname{ri}(C)+\operatorname{ri}(C')=\operatorname{ri}(C+C')$
		\item $\operatorname{cl}(C)+\operatorname{cl}(C')\subset\operatorname{cl}(C+C')$, with "$\supset$" holds if $C$ or $C'$ is bounded.
	\end{enumerate}
\end{prop}

\begin{proof}
	By \Cref{prop:012-product-rules}, (b) of \Cref{prop:012-yoga-ri-cl} (applied to summation map $\mathbb{R}^{m}\times\mathbb{R}^{m}\to \mathbb{R}^m$), we only need to show "$\supset$" in (b) under boundedness condition. Given $\{x_i\}_i\subset C,\{y_i\}_i\subset C'$ with $\{x_i+y_i\}_i\to z\in \operatorname{cl}(C+C')$, then both $\{x_i\}_i,\{y_i\}_i$ are bounded, so $\{(x_i,y_i)\}_i$ is bounded, hence has a subsequence converging to some $(x,y)\in \operatorname{cl}(C\times C')$; we have $z=x+y\in \operatorname{cl}(C)+\operatorname{cl}(C')$.
\end{proof}

\begin{rmrk}
	\Cref{prop:012-sum-rules} can be generalized to linear combinations of convex sets; proof is similar.
\end{rmrk}

\begin{prop}[Finite Intersection Rules]
	\label{prop:012-fin-intersection-rules}
	Let $C'$ be another nonempty convex set in $\mathbb{R}^m$. Then:
	\begin{enumerate}[label=(\alph*)]
		\item $\operatorname{ri}(C)\cap\operatorname{ri}(C')\subset\operatorname{ri}(C\cap C')$.
		\item $\operatorname{cl}(C)\cap\operatorname{cl}(C')\supset\operatorname{cl}(C\cap C')$.
		\item "=" holds in (a) and (b) if $\operatorname{ri}(C)\cap\operatorname{ri}(C')\neq\emptyset$.
	\end{enumerate}
\end{prop}

\begin{proof}[Proof of (a)]
	We use (b) of \Cref{prop:012-basic-ri}. Take $x\in C\cap C'$, $y\in \operatorname{ri}(C)\cap \operatorname{ri}(C')$, choose $\varepsilon,\varepsilon'>0$ with
	\[
		I(x,y;[0,1+\varepsilon ])\subset C,\;
		I(x,y;[0,1+\varepsilon' ])\subset C'
	\]
	then by letting $\varepsilon''=\min(\varepsilon ,\varepsilon')$, we have $I(x,y;[0,1+\varepsilon'' ])\subset C\cap C'$.
\end{proof}

\begin{proof}[Proof of (b)]
	It suffices to show $\operatorname{cl}(C\cap C')\subset \operatorname{cl}(C)$, which is evident.
\end{proof}

\begin{proof}[Proof of (c)]
	Take $x\in \operatorname{ri}(C) \cap\operatorname{ri}(C')$. Let $y\in \operatorname{cl}(C)\cap \operatorname{cl}(C')$. By (b) of \Cref{prop:012-basic-ri}, we have:
	\[
		y\in I(x, y;[0, 1])=\operatorname{cl}(I(x, y;[0, 1)))\subset\operatorname{cl}(\operatorname{ri}(C)\cap \operatorname{ri}(C'))
	\]
	This gives:
	\[
		\operatorname{cl}(C)\cap \operatorname{cl}(C')\subset \operatorname{cl}(\operatorname{ri}(C)\cap \operatorname{ri}(C'))\subset \operatorname{cl}(C\cap C')
	\]
	giving the converse of (b). To show the converse of (a), note that by the above identity, $\operatorname{ri}(C)\cap \operatorname{ri}(C')$ and $C\cap C'$ has same cl, so they have same ri, so it suffices to show:
	\[
		\operatorname{ri}(C)\cap \operatorname{ri}(C')\subset
		\operatorname{ri}(\operatorname{ri}(C)\cap \operatorname{ri}(C'))
	\]
	but this follows from (a) applied to $\operatorname{ri}(C),\operatorname{ri}(C')$ (and (a) of \Cref{prop:012-yoga-ri-cl}).
\end{proof}

\begin{prop}[Inverse Image Rule]
	\label{prop:012-inv-img-rule}
	Let $A:\mathbb{R}^l\to \mathbb{R}^m$ be an affine map. Suppose $A^{-1}\operatorname{ri}(C)\neq \emptyset$.
	\begin{enumerate}[label=(\alph*)]
		\item $A^{-1}\operatorname{ri}(C)=\operatorname{ri}(A^{-1}C)$
		\item $A^{-1}\operatorname{cl}(C)=\operatorname{cl}(A^{-1}C)$
	\end{enumerate}
\end{prop}

\paragraph{}The proof of this proposition is a formal combination of the few preceding propositions.

\begin{proof}
	For (a), we consider the following factorization of maps:
	\[
		\begin{tikzcd}
			\mathbb{R}^l \arrow[rr,"A"]\arrow[dr,swap,"g"] &&\mathbb{R}^{m}\\
			&\mathbb{R}^{l+m}\arrow[ru,swap,"\pi"]&
		\end{tikzcd}
	\]
	where $g$ is the map $x\mapsto (x,Ax)$, $\pi$ is the map $(x, y)\mapsto y$. Write $\Gamma_A$ as the image of $g$ - this is sometimes referred to as the graph of $A$. We have:
	\[
		A^{-1}C=\pi (\Gamma_A\cap \mathbb{R}^m\times C)
	\]
	We wish to compute $\operatorname{ri}(A^{-1}C)$. Note first that by (b) of \Cref{prop:012-yoga-ri-cl}, \Cref{prop:012-product-rules}:
	\[
		\operatorname{ri}(\Gamma_A)=\Gamma_A,\; \operatorname{ri}(\mathbb{R}^m\times C)= \mathbb{R}^m \times \operatorname{ri}(C)
	\]
	so by (b) of \Cref{prop:012-yoga-ri-cl}, \Cref{prop:012-fin-intersection-rules}, and the condition $A^{-1}\operatorname{ri}(C)\neq\emptyset$:
	\begin{align*}
		\operatorname{ri}(A^{-1}C) & =\operatorname{ri}(\pi (\Gamma_A\cap \mathbb{R}^m\times C))                      \\
		                           & =\pi \operatorname{ri}(\Gamma_A\cap \mathbb{R}^m\times C)                        \\
		                           & =\pi (\operatorname{ri}( \Gamma_A )\cap \operatorname{ri}(\mathbb{R}^m\times C)) \\
		                           & =\pi ( \Gamma_A \cap \mathbb{R}^m\times \operatorname{ri}(C))                    \\
		                           & =A^{-1}\operatorname{ri}(C)
	\end{align*}
	For (b), we have:
	\[
		\operatorname{cl}(\Gamma_A)=\Gamma_A,\; \operatorname{cl}(\mathbb{R}^m\times C)= \mathbb{R}^m \times \operatorname{cl}(C)
	\]
	where the first "$=$" is by the fact that $A$ is affine. In a similar way, we get:
	\begin{align*}
		\operatorname{cl}(A^{-1}C) & =\operatorname{cl}(\pi (\Gamma_A\cap \mathbb{R}^m\times C))                      \\
		                           & \supset\pi \operatorname{cl}(\Gamma_A\cap \mathbb{R}^m\times C)                  \\
		                           & =\pi (\operatorname{cl}( \Gamma_A )\cap \operatorname{cl}(\mathbb{R}^m\times C)) \\
		                           & =\pi ( \Gamma_A \cap \mathbb{R}^m\times \operatorname{cl}(C))                    \\
		                           & =A^{-1}\operatorname{cl}(C)
	\end{align*}
	Conversely, given $A^{-1}C\supset (x_k)_k\to x\in \operatorname{cl}(A^{-1}C)$, then $C\supset (Ax_k)_k$ converges, to $Ax\in \operatorname{cl}(C)$.
\end{proof}

\paragraph{}The following proposition gives ways to represent $\operatorname{ri}(C)$ via fibers.

\begin{prop}[Slice and Fibers Rules]
	\label{prop:012-slice-fiber-rules}
	For $x\in \mathbb{R}^l$ where $l< m$, define a set of "slice supports":
	\[
		D=\{x\in \mathbb{R}^l:C_{x,:}\neq\emptyset\}
	\]
	Then we have:
	\[
		\operatorname{ri}(C)_{x,:}=
		\begin{cases}
			\emptyset                  & \text{ if }x\notin \operatorname{ri}(D) \\
			\operatorname{ri}(C_{x,:}) & \text{ if }x\in \operatorname{ri}(D)
		\end{cases}
	\]
	where we use the convention $\operatorname{ri}(\emptyset)=\emptyset$. One can formulate analogous statements for fibers.
\end{prop}

\paragraph{}This generalizes the product rule:

\begin{proof}
	Again, let $\pi:\mathbb{R}^{m}\to \mathbb{R}^l$ be projection $(x,y)\mapsto x$. We then have:
	\[
		\operatorname{ri}(D)=\operatorname{ri}(\pi C)=\pi \operatorname{ri}(C)
	\]
	Therefore, if $x\notin \operatorname{ri}(D)$, then we get $C_{x,:}=\emptyset$. Otherwise, we have:
	\begin{align*}
		\{x\}\times \operatorname{ri}(C)_{x,:} & = \{x\}\times \mathbb{R}^m\cap \operatorname{ri}(C)                      = \operatorname{ri}(\{x\}\times \mathbb{R}^m)\cap \operatorname{ri}(C) \\
		                                       & = \operatorname{ri}(\{x\}\times \mathbb{R}^m\cap C)                      = \operatorname{ri}(\{x\}\times C_{x,:})
	\end{align*}
	where we used \Cref{prop:012-product-rules} for the second "=", \Cref{prop:012-fin-intersection-rules} for the third "=" along with the proven fact $\operatorname{ri}(D)=\pi \operatorname{ri}(C)$ and the assumption $x\in \operatorname{ri}(D)$.
\end{proof}

