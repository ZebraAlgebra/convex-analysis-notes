\section{Polyhedral Functions}

\paragraph{}Let $f$ be a scalar function on $\mathbb{R}^m$.

\begin{defn}[Polyhedral Functions]\label{defn:026-polyhedral-functions}
	We say that $f$ is polyhedral if $f$ is proper and $\operatorname{epi}(f)$ is polyhedral.
\end{defn}

\paragraph{}It turns out that $f$ has simple descriptions.

\begin{prop}[Representation of Polyhedral Functions]\label{prop:026-polyhedral-function-representation}
	$f$ is polyhedral iff $\operatorname{dom}(f)$ is polyhedral, with $f=\sup_{i=1}^l f_i$ on $\operatorname{dom}(f)$ where each $f_i$ is affine with $l\geq 1$.
\end{prop}
\begin{proof}
	For "only if", suppose $\operatorname{epi}(f)$ is polyhedral and proper, we can write $\operatorname{epi}(f)$ as:
	\[
		\operatorname{epi}(f)=\left(\bigcap_{i=1}^{n_X}E_{(x_i,-1),a_i,-} \right)\cap\left(\bigcap_{j=1}^{n_Y}E_{(y_j,0),b_j,-} \right)\cap\left(\bigcap_{k=1}^{n_Z}E_{(z_k,1),c_k,-} \right)
	\]
	for some subsets $X=\{x_i\}_{i=1}^{n_X},Y=\{y_j\}_{j=1}^{n_Y},Z=\{z_k\}_{k=1}^{n_Z}$ of $\mathbb{R}^m$ with $O\notin Y$. As $\operatorname{epi}(f)$ is upward-closed and proper, $|X|>0=|Z|$. The intersections involving $Y$ gives $\operatorname{dom}(f)=\bigcap_{j=1}^{n_Y}E_{y_j,b_j,-}$, while those for $X$ gives the representation of $f$ on $\operatorname{dom}(f)$. On the other hand, if $f=\sup_{i=1}^kf_i$ with $f_i(x)=a_i^Tx+b_i$, then $\operatorname{epi}(f)=\left(\operatorname{dom}(f)\times \mathbb{R}\right)\cap\left(\bigcap_{i=1}^kE_{(a_i,-1),-b_i,-}\right)$, which is polyhedral.
\end{proof}

\begin{prop}[Yoga of Polyhedral Functions]\label{prop:026-yoga-polyhedral}
	We have the following yogas for polyhedral functions:
	\begin{enumerate}[label=(\alph*)]
		\item (Summation) Given polyhedral $f_1,f_2$ on $\mathbb{R}^m$ with $\operatorname{dom}(f_1)\cap\operatorname{dom}(f_2)\neq\emptyset$, $f_1+f_2$ is polyhedral.
		\item (Composition with Affine) Given polyhedral $f$ on $\mathbb{R}^m$ and affine $A:\mathbb{R}^l\to \mathbb{R}^m$ with $\operatorname{dom}(f)\cap A(\mathbb{R}^l)\neq\emptyset$, $f\circ A$ is polyhedral.
	\end{enumerate}
\end{prop}

\begin{proof}
	For (a), suppose $f_i(x)=\sup_{j=1}^{n_i}(a_{i,j}^Tx+b_{i,j})$ on $\operatorname{dom}(f_i)$, then $f_1+f_2=\sup_{1\leq j_1\leq n_1,\; 1\leq j_2\leq n_2}(a_{1,j_1}+a_{2,j_2})x+(b_{1,j_1}+b_{2,j_2})$ on $\operatorname{dom}(f_1)\cap \operatorname{dom}(f_2)=\operatorname{dom}(f_1+f_2)$. For (b), if $f(x)=\sup_{i=1}^{n}(a_{i}^Tx+b_{i})$, then $\operatorname{dom}(f\circ A)=A^{-1}\operatorname{dom}(f)$, and that on $\operatorname{dom}(f\circ A)$ we have $(f\circ A)(x)=\sup_{i=1}^{n}((A^Ta_{i})^Tx+b_{i})$.
\end{proof}
