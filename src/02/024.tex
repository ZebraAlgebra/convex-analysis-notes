\section{Recession Cones and Constancy Space of Functions}
\label{sect:024}

\paragraph{}Fix a proper, convex, closed function $f$ on $\mathbb{R}^m$. We can consider recession cones of $\operatorname{epi}(f),V_{f,u}$.

\begin{prop}[Recession Cones of Level Sets and Epigraphs]\label{prop:024-recess-epi-level}
	.
	\begin{enumerate}[label=(\alph*)]
		\item (Recession Cones of Epigraphs) The recession cone $R_{\operatorname{epi}(f)}$ is epigraph-like; we define $r_f:=R_{\operatorname{epi}(f)}^\vee$.
		\item (Recession Cones of Level Sets are Slices) The recession cones of $V_{f,u}$ can be described as follows:
		      \[
			      R_{V_{f,u}}=\begin{cases}
				      \mathbb{R}^m                                 & \text{ if }V_{f,u}=\emptyset    \\
				      \left(R_{\operatorname{epi}}(f)\right)_{:,0} & \text{ if }V_{f,u}\neq\emptyset
			      \end{cases}
		      \]
		      Note: when $V_{f,u}\neq\emptyset$, $R_{V_{f,u}}$ is independent of $u$; we denote it as $R_f$. Similarly, write $L_{f}=R_f\cap -R_f$. We call $R_f,L_f$ the recession cone and constancy space of $f$.
		\item (Recession Function Characterizes Recession Cone and Constancy Space) Suppose $f$ is closed, then:
		      \[
			      R_f=r_f^{-1}\left(-\mathbb{R}_{+}\right),\;
			      L_f=\left(r_f^{-1}\left(-\mathbb{R}_{+}\right)\right)\cap\left(-r_f^{-1}(-\mathbb{R}_{+})\right)=\left(r_f^{-1}(0)\right)\cap\left(-r_f^{-1}(0)\right)
		      \]
		\item (Formula for Recession Function) Suppose $f$ is closed, then for $x\in \operatorname{dom}(f)$:
		      \[
			      r_f(d)=\underset{\alpha\in \mathbb{R}_{++}}{\operatorname{sup}}\frac{f(x+\alpha d)-f(x)}{\alpha }
			      =\lim_{\alpha \to+\infty}\frac{f(x+\alpha d)-f(x)}{\alpha }
		      \]
	\end{enumerate}
\end{prop}

\begin{proof}
	For (a), use the fact that $\operatorname{epi}(f)$ is epigraph-like and closed.	For (b), use $V_{f,u}=\operatorname{epi}(f)_{:,u}=\operatorname{epi}(f)\cap \left(\mathbb{R}^m\times \{u\}\right)$ and \Cref{prop:013-yoga-recession} (giving $R_{V_{f,u}}=R_{\operatorname{epi}(f)}\cap \mathbb{R}^m\times\{0\})$. For (c), (b) gives:
		\[
			R_f=r_f^{-1}\left(-\mathbb{R}_{+}\right),\;
			L_f=\left(r_f^{-1}\left(-\mathbb{R}_{+}\right)\right)\cap\left(-r_f^{-1}(-\mathbb{R}_{+})\right)
		\]
		For $L_f=\left(r_f^{-1}(0)\right)\cap\left(-r_f^{-1}(0)\right)$, use convexity of $f$ and $r_f(0)=0$. For (d), convexity of $f$ gives the second "=" (as the expression is monotically increasing in $\alpha $). For the first "=", for fixed $d$ and $x\in \operatorname{dom}(f)$, we have by properness and closedness that:
	\begin{align*}
		r_f(d) & =\inf \left\{u\in \mathbb{R}:(x,f(x))+\rho_{(d,u)}\subset \operatorname{epi}(f)\right\}                                       \\
		       & = \inf \left\{u\in \mathbb{R}:f(x+\alpha d)\leq f(x)+\alpha u\text{ for }\alpha \in \mathbb{R}_+\right\}                      \\
		       & = \inf \left\{u\in \mathbb{R}:u\geq \alpha^{-1}\left(f(x+\alpha d)- f(x)\right)\text{ for }\alpha \in \mathbb{R}_{++}\right\} \\
		       & = \sup_{\alpha \in \mathbb{R}_{++}} \alpha^{-1}\left(f(x+\alpha d)- f(x)\right)\qedhere{}
	\end{align*}
\end{proof}


