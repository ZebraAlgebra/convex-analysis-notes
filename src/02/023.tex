\section{Closures of Functions}
\label{sect:023}

\paragraph{}In this section, we construct convex, closed, or convex and closed functions that "best approximates" a given function. The idea is that we can apply $\operatorname{cl},\operatorname{epi}$ to epigraphs. For this section, fix a scalar function $f$.

\begin{defn}[Various Closures of Functions]
	\label{defn:023-closure-funcs}
	We define:
	\[
		\operatorname{conv}f=
		\operatorname{conv}(\operatorname{epi}(f))^\vee,\;
		\operatorname{cl}f=
		\operatorname{cl}(\operatorname{epi}(f))^\vee,\;
		\operatorname{clconv}f=
		\operatorname{cl}(\operatorname{conv} (\operatorname{epi}(f)))^\vee
	\]
\end{defn}
\paragraph{}Note it is in general not true that $\operatorname{cl}(\operatorname{conv} (\operatorname{epi}(f)))=\operatorname{conv}(\operatorname{cl} (\operatorname{epi}(f)))$. The set $\operatorname{conv}(\operatorname{epi}(f))$ might also not be epigraph-like. Let us compare the functions in \Cref{defn:023-closure-funcs} by the following series of propositions.

\begin{prop}[Universal properties of cl, conv, clconv]\label{prop:023-closure-univ}
	The functions $\operatorname{conv}f,\operatorname{cl}f,\operatorname{clconv}f$ are the largest (in the dominance relation) convex, closed, convex and closed function dominated by $f$.
\end{prop}

\begin{proof}
	This is by the universal properties of $\operatorname{conv},\operatorname{cl}$ for subsets; see remark after \Cref{defn:021-conv-usual}.
\end{proof}


\begin{prop}[Optimality]
	\label{prop:023-closure-optimality}
	The functions $f,\operatorname{cl}f,\operatorname{conv}f,\operatorname{clconv}f$ have the same inf. Therefore, every argmin of $f$ is an argmin of $\operatorname{cl}f,\operatorname{clconv}f,\operatorname{conv}f$.
\end{prop}

\begin{proof}
	For a scalar function $g$, write $g^\ast:=\inf g$. By the containments of epigraphs, it suffices to show $f^\ast\leq(\operatorname{clconv}f)^\ast$. Note first that:
	\[
		f^\ast = \sup \{u\in \mathbb{R}: \operatorname{epi}(f)\subset \mathbb{R}^n\times (u + \mathbb{R}_+)\}
	\]
	However, given $u$, whenever $\operatorname{epi}(f)\subset \mathbb{R}^n\times (u + \mathbb{R}_+)$, since $\mathbb{R}^n\times (u + \mathbb{R}_+)$ is closed and convex, we have:
	\[
		\operatorname{cl}(\operatorname{conv}(\operatorname{epi}(f)))\subset \mathbb{R}^n\times (u + \mathbb{R}_+)\qedhere{}
	\]
\end{proof}

\paragraph{}We next compare $\operatorname{cl}f$ and $f$ when $f$ is convex. We would use the following lemma.

\begin{lemm}[Improper, Closed, Convex]
	\label{lemm:023-improper-closed-convex}
	Suppose $f$ is improper, convex, closed, then $f(\mathbb{R}^m)\subset\{\pm\infty\}$.
\end{lemm}

\begin{proof}
	May assume $\operatorname{dom}(f)\neq \emptyset$, and $f(x)=-\infty$ for some $x\in \operatorname{dom}(f)$. Take $y\in \operatorname{dom}(f)$, then $f(I(x,y;[0, 1))=\{-\infty\}$ by convexity, hence $f(y)=-\infty$ by closedness.
\end{proof}


\begin{prop}[Yoga of Taking Closures of Convex Functions]
	\label{prop:023-compare-func-closure}
	Suppose $f$ is convex.
	\begin{enumerate}[label=(\alph*)]
		\item (Comparison of Effective Domain) We have
		      \[
			      \operatorname{cl}(\operatorname{dom}(f))=
			      \operatorname{cl}(\operatorname{dom}(\operatorname{cl}f)),\;
			      \operatorname{ri}(\operatorname{dom}(f))=
			      \operatorname{ri}(\operatorname{dom}(\operatorname{cl}f))
		      \]
		\item (Comparison of Properness) $f$ is proper iff $\operatorname{cl}f$ is proper.
		\item (Comparison of Values I) $f,\operatorname{cl}f$ coincides on $\operatorname{ri}(\operatorname{dom}(f))$.
		\item (Comparison of Values II) Furthremore, for $x\in \operatorname{ri}(\operatorname{dom}(f))$, we have:
		      \[
			      (\operatorname{cl}f)(y)=\lim_{\alpha \to0^+}f(I(y,x;\alpha ))\;\text{ for }y\in \mathbb{R}^n
		      \]
	\end{enumerate}
\end{prop}

\paragraph{}In short, the geometry of $\operatorname{dom}(f),\operatorname{dom}(\operatorname{cl}f)$ are not too different by (a), (b), $f,\operatorname{cl}f$ differ only outside of $\operatorname{ri}(\operatorname{dom}(f))$ by (c), and that on those other points, (d) can be used to describe the values of $f$.

\begin{proof}
	For (a), it suffices to show the statement for ri. Let $\pi:\mathbb{R}^{n+1}\to \mathbb{R}^n,(x,u)\mapsto x$ be projection, then:
	\begin{align*}
		\operatorname{ri}(\operatorname{dom}(f)) & =\operatorname{ri}(\pi \operatorname{epi}(f))                 \\
		                                         & =\pi\operatorname{ri}(\operatorname{epi}(f))                  \\
		                                         & =\pi\operatorname{ri}(\operatorname{epi}(\operatorname{cl}f)) \\
		                                         & =\operatorname{ri}(\pi\operatorname{epi}(\operatorname{cl}f)) \\
		                                         & =\operatorname{ri}(\operatorname{dom}(\operatorname{cl}f))
	\end{align*}
	where the third "=" is by $\operatorname{cl}(\operatorname{epi}(f))=\operatorname{cl}(\operatorname{epi}(\operatorname{cl}f))$, the first and fifth are by definition of epigraphs, the second and forth are by the slice principle. For (c), we use slice principle. For $x\in \operatorname{ri}(\operatorname{dom}(f))$:
	\begin{align*}
		(f(x),+\infty) & =\operatorname{ri}(\operatorname{epi}(f)_{x,:})                  \\
		               & =\operatorname{ri}(\operatorname{epi}(f))_{x,:}                  \\
		               & =\operatorname{ri}(\operatorname{epi}(\operatorname{cl}f))_{x,:} \\
		               & =\operatorname{ri}(\operatorname{epi}(\operatorname{cl}f)_{x,:}) \\
		               & =(\operatorname{cl}f(x),+\infty)
	\end{align*}
	where the third "=" is by $\operatorname{cl}(\operatorname{epi}(f))=\operatorname{cl}(\operatorname{epi}(\operatorname{cl}f))$, the first and fifth are by definition of epigraphs, the second and forth are by the slice principle and (a). For (d), we have two cases:
	\begin{itemize}
		\item Case $y\in \operatorname{dom}(\operatorname{cl}f)$: by (b) of \Cref{prop:012-basic-ri}, $I(x,y;[0, 1))\subset \operatorname{ri}(\operatorname{dom}(f))$, so we see that $f,\operatorname{cl}f$ coincides on $I(x,y;[0,1))$ by (c). If $\operatorname{cl}f(y)\in \mathbb{R}$, then by \Cref{coro:022-line-continuity}, (d) holds in this case; otherwise, if $\operatorname{cl}f(y)=\infty$, then $\operatorname{cl}f$ is $-\infty$ on $I(x,y;[0,1))$ by the lemma, so (d) holds.
		\item Case $y\notin \operatorname{dom}(\operatorname{cl}f)$: since $\operatorname{cl}f$ is closed, we have:
		      \[
			      \underset{\alpha \to 0^+}{\operatorname{liminf}}\operatorname{cl}f(I(y,x;\alpha ))\leq \operatorname{cl}f(y)=\infty
		      \]
		      hence (d) holds, as $f$ dominates $\operatorname{cl}f$.
	\end{itemize}
	For (b), if $\operatorname{cl}f$ is improper, then by \Cref{lemm:023-improper-closed-convex}, it takes $-\infty$ on $\operatorname{dom}(f)$, which would make $f$ improper by (a), (c). That $f$ is improper implies $\operatorname{cl}(f)$ is easy to show.
\end{proof}

% TODO: complete adaptation
