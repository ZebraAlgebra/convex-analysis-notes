\chapter{Duality Theory in Optimization}
\label{chap:04}

\paragraph{}In this section, we utilize the MC/MC framework to derive various duality results in convex and linear optimization. First we define what an optimization problem is:

\begin{defn}[Optimization Problem]\label{defn:040-opt-prob}
	An optimization problem is a problem of the form:
	\begin{align*}
		\text{Minimize}\quad   & f(x)                                   \\
		\text{subject to}\quad & x\in X\subset \mathbb{R}^m, g(x)\leq O
	\end{align*}
	where $f$ (resp. $g,X\cap \operatorname{dom}(f)\subset \mathbb{R}^m,\{x\in X:g(x)\leq O\}$) is called the objective function (resp. constraint function, basic feasibility region, feasibility region) of the problem. The optimal value is written as $f^\ast$. The set of feasible solutions is written as $F$, with the optimal points is written as $F^\ast$.
\end{defn}

\paragraph{}For convex problems, $f,X$ are convex, with $g$ having convex component functions, so $F$ is convex. In linear programming, $f,g$ are affine, with $X=\mathbb{R}^m$, so $F$ is polyhedral. For maximization problems, we regard it as the same as a minimization problem over $-f$ with other components fixed. For this chapter, we will always assume that the problem is convex.

\section{Farkas-type, Lagrangian, Dual Problems}
\label{sect:041}

\paragraph{}The MC/MC duality theory in \Cref{chap:03} gives rise to many duality results in optimization.

\paragraph{}The first duality result we shall prove is Farkas lemma, one in its original form for linear optimization, the other is a generalized form. The linear form of Farkas lemma asserts that a polyhedral set is non-empty iff there is a containment of certain polyhedral cones. The statements are as follows:

\begin{prop}[Linear Farkas Lemma]\label{prop:041-farkas-lemma-linear}
	Let $A\in \mathbf{M}_{m,r}(\mathbb{R})$, $c\in \mathbb{R}^m$.
	\begin{enumerate}[label=(\alph*)]
		\item The system $Ay=c,y\leq O$ has solution iff $A^Tx\leq O$ implies $c^Tx\geq 0$.
		\item The system $Ay\leq c$ has solution iff $A^Tx=O,x\geq O$ implies $c^Tx\geq 0$.
	\end{enumerate}
\end{prop}

\begin{proof}
	(b) follows from (a) by writing $Ay\leq c$ as $Ay^+-Ay^--I_mz=c,(y^+,y^-,z)\leq O$. For "only if"'s (a):
	\[
		c^Tx=y^T(A^Tx)\geq 0
	\]
	if $Ay=c,y\leq O$ for some $y$. For "if", by (b) of \Cref{prop:041-farkas-lemma-nonlinear} with $X=\mathbb{R}^m$, $f(x)=c^Tx$, $g(x)=A^Tx$, $x_0=O$, there is $y_0\leq O_r$, with $f(x)-y_0^Tg(x)=(c-Ay_0)^Tx\geq 0$ for any $x\in \mathbb{R}^m$, giving $Ay_0=c$.
\end{proof}

\paragraph{}A loose thread in the above proof is deferred to the nonlinear Farkas lemma below, which would be a starting point of many duality results to be introduced in the subsequent subsections.

\begin{prop}[Nonlinear Farkas Lemma]\label{prop:041-farkas-lemma-nonlinear}
	Let $X\subset \mathbb{R}^m$ be nonempty, convex, $f:X\to \mathbb{R}$ be convex, $g=(g_1,\dotsc,g_r):X\to \mathbb{R}^r$ with each $g_i$ convex. Suppose $g(x)\leq O,x\in X$ implies $f(x)\geq 0$. Define $Q_0$ as:
	\[
		Q_0=\{y\in \mathbb{R}^r:y\leq O,f(x)-y^Tg(x)\geq0,\text{ for all }x\in X\}
	\]
	\begin{enumerate}[label=(\alph*)]
		\item (Slater's Condition I) If there is some $x_0\in X$ with $g(x_0)<O$, then $Q_0$ is nonempty and compact.
		\item (Slater's Condition II) If $g$ is affine with some $x_0\in \operatorname{ri}(X)$ with $g(x_0)\leq O$, then $Q_0$ is nonempty.
	\end{enumerate}
\end{prop}

\paragraph{}Several comments on this proposition:
\begin{itemize}
	\item The condition "$g(x)\leq O,x\in X$ implies $f(x)\geq 0$" is analogous to the one given in \Cref{prop:041-farkas-lemma-linear}, hence the name "Nonlinear Farkas Lemma"; let us call this the "Farkas condition".
	\item The function $f(x)-y^Tg(x)$ is more often written as $f(x)+y^Tg(x)$, and is often referred to as a Lagrangian function; the emergence of such function can be explained via MC/MC in several different ways, one of such given in the proof below. Here we will just refer to $f(x)-y^Tg(x)$ as the Lagrangian.
	\item The set $Q_0$ is in fact a certain type of sublevel set of the negative of a dual function $-q$ (just like the set of dual optimal values $Q^\ast$ introduced in \Cref{chap:03}, which is $V_{-q,q^\ast}$).
\end{itemize}

\begin{proof}
	The idea is to apply MC/MC to the set $M$ defined by:
	\[
		M:=\left\{
		(y,u):(g(x),f(x))\leq (y,u)\text{ for some }x\in X
		\right\}
	\]
	One can easily check that this set is convex (either directly, or notice that $M$ is the image of a projection of a convex subset of $\bigtimes_{i=1}^{r+1} \operatorname{epi}(g_i)$ with $g_{r+1}=f$). The corresponding $q,w^\ast$ associated to $M$ are then:
	\[
		w^\ast = \inf \left\{f(x):x\in X,g(x)\leq O\right\}
	\]
	\[
		q(y) = \inf (-y,1)^TM = \begin{cases}
			\inf_{x\in X} f(x)-y^Tg(x) & \text{ if }y\leq O \\
			-\infty                    & \text{ otherwise}
		\end{cases}
	\]
	Therefore, $Q_0=V_{-q,0}$. Also, the Farkas condition gives $w^\ast \geq 0 > -\infty$. Now $M$ admits a decomposition:
	\[
		M=M'-\left(P\times \{0\}\right)
	\]
	where $P$ is the polyhedral set $\{x:x\leq O_r\}$, $M'$ is $\left(g\times {\operatorname{id}}_{\mathbb{R}}\right)\operatorname{epi}(f)$, We have $\pi M'=g(X),\pi M=g(X)-P$ where $\pi: \mathbb{R}^{m+1}\to \mathbb{R}^m$ is the map $(x,u)\mapsto x$. Note: $M'$ is convex if $g$ is affine. Now we prove (a) and (b):
	\begin{enumerate}[label=(\alph*)]
		\item By \Cref{prop:032-reachability} and \Cref{prop:032-optimal-set}, we verify: $w^\ast>-\infty$, $M$ convex, $O\in \operatorname{int}(\pi M)=\operatorname{int}(g(X)-P)$. First two conditions were verified above, $g(x_0)<O$ for some $x_0\in X$ gives the third.
		\item By \Cref{coro:032-variants-reachability}, we verify: $w^\ast>-\infty$, $M'$ convex, $\operatorname{ri}(\pi M')\cap P\neq\emptyset$. First condition was verified above. Last condition is by $\operatorname{ri}(\pi M')=\operatorname{ri}(g(X))=g(\operatorname{ri}(X))$ (as $g$ is affine), so the assumption in (b) implies the third condition. \qedhere
	\end{enumerate}
\end{proof}

\begin{defn}[Lagrangian, Dual Problem]\label{defn:041-lagrangian-dual-prob}
	For the optimization problem in \Cref{defn:040-opt-prob}, define the Lagrangian as $L(x,y)=f(x)-y^Tg(x)$, the dual function $q(y)=\inf_{x\in X}L(x,y)$, and the dual problem as:
	\begin{align*}
		\text{Maximize}\quad   & q(y)=\inf_{x\in X}L(x,y) \\
		\text{subject to}\quad & y\leq O
	\end{align*}
	The optimal value and points are written as $q^\ast$, $Q^\ast$, the convex set of feasible points written as $Q=q^{-1}\mathbb{R}$.
\end{defn}


\section{Convex Optimization Duality - General Results}
\label{sect:042}

\paragraph{}As in our discussion in MC/MC framework in \Cref{sect:031}, we can ask the following analogous questions:
\begin{itemize}
	\item (Weak Duality) Do we have $f^\ast\geq q^\ast$?
	\item (Strong Duality) When do we have $f^\ast=q^\ast$?
	\item (Optimality) Do we have $F^\ast\neq\emptyset,Q^\ast\neq\emptyset$, given information on nonemptiness of $F,Q$? Are they compact? In general, Farkas lemma gives results on $Q^\ast$; the problem is more intricate for $F^\ast$.
\end{itemize}

\paragraph{}We can first give some partial answers, some of which utilizes both the proof and the statement of the nonlinear Farkas lemma:
\begin{itemize}
	\item (Weak duality) By the proof of \Cref{prop:041-farkas-lemma-nonlinear}, $f^\ast$, $q^\ast$ are min-common and max-crossing values, so MC/MC weak duality gives $f^\ast\geq q^\ast$.
	\item (Strong Duality) The duality gap $f^\ast-q^\ast$ has the following bound:
	      \[
		      \underset{(x,y)\in F\times Q}{\operatorname{inf}}\left(f(x)-q(y)\right)\leq
		      \underset{(x,y)\in F\times Q}{\operatorname{inf}}\left(f(x)-L(x,y)\right)=
		      \underset{(x,y)\in F\times Q}{\operatorname{inf}}y^Tg(x)
	      \]
	      Note that $g(F)\leq O,Q\leq O$. One can replace $F\times Q$ by $F^\ast\times Q^\ast$ if both $F^\ast,Q^\ast$ are nonempty. A particular case is given by the notion of complementary slackness; see \Cref{prop:042-complementary-slackness}.
	      %TODO: add reference for reachability of linear programs.
	\item (Optimality) If one applies \Cref{prop:041-farkas-lemma-nonlinear} to \Cref{defn:040-opt-prob} with $f$ being $f-f^\ast$, assuming $f^\ast$ is finite, then the corresponding $Q_0$ is the same as $Q^\ast$. In other words, the "Slater conditions" in fact can be used to give conditions for nonemptiness or compactness of optimal sets. In fact, the situation is even better: Slater conditions gives strong duality, as for $y_0\in Q^\ast$ in this case, we have:
	      \[
		      0\leq \underset{x\in X}{\operatorname{inf}} \left(f(x)-f^\ast-y_0^T g(x)\right) \leq q^\ast-f^\ast
	      \]
\end{itemize}

\paragraph{}We collect these as propositions.

\begin{prop}[Weak Duality]\label{prop:042-weak-duality}
	We have $f^\ast\geq q^\ast$. Consequently, if the primal (resp. dual) problem has no optimal values, then the other is infeasible.
\end{prop}

\begin{prop}[Complementary Slackness]\label{prop:042-complementary-slackness}
	Given $x_0\in F$ and $y_0\leq O\in \mathbb{R}^r$, then TFAE:
	\begin{enumerate}[label=(\alph*)]
		\item (Strong Duality and Reachability) $(x_0,y_0)\in F^\ast\times Q^\ast$ and $f^\ast=q^\ast$.
		\item (Complementary Slackness) $x_0\in \underset{x\in X}{\operatorname{argmin}}L(x,y_0)$, and $y_0^Tg(x_0)=0$.
	\end{enumerate}
\end{prop}

\begin{proof}
	"(b) implies (a)" is by the bound on $f^\ast - q^\ast$ given above. The converse uses weak duality.
\end{proof}

\begin{prop}[Slater Condition]\label{prop:042-slater-optimality}
	Assume $f^\ast<\infty$, then
	\begin{enumerate}[label=(\alph*)]
		\item (Slater Condition I) If there is some $x_0\in X$ with $g(x_0)<O$, then $Q^\ast$ is nonempty and compact.
		\item (Slater Condition II) If $g$ is affine with some $x_0\in \operatorname{ri}(X)$ with $g(x_0)\leq O$, then $Q^\ast$ is nonempty.
	\end{enumerate}
	If either of (a) or (b) holds, then $f^\ast=q^\ast$.
\end{prop}

\section{Duality for Linear Programs}
\label{sect:043}

\paragraph{}Assume the optimization problem in \Cref{defn:041-lagrangian-dual-prob} is linear in this section. Let the objective, constraints be $f(x)=c^Tx,g(x)=Ax-b$, with $c\in \mathbb{R}^m,A\in \mathbf{M}_{r,m}(\mathbb{R})$, rows of $A$ being $a_1^T,\dotsc,a_r^T$; note the basic feasibility region $X=\mathbb{R}^m$. We have the following description of the Lagrangian and the dual function:

\[
	L(x,y) = (c-A^Ty)^Tx+b^Ty
\]
\[
	q(y) = \inf_{x\in \mathbb{R}^m} (c-A^Ty)^Tx + b^Ty=\begin{cases}
		-\infty & \text{ if }A^Ty\neq c \\
		b^Ty    & \text{ if }A^Ty=c
	\end{cases}
\]

\paragraph{}Therefore, the dual problem is still a linear program, the same as:
\begin{align*}
	\text{Maximize}\quad   & b^Ty           \\
	\text{subject to}\quad & y\leq O,A^Ty=c
\end{align*}

\paragraph{}Note that the dual problem of the dual problem is "essentially the same" as the primal problem (try verifying this yourself!). Now we will study the relations between $f^\ast,q^\ast$, as well as $F^\ast,Q^\ast$.

\begin{prop}[Primal Reachability]\label{prop:043-linear-reachability}
	Suppose $f^\ast\in \mathbb{R}$ (resp. $q^\ast\in \mathbb{R}$), then $F^\ast\neq\emptyset$ (resp. $Q^\ast\neq\emptyset$).
\end{prop}

\begin{proof}
	We only show the statement for $f,F$ assuming $c\neq O$. Note: $F^\ast=V_{f,f^\ast}\cap F=\left(\bigcap_{k\in \mathbb{N}}V_{f,f^\ast+ k^{-1}}\right)\cap F$. We will use \Cref{prop:014-non-empt-II} to show $F^\ast\neq\emptyset$ - using the same notation as in that proposition, we let:
	\[
		C_0=F,\;C_k=V_{f,f^\ast+k^{-1}}\;\text{ for }k\geq 1
	\]
	then $C_0$ is polyhedral (hence retractive, closed, convex), $R=R_f=E_{c,0;-}$, $L=L_f=H_{c,0}$. As $f^\ast>-\infty$, we have $R_F\cap E_{c,0;-}\subset H_{c,0}$, so \Cref{prop:014-non-empt-II} applies to show that $F^\ast\neq\emptyset$.
\end{proof}

\begin{prop}[Linear Programming Duality]\label{prop:043-linear-strong-duality}For linear programs, we have the following:
	\begin{enumerate}[label=(\alph*)]
		\item (Weak Duality) We have $f^\ast\geq q^\ast$.
		\item (Strong Duality) If any of $f^\ast$ or $q^\ast$ is finite, $f^\ast=q^\ast$.
		\item (Reachability) Given $(x_0,y_0)\in F\times Q$, we have $(x_0,y_0)\in F^\ast\times Q^\ast$ iff $y_0^Tg(x_0)=0$.
	\end{enumerate}
\end{prop}

\begin{proof}
	(a) is given before. For (b), we show that if $f^\ast\in \mathbb{R}$, then \Cref{prop:042-complementary-slackness} is satisfied; the proof for $q^\ast$ is similar. We have $F\neq \emptyset$. By \Cref{prop:043-linear-reachability}, we have some $x_0\in F^\ast$. We will construct a particular $y_0\in Q$ using linear Farkas lemma, then use complementary slackness to show strong duality. Define:
	\[
		J=\{j:a_j^Tx_0=b_j\}
	\]
	Since $x_0\in F^\ast$, we have the Farkas condition: $a_j^Tx\geq 0$ for all $j\in J$ implies $c^Tx\geq 0$; otherwise, $x_0+\delta x$ would be a better feasible solution than $x$ for some small $\delta>0$. Therefore, we may write:
	\[
		A^Ty_0=c,\; \text{ with }y_0\leq O,\;(y_0)_j=0\text{ if }j\notin J
	\]
	Complementary slackness is satisfied in this case, giving (b):
	\[
		c^Tx_0-b^Ty_0=(Ax_0-b)^Ty_0=\left(\sum_{j\in J}+\sum_{j\notin J}\right)(a_j^Tx_0-b_j)y_{0,j}=0
	\]
	For (c), $F\times Q\neq\emptyset$ implies $f^\ast,q^\ast$ are both in $\mathbb{R}$ by (a), (b); now use $f^\ast-q^\ast\leq f(x_0)-q(y_0)=y_0^Tg(x_0)$.
\end{proof}


\section{Variants of Convex Programming Duality}
\label{sect:044}

\paragraph{}We first show a slight extension fo Slater's condition. In the case where some constraints are polyhedral and some are not, we get the following:

\begin{coro}[Extended Slater's Condition]\label{coro:044-slater-combined}
	Suppose in the \Cref{defn:040-opt-prob} that the constraints $g(x)$ can be written as $(g^{(1)}(x),g^{(2)}(x))$ where $g^{(i)}:\mathbb{R}^m\to \mathbb{R}^{r_i}$, with $g^{(1)}$ affine, $g^{(2)}$ convex, then under the assumption that the following holds:
	\begin{enumerate}[label=(\alph*)]
		\item There is $x^{(1)}\in \operatorname{ri}(X)$ with $g^{(1)}(x)\leq O$
		\item There is $x^{(2)}\in X$ with $g^{(1)}(x)\leq O,g^{(2)}(x)<O$
	\end{enumerate}
	then strong duality holds ($q^\ast=f^\ast$) and $Q^\ast\neq\emptyset$.
\end{coro}
In particular, if there is $x\in \operatorname{ri}(X)$ with $g^{(1)}\leq O,g^{(2)}(x)< O$, (a), (b) are satisfied.

\begin{proof}
	Let $P=\{x:g^{(1)}(x)\leq O\}$. We can formulate three primal-dual problems pairs:
	\[
		\begin{matrix}
			(1)                                    &
			\begin{Bmatrix}
				\text{Min.}\;    f(x) \\
				\text{subject to}\;  x\in X,g(x)\leq O
			\end{Bmatrix} & \leftrightarrow                                    &
			\begin{Bmatrix}
				\text{Max.}\;    \inf_{x\in X}f(x)-y^Tg(x) \\
				\text{subject to}\;    y\leq O
			\end{Bmatrix}
			\\
			(2)                                    & \begin{Bmatrix}
				                                         \text{Min.}\;    f(x) \\
				                                         \text{subject to}\;  x\in X\cap P,g^{(2)}(x)\leq O
			                                         \end{Bmatrix} & \leftrightarrow &
			\begin{Bmatrix}
				\text{Max.}\;    \inf_{x\in X\cap P}f(x)-y^{(2),T}g^{(2)}(x) \\
				\text{subject to}\;    y^{(2)}\leq O
			\end{Bmatrix}                               \\
			(3)                                    & \begin{Bmatrix}
				                                         \text{Min.}\;    f(x)-y^{(2),T}_0g^{(2)}(x) \\
				                                         \text{subject to}\;  x\in X,g^{(1)}(x)\leq O
			                                         \end{Bmatrix}     & \leftrightarrow &
			\begin{Bmatrix}
				\text{Max.}\;    \inf_{x\in X}f(x)-y^{(2),T}_0g^{(2)}(x)-y^{(1),T}_0g^{(1)}(x) \\
				\text{subject to}\;    y^{(1)}\leq O
			\end{Bmatrix}
		\end{matrix}
	\]
	where problem 3 has an additional $y^{(2)}_0\leq O$ to be determined. Denote the $f^\ast,q^\ast,Q^\ast$ associated to these problems as $f^{(i),\ast},q^{(i),\ast},Q^{(i),\ast}$. By Slater's condition 1 and condition (b), we have $f^{(1),\ast}=f^{(2),\ast}=q^{(2),\ast}$ and $Q^{(2),\ast}\neq\emptyset$; take $y^{(2)}_0\in Q^{(2),\ast}$. By Slater's condition 2 and condition (a), we have $q^{(2),\ast}=f^{(3),\ast}=q^{(3),\ast}$ and $Q^{(3),\ast}\neq\emptyset$; take $y^{(1)}_0\in Q^{(3),\ast}$, then $(y^{(1)}_0,y^{(2)}_0)\in Q^{(1),\ast}$ with $q^{(1),\ast}\geq q^{(3),\ast}=f^{(1),\ast}$.
\end{proof}

\paragraph{}We now derive Fenchel duality and conic duality as corollaries to the convex programming duality.

\begin{coro}[Fenchel's Duality]\label{coro:044-fenchel}
	Consider the problem:
	\begin{align*}
		\text{Minimize}\quad   & f^{(1)}(x)+f^{(2)}(Ax) \\
		\text{subject to}\quad & x\in \mathbb{R}^m
	\end{align*}
	where $f^{(1)},f^{(2)}$ are closed convex scalar functions on $\mathbb{R}^m,\mathbb{R}^l$ with domains $X^{(1)},X^{(2)}$, $A\in \mathbf{M}_{m,l}(\mathbb{R})$. We can rewrite this problem in a form that is closer to our formulation, and derive its dual:
	\[
		\begin{matrix}
			\begin{Bmatrix}
				\text{Min.}\;    f^{(1)}(x^{(1)})+f^{(2)}(x^{(2)}) \\
				\text{subject to}\;  (x^{(1)},x^{(2)})\in X^{(1)}\times X^{(2)},Ax^{(1)}=x^{(2)}
			\end{Bmatrix} & \leftrightarrow &
			\begin{Bmatrix}
				\text{Min.}\;    f^{(1),\star}(A^Ty)+f^{(2),\star}(-y) \\
				\text{subject to}\;    y\in \mathbb{R}^m
			\end{Bmatrix}
		\end{matrix}
	\]
	\begin{enumerate}[label=(\alph*)]
		\item Slater condtiion 2: if $f^\ast<\infty,A \operatorname{ri}(X_1)\cap \operatorname{ri}(X_2)\neq\emptyset$, then $Q^\ast\neq\emptyset,f^\ast=q^\ast$.
		\item Complementary slackness: given $((x^{(1)}_0,Ax^{(1)}_0),y_0)\in F\times Q$, then $f^\ast=q^\ast$ and $((x^{(1)}_0,Ax^{(1)}_0),y_0)\in F^\ast\times Q^\ast$ iff $x_0^{(1)}\in \underset{x^{(1)}\in \mathbb{R}^l}{\operatorname{argmin}}\left(f^{(1)}(x^{(1)})-(Ax^{(1)})^Ty_0\right)$ and $Ax_0^{(1)}\in \underset{x^{(2)}\in \mathbb{R}^l}{\operatorname{argmin}}\left(f^{(2)}(x^{(2)})-x^{(2),T}y_0\right)$.
	\end{enumerate}
\end{coro}

\begin{coro}[Conic Duality]\label{coro:044-conic}
	Assume $C$ is closed convex cone, $f$ is closed convex proper function on $\mathbb{R}^m$. Define $\iota_C$ as function taking $\infty$ outside $C$ and $0$ on $C$, then the following is a primal-dual pair:
	\[
		\begin{matrix}
			\begin{Bmatrix}
				\text{Min.}\;    f(x) \\
				\text{subject to}\;  x\in C
			\end{Bmatrix} & \leftrightarrow &
			\begin{Bmatrix}
				\text{Min.}\;    f^\star(y) \\
				\text{subject to}\;    y\in -C^\ast
			\end{Bmatrix}
		\end{matrix}
	\]
	by applying Fenchel's duality to $f^{(1)}=f$, $f^{(2)}=\iota_C$, $m=l$, $A=I_m$, noting $f^{(2),\star}=\iota_{C^\ast}$.\\
	Slater condition 2 translates to: if $f^\ast<\infty$, $\operatorname{ri}(X)\cap \operatorname{ri}(C)\neq\emptyset$, then $f^\ast=q^\ast,Q^\ast\neq\emptyset$.
\end{coro}

\section{Theory of Subgradiants (WIP \faWrench)}
\label{sect:045}

\paragraph{}We will define subgradients via MC/MC. For this section, fix a proper convex function $f$ on $\mathbb{R}^m$, write $X=\operatorname{dom}(f)$. In the case where $f$ is differentiable, a property that holds for $\nabla f(x)$ at some $x\in X$ is that the following holds:
\[
	f(z)-f(x)\geq \nabla f(x)^T(z-x),\; \text{ for all }z\in \mathbb{R}^m
\]
which motivates the following definition (written in MC/MC formalism):

\begin{defn}[Subgradients]\label{defn-subgradients}
	For $x\in X$, define $\operatorname{epi}(f)(x) = \operatorname{epi}(f) - (x,f(x))$. Define:
	\[
		\partial f(x) = \left\{
		y\in \mathbb{R}^m,
		\operatorname{epi}(f)(x)\subset E_{(y,-1),0,-}
		\right\}
	\]
	as the "subgradient of $f$ at $x$". Note: since $w^\ast_{\operatorname{epi}(f)(x)}=0$, $\partial f(x)=Q^\ast_{\operatorname{epi}(f)(x)}$ (weak duality). Therefore:
	\[
		\partial f(x) = \left\{y:\inf\left\{(-y,1)^T \operatorname{epi}(f)(x)\right\}=0\right\}=\left\{y:\sup_z \left\{ z^Ty-f(z)\right\}=x^Ty-f(x)\right\}
	\]
\end{defn}

\paragraph{}One can show that when $f$ is differentiable at $x$, $\partial f(x)=\{\nabla f(x)\}$. Also, \Cref{prop:032-optimal-set} translates to:

\begin{prop}[Decomposition of Set of Subgradients]\label{prop:045-set-of-subgrad}
	Write $V=\operatorname{lin}(X)$. For $x\in \operatorname{ri}(X)$, $\partial f(x) = V^\perp + V\cap \partial f(x)$ with $V\cap \partial f(x)$ compact and convex. In particular, if $x\in \operatorname{int}(X)$, $\partial f(x)$ is compact.
\end{prop}

\paragraph{}Subgradients can be classified via conjugate functions. By \Cref{defn:025-conjugate}, we have Fenchel's inequality:
\[
	f(x) + f^\star(y) = f(x) + \sup_{x}(x^Ty-f(x)) \geq x^Ty
\]

\begin{prop}[Subgradients v.s. Conjugate Functions]\label{prop:045-subgrad-conjugate}
	Consider the conditions:
	\begin{enumerate}[label=(\alph*)]
		\item $f(x) + f^\star(y) = x^Ty$
		\item $y\in \partial f(x)$
		\item $x\in \partial f^\star(x)$
	\end{enumerate}
	then (a) and (b) are equivalent, and they are equivalent to (c) if $f$ is closed.
\end{prop}

\begin{proof}
	"(a) iff (b)": use $\partial f(x) = \left\{y:f^\star(y)=x^Ty-f(x)\right\}$. For (c), use (d) of \Cref{prop:025-conjugacy-theorem}.
\end{proof}

\paragraph{}Subgradients of $f$ can be used to infer smootheness conditions of $f$.

\begin{prop}[Subgradients and Smootheness]\label{prop:045-subgradients-and-smoothness}.
	TBD
\end{prop}

\paragraph{}We can now formulate two yogas for subgradients:

\begin{prop}[Yoga of Subgradients]\label{prop:045-yoga-subgradients}.
	\begin{enumerate}[label=(\alph*)]
		\item (Chain Rule) We have $A^T\partial f(Ax)\subset \partial F(x)$.
		\item
	\end{enumerate}
\end{prop}

\begin{proof}
	For (a), use \Cref{prop:045-subgrad-conjugate}: for $y\in \partial f(Ax)$, $F(x)+F^\star(A^Ty)\leq f(Ax)+f^\star(y)=x^TA^Ty$.
\end{proof}


\section{Subgradiants - Revisited via MC/MC}
\label{sect:046}

\paragraph{}We will revist subgradients via MC/MC. Fix a proper convex function $f$ on $\mathbb{R}^m$, write $X=\operatorname{dom}(f)$.

\begin{defn}[Subgradients]\label{defn-subgradients}
	For $x\in X$, define $\operatorname{epi}(f)(x) = \operatorname{epi}(f) - (x,f(x))$. Then we have:
	\[
		\partial f(x) = \left\{
		y\in \mathbb{R}^m,
		\operatorname{epi}(f)(x)\subset E_{(y,-1),0,-}
		\right\}
	\]
	as the "subgradient of $f$ at $x$". Note: since $w^\ast_{\operatorname{epi}(f)(x)}=0$, $\partial f(x)=Q^\ast_{\operatorname{epi}(f)(x)}$ (weak duality). Therefore:
	\begin{align*}
		\partial f(x) & =\left\{y:\inf\left\{(-y,1)^T \operatorname{epi}(f)(x)\right\}=0\right\}                                       \\
		              & =\left\{y:\sup_z \left\{ z^Ty-f(z)\right\}=x^Ty-f(x)\right\}            =\left\{y:f^\star(y)=x^Ty-f(x)\right\}
	\end{align*}
\end{defn}

\begin{prop}[Decomposition of Set of Subgradients]\label{prop:046-set-of-subgrad}
	Write $V=\operatorname{lin}(X)$. For $x\in \operatorname{ri}(X)$, $\partial f(x) = V^\perp + V\cap \partial f(x)$ with $V\cap \partial f(x)\neq\emptyset$ compact and convex. In particular, if $x\in \operatorname{int}(X)$, $\partial f(x)$ is compact.
\end{prop}
\begin{proof}
	By \Cref{prop:032-optimal-set}.
\end{proof}

\paragraph{}Subgradients can be classified via conjugate functions. By \Cref{defn:025-conjugate}, we have Fenchel's inequality:
\[
	f(x) + f^\star(y) = f(x) + \sup_{x}(x^Ty-f(x)) \geq x^Ty
\]

\begin{prop}[Subgradients v.s. Conjugate Functions]\label{prop:046-subgrad-conjugate}
	Consider the conditions:
	\begin{center}
		(a) $f(x) + f^\star(y) = x^Ty$ \quad (b) $y\in \partial f(x)$ \quad (c) $x\in \partial f^\star(x)$
	\end{center}
	then (a) and (b) are equivalent, and they are equivalent to (c) if $f$ is closed.
\end{prop}

\begin{proof}
	"(a) iff (b)": use $\partial f(x) = \left\{y:f^\star(y)=x^Ty-f(x)\right\}$. For (c), use (d) of \Cref{prop:025-conjugacy-theorem}.
\end{proof}

\paragraph{}Subgradients can also be defined as optimization problems: as $d\in\partial f(x)$ iff $x$ minimizes $f(y)-d^Ty$ (over $y$). We will use this to give some yogas of subgradients, and derive a constrained optimality condition.

\begin{prop}[Yoga of Subgradients]\label{prop:046-yoga-subgradients}.
	\begin{enumerate}[label=(\alph*)]
		\item (Chain Rule) Given proper convex $f$ on $\mathbb{R}^m$, $A\in \mathbf{M}_{m,l}(\mathbb{R})$, write $X=\operatorname{dom}(f)$. If $f\circ A$ is proper, then $A^T\partial f(Ax)\subset \partial (f\circ A)(x)$, and "$=$" holds if either $A^{-1}\operatorname{ri}(X)\neq\emptyset$ or $f$ is polyhedral.
		\item (Summation Rule) Given convex proper functions $f_1,\dotsc, f_r,f_{r+1},\dotsc,f_{s}$ each having domain $X_1,\dotsc,X_s$, with $f_1,\dotsc,f_r$ polyhedral, on $\mathbb{R}^m$ such that $g:=\sum_{i=1}^sf_i$ is proper, then $\sum_{i=1}^s\partial f_i(x)\subset \partial g(x)$, and "$=$" holds if $\left(\bigcap_{i=1}^r X_i\right)\cap \left(\bigcap_{i=r+1}^s \operatorname{ri}(X_i)\right)\neq\emptyset$.
	\end{enumerate}
\end{prop}

\begin{proof}
	For (a), to show "$\subset$", use \Cref{prop:046-subgrad-conjugate}: for $y\in \partial f(Ax)$, $F(x)+F^\star(A^Ty)\leq f(Ax)+f^\star(y)=x^TA^Ty$. For "$=$", we use duality: given $d\in \mathbb{R}^l$, consider three problems (with (3) having also its dual):
	\[
		\begin{matrix}
			(1)                             &
			\begin{Bmatrix}
				\text{Min.}\;    f(z) - w^Tz \\
				\text{s.t.}\;  z\in X
			\end{Bmatrix} & (2)                                                & \begin{Bmatrix}
				                                                                     \text{Min.}\;    f(Ay)-d^Ty \\
				                                                                     \text{s.t.}\;  y\in A^{-1}X
			                                                                     \end{Bmatrix} \\
			(3)                             & \begin{Bmatrix}
				                                  \text{Min.}\;   f(z)-d^Ty \\
				                                  \text{s.t.}\;  (y,z)\in \mathbb{R}^l\times X, Ay=z
			                                  \end{Bmatrix} & \leftrightarrow                &
			\begin{Bmatrix}
				\text{Max.}\;    \inf_{(y,z)\in \mathbb{R}^l\times X}f(z)-w^Tz-(d-A^Tw)^Ty \\
				\text{s.t.}\;    w\in \mathbb{R}^m
			\end{Bmatrix}
		\end{matrix}
	\]
	where problem $(1)$ has a vector $w$ to be determined. Write the primal optimal sets, values for problem $(i)$ as $F^{(i),\ast},f^{(i),\ast}$, the dual optimal set, dual optimal value of problem $(3)$ as $q^\ast,Q^\ast$. We have $f^{(2),\ast}=f^{(3),\ast}=q^{\ast},Q^\ast\neq\emptyset$ by \Cref{coro:044-slater-combined}. For any $w\in Q^\ast$, we must have $d=A^Tw$. By taking such $w$ in problem $(1)$, we have $f^{(1),\ast}\leq f^{(2),\ast}$. Note that in this case, $q^\ast=f^{(1),\ast}$ - as $A^Tw=d$, we have:
	\[
		q^\ast = \inf_{(y,z)\in \mathbb{R}^l\times X}f(z)-w^Tz-(d-A^Tw)^Ty =  \inf_{z\in X}f(z)-w^Tz=f^{(1),\ast}
	\]
	giving $f^{(1),\ast}=f^{(2),\ast}$ and $AF^{(2),\ast}\subset F^{(1),\ast}$. Therefore, given $d\in \partial (f\circ A)(x)$, define problem $(1)$ according to any element $w\in Q^\ast$, then $Ax \in AF^{(2),\ast}\subset F^{(1),\ast}$, giving $w\in \partial f(Ax)$, showing $d=A^Tw\in A^T\partial f(Ax)$. For (b), use (a)'s proof with $A:\mathbb{R}^m\to \mathbb{R}^{ms},x\mapsto (x,x,\dotsc,x)$, $f:\mathbb{R}^{ms}\to \overline{\mathbb{R}}$, $(x_i)_{i=1}^s\mapsto \sum_{i=1}^sf_i(x_i)$.
\end{proof}

\begin{prop}[Subgradients and Optimality]\label{prop:046-subgrad-optimality-conditions}
	Let $f$ be proper convex on $\mathbb{R}^m$ with $\operatorname{dom}(f)=X$, and let $C\subset \mathbb{R}^m$ be a convex subset. If any of the following holds:
	\begin{enumerate}[label=(\alph*)]
		\item $\operatorname{ri}(X)\cap \operatorname{ri}(C)\neq\emptyset$
		\item $f$ is polyhedral and $\operatorname{ri}(C)\cap X\neq\emptyset$.
		\item $C$ is polyhedral and $\operatorname{ri}(X)\cap C\neq\emptyset$.
		\item both $f$ and $C$ are polyhedral with $X\cap C\neq\emptyset$
	\end{enumerate}
	then $x\in C$ minimizes $f$ over $C$ iff $O\in\partial f(x)+N_C(x)$, where $N_C(x):=\partial \delta_C(x)=(C-x)^\ast$. That is: the set of subgradients $\partial f(x)$, along with the "normal cone" $N_C(x)$ is enough to determine optimality over $C$.
\end{prop}
\begin{proof}
	This is the same as minimizing $\delta_C+f$, so optimality is the same as $O\in\partial(\delta_C+f)(x)=N_C(x)+\partial f(x)$ (by \Cref{prop:046-yoga-subgradients} and conditions (a)-(d)). That $N_C(x)=(C-x)^\ast$ (polar cone of $C-x$) is easy to show.
\end{proof}

